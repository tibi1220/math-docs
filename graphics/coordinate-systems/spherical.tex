\documentclass[graphics]{math-standalone}

\begin{document}
\tdplotsetmaincoords{55}{110}
\begin{tikzpicture}[
    % 3d view={130}{35.26},
    baseline,
    tdplot_main_coords,
  ]
  % Origin coordinate
  \coordinate (O) at (0,0,0);
  \def\s{1.25}

  % Coordinate system
  \draw[-to] (O) -- ++(1.75,0,0) node[anchor=south] {$x$};
  \draw[-to] (O) -- ++(0,1.75,0) node[anchor=south] {$y$};
  \draw[-to] (O) -- ++(0,0,1.75) node[anchor=north west] {$z$};

  % Helper square
  \draw[gray,dashed]
  (\s,\s,0) coordinate(A) --
  (0,\s,0) coordinate(B) --
  (0,\s,\s) coordinate(C) --
  (\s,\s,\s) coordinate(c) --
  cycle
  ;

  % Helper lines
  \draw[gray]
  (O) -- (C)
  (O) -- (A)
  ;

  % r
  \draw[thick,draw=red-base,-to]
  (O) -- ++(\s,\s,\s)
  node[pos=.5, anchor=south west, font=\scriptsize, inner sep=2pt]{$r$};

  % phi
  \tdplotgetpolarcoords{0.001}{1}{1}
  \tdplotsetthetaplanecoords{\tdplotresphi}

  \tdplotdrawarc[tdplot_rotated_coords, blue-base, thick, -to]
  {(O)}{1}{0}{45}
  {anchor=north east,font=\scriptsize, inner sep=2pt, black}{$\varphi$}

  % theta
  \tdplotdrawarc[yellow-base, thick, -to]
  {(O)}{1}{0}{45}
  {anchor=south,font=\scriptsize, inner sep=2pt, black}{$\vartheta$};
\end{tikzpicture}
\end{document}
