\documentclass[lang=magyar]{math-handout}

\title{Összefoglalás}
\area{Többváltozós analízis}
\subject{Matematika G3}
\date{Utoljára frissítve: \today}
\author{Sándor Tibor}
\docno{7}

\begin{document}
\allowdisplaybreaks

\maketitle

\vspace{1em}

\begin{summary}
  Ebben a leckében összefoglaljuk az eddig tanult fogalmakat.
\end{summary}

\vspace{-1em}
\section{Elméleti áttekinő}
% \vspace{1em}

\vfill

\begin{block}[\underline{Rotáció, Divergencia, Gradiens}]
  \vspace*{-1em}
  \begin{center}
    \def\arraystretch{1.5}
    \newenvironment{bm}{\bgroup\renewcommand*{\arraystretch}{1.1}\begin{bmatrix}}{\end{bmatrix}\egroup}
    \newcommand{\dspl}[3]{\begin{bm}#1\\#2\\#3\end{bm}}
    \newcommand\nablavec{\dspl{\partial_x}{\partial_y}{\partial_z}}

    \begin{tabular}{*{3}{>{\centering\arraybackslash}p{3.5cm}}}
      \def\arraystretch{1}
      % &
      \bfseries Rotáció
       & \bfseries Divergencia
       & \bfseries Gradiens
      \\
      \hline
      % Jelölés & 
      $\rot \rvec v$
       & $\Div \rvec v$
       & $\grad \varphi$
      \\
      % Operátor & 
      $\nabla \times \rvec v$
       & $\scalar{\nabla}{\rvec v}$
       & $\nabla \cdot \varphi$
      \\
      % Számítás &
      $\nablavec \times \dspl{v_x}{v_y}{v_z}$
       & $\scalar{\nablavec}{\dspl{v_x}{v_y}{v_z}}$
       & $\dspl{\partial_x \varphi}{\partial_y \varphi}{\partial_z \varphi}$
      \\
      % Ért. tart. & 
      $\mathcal D_{\rvec v} = \mathbb R^3$
       & $\mathcal D_{\rvec v} = \mathbb R^3$
       & $\mathcal D_{\varphi} = \mathbb R^3$
      \\
      % Ért. készl. &
      $\mathcal R_{\rvec v} = \mathbb R^3$
       & $\mathcal R_{\rvec v} = \mathbb R^3$
       & $\mathcal R_{\varphi} = \mathbb R$
      \\
      % Ért. készl. &
      $\mathcal R_{\rot \rvec v} = \mathbb R^3$
       & $\mathcal R_{\Div \rvec v} = \mathbb R$
       & $\mathcal R_{\grad \varphi} = \mathbb R^3$
      \\
      % \hline
    \end{tabular}
  \end{center}
\end{block}

\vfill

\begin{block}[\underline{Integrálszámítás}]
  \vspace*{-2em}
  \begin{align*}
    \int_\gamma \varphi(\rvec r) \, \diff s
     & =
    \int_a^b \varphi(\rvec r(t)) \norma{\dot{\rvec r}(t)} \, \diff t
    \\
    \int_\gamma \scalar{\rvec v(\rvec r)}{\diff \rvec r}
     & =
    \int_a^b \scalar{\rvec v(\rvec r(t))}{\dot{\rvec r}(t)} \diff t
    \\
    \int_\gamma \rvec v(\rvec r) \times \diff \rvec r
     & =
    \int_a^b \rvec v(\rvec r(t)) \times \dot{\rvec r}(t) \, \diff t
    \\[3mm]
    \int_S \varphi (\rvec r) \, \diff S
     & =
    \iint_T \varphi (\rvec \varrho(s; t))
    \norma{\pdv{\rvec \varrho}{s} \times \pdv{\rvec \varrho}{t}}
    \, \diff s \, \diff t
    \\
    \int_S \scalar{\rvec v (\rvec r)}{\diff \rvec S}
     & =
    \iint_T \scalar
    {\rvec v (\rvec \varrho(s; t))}
    {\left(\pdv{\rvec \varrho}{s} \times \pdv{\rvec \varrho}{t}\right)}
    \, \diff s \, \diff t
    \\
    \int_S \rvec v (\rvec r) \times \diff \rvec S
     & =
    \iint_T {\rvec v (\rvec \varrho(s; t))} \times
    \left(\pdv{\rvec \varrho}{s} \times \pdv{\rvec \varrho}{t}\right)
    \, \diff s \, \diff t
    \\[3mm]
    \int_V \varphi (\rvec r) \, \diff V
     & =
    \iiint_D \varphi (\rvec Z(t; u; v))
    \left\langle\pdv{\rvec Z}{t}\pdv{\rvec Z}{u}\pdv{\rvec Z}{v}\right\rangle
    \, \diff t \, \diff u \, \diff v
  \end{align*}
\end{block}

\clearpage
\newcommand{\bitem}[1]{\bullet\; \textbf{{#1}}}

\begin{block}[\underline{Azonosságok, Potenciálosság}]
  \bitem{Skalárpotenciálosság:}

  Egy $\rvec v: V \rightarrow V$ vektormező skalárpotenciálos, ha létezik olyan
  $\varphi: V \rightarrow \mathbb R$ skalármező, hogy $\rvec v = \grad \varphi$.
  \[
    \rot \grad \varphi \equiv \nvec
  \]

  \bitem{Vektorpotenciálosság:}

  Egy $\rvec v: V \rightarrow V$ vektormező vektorpotenciálos, ha létezik olyan
  $\rvec u: V \rightarrow V$ vektormező, hogy $\rvec v = \rot \rvec u$.
  \[
    \Div \rot \rvec v \equiv 0
  \]

  \bitem{További azonosságok:}
  \begin{align*}
    \grad \left( \lambda \, \varphi + \mu \, \psi \right)
     & =
    \lambda \grad \varphi + \mu \grad \psi
    \\
    \Div \left( \lambda \, \rvec v + \mu \, \rvec w \right)
     & =
    \lambda \Div \rvec v + \mu \Div \rvec w
    \\
    \rot \left( \lambda \, \rvec v + \mu \, \rvec w \right)
     & =
    \lambda \rot \rvec v + \mu \rot \rvec w
    \\[
    5mm
    ]
    \grad \left( \varphi \, \psi \right)
     & =
    \varphi \, \grad \psi + \psi \, \grad \varphi
    \\
    \Div \left( \varphi \, \rvec v \right)
     & =
    \varphi \, \Div \rvec v + \scalar{\grad \varphi}{\rvec v}
    \\
    \rot \left( \varphi \, \rvec v \right)
     & =
    \varphi \, \rot \rvec v + \grad \varphi \times \rvec v
    \\[
    5mm
    ]
    \rot \rot \rvec v
     & =
    \grad \Div \rvec v - \Delta \rvec v
    \\
    \rot \left( \rvec u \times \rvec v \right)
     & =
    \rvec u \, \Div \rvec v - \rvec v \, \Div \rvec u
    + (\Diff \rvec u) \rvec v - (\Diff \rvec v) \rvec u
    \\
    \Div \left( \rvec u \times \rvec v \right)
     & =
    \scalar{\rvec v}{\rot \rvec u} - \scalar{\rvec u}{\rot \rvec v}
    \\
    \grad \left( \scalar{\rvec u}{\rvec v} \right)
     & =
    \rvec u \times \rot \rvec v + \rvec v \times \rot \rvec u
    + (\Diff \rvec u) \rvec v + (\Diff \rvec v) \rvec u
  \end{align*}
\end{block}

\vfill

\begin{block}[\underline{Integrálási tételek}]
  \bitem{Gradiens-tétel:}
  \[
    \int_\gamma \scalar{\grad \varphi}{\diff \rvec r}
    =
    \varphi(\rvec r(b)) - \varphi(\rvec r(a))
  \]
  Vagyis ha egy vektormező előáll egy skalármező gradienseként, akkor annak
  bármely zárt görbe mentén vett integrálja csak a kezdő- és végpontoktól függ.

  \bitem{Stokes-tétel:}
  \[
    \int_S \scalar{\rot \rvec v}{\diff \rvec S}
    =
    \oint_{\partial S} \scalar{\rvec v}{\diff \rvec r}
  \]
  A tételből következik, hogy skalárpotenciálos vektormező bármely zárt görbén
  vett integrálja zérus.

  \bitem{Gauss-Osztogradszkij-tétel:}
  \[
    \int_V \Div \rvec v \, \diff V
    =
    \oint_{\partial V} \scalar{\rvec v}{\diff \rvec S}
  \]
  A tételből következik, hogy vektorpotenciálos vektormező bármely zárt
  felületre vett integrálja zérus.
\end{block}

\end{document}
