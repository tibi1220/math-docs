\documentclass[lang=magyar]{math-handout}

\title{Felületmenti integrálok}
\area{Többváltozós analízis}
\subject{Matematika G3}
\date{Utoljára frissítve: \today}
\author{Sándor Tibor}
\docno{4}

\begin{document}
\allowdisplaybreaks

\maketitle

\vspace{1em}

\begin{summary}
  Ebben a fejezetben megismerkedünk a felületmenti integrálok fogalmával, és
  azzal, hogy hogyan számíthatjuk ki őket.
\end{summary}

\vspace{-1em}

\section{Elméleti áttekinő}

\vspace{1em}

\begin{definition}[Reguláris felület]
  Legyen $S \subseteq \mathbb R^3$. Azt mondjuk, hogy az $S$ reguláris felület,
  ha $\forall \, p \in S$ ponthoz megadható olyan $p$-t tartalmazó $V \subset
    \mathbb R^3$ nyílt halmaz és $\rvec \varrho : U \subseteq \mathbb R^2
    \rightarrow S \cap V$ leképezés, melyre teljesülnek az alábbiak:
  \begin{itemize}
    \item $\rvec \varrho$ differenciálható homeomorfizmus,

    \item $\rvec \varrho$ immerzió (derivált leképezése injektív).
  \end{itemize}
  Ha ezek teljesülnek, akkor $\rvec \varrho$-t parametrációnak,
  $V \cap S$-t  koordinátakörnyezetnek nevezzük.
\end{definition}

\vfill

\begin{definition}[Elemi felület]
  A $\rvec \varrho: U \subseteq \mathbb R^2 \rightarrow S \subseteq \mathbb R^3$
  elemi felület, ha $\rvec \varrho$ legalább egyszer differenciálható és
  injektív.
\end{definition}

\vfill

\begin{definition}[Skalármező skalárérékű felületmenti integrálja]
  Legyen $\varphi: D \subseteq \mathbb R^3 \rightarrow \mathbb R$ leképezés,
  $\rvec \varrho: U \subseteq \mathbb R^2 \rightarrow S \subset D$, $t \mapsto
    \rvec \varrho(t)$ pedig az $S$ felület parametrizált egyenlete. Ekkor a
  $\varphi$ skalármező $S$ felület menti skalárértékű integrálja:
  \[
    \int_S \varphi (\rvec r) \, \diff S =
    \iint_U \varphi (\rvec \varrho(s; t))
    \norma{\pdv{\rvec \varrho}{s} \times \pdv{\rvec \varrho}{t}}
    \, \diff s \, \diff t
    \text.
  \]
  Ha a felület $z = \varPhi(x; y)$ implicit alakban van:
  \[
    \int_S \varphi (\rvec r) \, \diff S =
    \iint_U \varphi (x; y; \varPhi(x;y))
    \sqrt{1 + (\partial_x \varPhi)^2 + (\partial_y \varPhi)^2}
    \, \diff x \, \diff y
    \text.
  \]
\end{definition}

\vfill

\begin{definition}[Vektormező skalár- és vektorértékű felületmenti integrálja]
  Legyen $\rvec v: D \subseteq \mathbb R^3 \rightarrow \mathbb R^3$ leképezés,
  $\rvec \varrho: U \subseteq \mathbb R^2 \rightarrow S \subset D$, $(s;t)
    \mapsto \rvec \varrho(s;t)$ pedig az $S$ felület parametrizált egyenlete.
  Ekkor az $\rvec v$ vektormező $S$ felület menti
  \begin{itemize}
    \item skalárértékű integrálja:
          \(
            \displaystyle
            \int_S \scalar{\rvec v (\rvec r)}{\diff \rvec S} =
            \iint_D \scalar
            {\rvec v (\rvec \varrho(s; t))}
            {\left(\pdv{\rvec \varrho}{s} \times \pdv{\rvec \varrho}{t}\right)}
            \, \diff s \, \diff t
            \text,
          \)
    \item vektorértékű integrálja:
          \(
            \displaystyle
            \int_S \rvec v (\rvec r) \times \diff \rvec S =
            \iint_D {\rvec v (\rvec \varrho(s; t))} \times
            \left(\pdv{\rvec \varrho}{s} \times \pdv{\rvec \varrho}{t}\right)
            \, \diff s \, \diff t
            \text.
          \)

  \end{itemize}
\end{definition}

\clearpage
\section{Feladatok}

\clearpage
\section{Segédlet}

\subsection{Felületek paraméterezése}

\bgroup
\def\tskip{15mm}
\begin{tabular}{
  >{\bullet\;}p{3.5cm}
  p{5cm}
  m{2.75cm}
  m{4cm}
  }
  \textbf{Körlap:} \newline \phantom{1} ($xy$ sík)
   & $\rvec \varrho (s;t) = \begin{bmatrix} s \cos t \\ s \sin t \\ 0 \end{bmatrix}$
   & $s \in [0;r]$ \newline $t \in [0, 2\pi)$
   & \relativestandalone{../../../graphics/surface-parametrization/circle}
  \\[\tskip]
    \textbf{Ellipszislap:} \newline \phantom{1} ($xy$ sík)
   & $\rvec \varrho (s;t) = \begin{bmatrix} a \, s \cos t \\ b \, s \sin t \\ 0 \end{bmatrix}$
   & $s \in [0;1]$ \newline $t \in [0, 2\pi)$
   & \relativestandalone{../../../graphics/surface-parametrization/ellipse}
  \\[\tskip]
  \textbf{Hengerfelület:}
   & $\rvec \varrho (s;t) = \rvec r_0(s) + t \rvec n$
   & $s \in \mathcal D_{\rvec r_0}$ \newline $t \in [0, T]$
   & \relativestandalone{../../../graphics/surface-parametrization/cylinder}
  \\[\tskip]
  \textbf{Forgásfelület:} \newline \phantom{1} ($z$ tengely körül) \newline \phantom{1} ($z = f(x)$)
   & $\rvec \varrho (s;t) = \begin{bmatrix} s \cos t \\ s \sin t \\ f(s) \end{bmatrix}$
   & $s \in [0;2\pi)$ \newline $t \in \mathcal D_f$
   & \relativestandalone{../../../graphics/surface-parametrization/revolution}
  \\[\tskip]
    \textbf{Gömbfelület:}
   & $\rvec \varrho (s;t) = \begin{bmatrix} R \sin s \cos t \\ R \sin s \sin t \\ R \cos s \end{bmatrix}$
   & $s \in [0;\pi]$ \newline $t \in [0, 2\pi)$
   & \relativestandalone{../../../graphics/surface-parametrization/sphere}
  \\[\tskip]
  \textbf{Ellipszoid:}
   & $\rvec \varrho (s;t) = \begin{bmatrix} a \sin s \cos t \\ b \sin s \sin t \\ c \cos s \end{bmatrix}$
   & $s \in [0;\pi]$ \newline $t \in [0, 2\pi)$
   & \relativestandalone{../../../graphics/surface-parametrization/ellipsoid}
  \\[\tskip]
    \textbf{Тórusz:}
   & $\rvec \varrho (s;t) = \begin{bmatrix} (R + r \cos s) \cos t \\ (R + r \cos s) \sin t \\ r \sin s \end{bmatrix}$
   & $s \in [0;2\pi)$ \newline $t \in [0, 2\pi)$
   & \relativestandalone{../../../graphics/surface-parametrization/torus}
  \\[\tskip]
  \textbf{Kúp:}
   & $\rvec \varrho (s;t) = \begin{bmatrix} s \cos t \\ s \sin t \\ s \end{bmatrix}$
   & $s \in [0;U]$ \newline $t \in [0, 2\pi)$
   & \relativestandalone{../../../graphics/surface-parametrization/cone}
\end{tabular}
\egroup

\end{document}
