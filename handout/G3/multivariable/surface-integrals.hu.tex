\documentclass[lang=magyar]{math-handout}

\title{Felületmenti integrálok}
\area{Többváltozós analízis}
\subject{Matematika G3}
\date{Utoljára frissítve: \today}
\author{Sándor Tibor}
\docno{4}

\begin{document}
\allowdisplaybreaks

\maketitle

\vspace{1em}

\begin{summary}
  Ebben a fejezetben megismerkedünk a felületmenti integrálok fogalmával, és
  azzal, hogy hogyan számíthatjuk ki őket.
\end{summary}

\vspace{-1em}

\section{Elméleti áttekinő}

\vspace{1em}

\begin{definition}[Reguláris felület]
  Legyen $S \subset  \mathbb R^3$. Azt mondjuk, hogy az $S$ reguláris felület,
  ha $\forall \, p \in S$ ponthoz megadható olyan $p$-t tartalmazó $V \subset
    \mathbb R^3$ nyílt halmaz és $\rvec r : U \subset \mathbb R^2 \rightarrow S
    \cap V$ leképezés, melyre teljesülnek az alábbiak:
  \begin{itemize}
    \item $\rvec r$ differenciálható homeomorfizmus,

    \item $\rvec r$ immerzió (derivált leképezése injektív).
  \end{itemize}
  Ha ezek teljesülnek, akkor $\rvec r$-t parametrációnak,
  $V \cap S$-t  koordinátakörnyezetnek nevezzük.
\end{definition}

\vfill

\begin{definition}[Elemi felület]
  Egy $S: T \subseteq \mathbb R^2 \rightarrow \mathbb R^3$ elemi felület, ha
  $S$ legalább egyszer differenciálható és injektív.
\end{definition}

\vfill

\begin{definition}[Skalármező skalárérékű felületmenti integrálja]
  Legyen $\varphi: U \subseteq \mathbb R^3 \rightarrow \mathbb R$ leképezés,
  $\rvec r: T \subseteq \mathbb R^2 \rightarrow S \subset U$, $t \mapsto
    \rvec r(t)$ pedig az $S$ felület parametrizált egyenlete. Ekkor az $\varphi$
  skalármező $S$ felület menti skalárértékű integrálja:
  \[
    \int_S \varphi (\rvec r) \, dS =
    \iint_T \varphi (\rvec r(s; t))
    \norma{\pdv{\rvec r}{s} \times \pdv{\rvec r}{t}}
    \, \diff s \, \diff t
    \text.
  \]
  Ha a felület $z = \varPhi(x; y)$ implicit alakban van:
  \[
    \int_S \varphi (\rvec r) \, dS =
    \iint_T \varphi (x; y; \varPhi(x;y))
    \sqrt{1 + (\partial_x \varPhi)^2 + (\partial_y \varPhi)^2}
    \, \diff x \, \diff y
    \text.
  \]
\end{definition}

\vfill

\begin{definition}[Vektormező skalár- és vektorértékű felületmenti integrálja]
  Legyen $\rvec v:  U \subseteq \mathbb R^3 \rightarrow \mathbb R^3$ leképezés,
  $\rvec r: T \subseteq \mathbb R^2 \rightarrow S \subset U$, $(s;t) \mapsto
    \rvec r(s;t)$ pedig az $S$ felület parametrizált egyenlete. Ekkor az
  $\rvec r$ vektormező $S$ felület menti
  \begin{itemize}
    \item skalárértékű integrálja:
          \(
            \displaystyle
            \int_S \scalar{\rvec v (\rvec r)}{d \rvec S} =
            \iint_T \scalar
            {\rvec v (\rvec r(s; t))}
            {\left(\pdv{\rvec r}{s} \times \pdv{\rvec r}{t}\right)}
            \, \diff s \, \diff t
            \text,
          \)
    \item vektorértékű integrálja:
          \(
            \displaystyle
            \int_S \rvec v (\rvec r) \times d \rvec S =
            \iint_T {\rvec v (\rvec r(s; t))} \times
            \left(\pdv{\rvec r}{s} \times \pdv{\rvec r}{t}\right)
            \, \diff s \, \diff t
            \text.
          \)

  \end{itemize}
\end{definition}

\clearpage
\section{Feladatok}

\end{document}
