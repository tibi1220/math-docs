\documentclass[lang=magyar]{math-handout}

\title{Felületmenti integrálok}
\area{Többváltozós analízis}
\subject{Matematika G3}
\date{Utoljára frissítve: \today}
\author{Sándor Tibor}
\docno{4}

\begin{document}
\allowdisplaybreaks

\maketitle

\vspace{1em}

\begin{summary}
  Ebben a fejezetben megismerkedünk a felületmenti integrálok fogalmával, és
  azzal, hogy hogyan számíthatjuk ki őket.
\end{summary}

\vspace{-1em}

\section{Elméleti áttekinő}

\vspace{1em}

\begin{definition}[Reguláris felület]
  Legyen $S \subset  \mathbb R^3$. Azt mondjuk, hogy az $S$ reguláris felület,
  ha $\forall \, p \in S$ ponthoz megadható olyan $p$-t tartalmazó $V \subset
    \mathbb R^3$ nyílt halmaz és $\rvec \varrho : U \subset \mathbb R^2
    \rightarrow S \cap V$ leképezés, melyre teljesülnek az alábbiak:
  \begin{itemize}
    \item $\rvec \varrho$ differenciálható homeomorfizmus,

    \item $\rvec \varrho$ immerzió (derivált leképezése injektív).
  \end{itemize}
  Ha ezek teljesülnek, akkor $\rvec \varrho$-t parametrációnak,
  $V \cap S$-t  koordinátakörnyezetnek nevezzük.
\end{definition}

\vfill

\begin{definition}[Elemi felület]
  Egy $S: T \subseteq \mathbb R^2 \rightarrow \mathbb R^3$ elemi felület, ha
  $S$ legalább egyszer differenciálható és injektív.
\end{definition}

\vfill

\begin{definition}[Skalármező skalárérékű felületmenti integrálja]
  Legyen $\varphi: U \subseteq \mathbb R^3 \rightarrow \mathbb R$ leképezés,
  $\rvec \varrho: T \subseteq \mathbb R^2 \rightarrow S \subset U$, $t \mapsto
    \rvec \varrho(t)$ pedig az $S$ felület parametrizált egyenlete. Ekkor a
  $\varphi$ skalármező $S$ felület menti skalárértékű integrálja:
  \[
    \int_S \varphi (\rvec r) \, dS =
    \iint_T \varphi (\rvec \varrho(s; t))
    \norma{\pdv{\rvec \varrho}{s} \times \pdv{\rvec \varrho}{t}}
    \, \diff s \, \diff t
    \text.
  \]
  Ha a felület $z = \varPhi(x; y)$ implicit alakban van:
  \[
    \int_S \varphi (\rvec r) \, dS =
    \iint_T \varphi (x; y; \varPhi(x;y))
    \sqrt{1 + (\partial_x \varPhi)^2 + (\partial_y \varPhi)^2}
    \, \diff x \, \diff y
    \text.
  \]
\end{definition}

\vfill

\begin{definition}[Vektormező skalár- és vektorértékű felületmenti integrálja]
  Legyen $\rvec v:  U \subseteq \mathbb R^3 \rightarrow \mathbb R^3$ leképezés,
  $\rvec \varrho T \subseteq \mathbb R^2 \rightarrow S \subset U$, $(s;t)
    \mapsto \rvec \varrho(s;t)$ pedig az $S$ felület parametrizált egyenlete.
  Ekkor az $\rvec v$ vektormező $S$ felület menti
  \begin{itemize}
    \item skalárértékű integrálja:
          \(
            \displaystyle
            \int_S \scalar{\rvec v (\rvec r)}{d \rvec S} =
            \iint_T \scalar
            {\rvec \varrho (\rvec r(s; t))}
            {\left(\pdv{\rvec \varrho}{s} \times \pdv{\rvec \varrho}{t}\right)}
            \, \diff s \, \diff t
            \text,
          \)
    \item vektorértékű integrálja:
          \(
            \displaystyle
            \int_S \rvec v (\rvec r) \times d \rvec S =
            \iint_T {\rvec \varrho (\rvec r(s; t))} \times
            \left(\pdv{\rvec \varrho}{s} \times \pdv{\rvec \varrho}{t}\right)
            \, \diff s \, \diff t
            \text.
          \)

  \end{itemize}
\end{definition}

\clearpage
\section{Feladatok}

\clearpage
\section{Segédlet}

\subsection{Felületek paraméterezése}

\bgroup
\newcommandx{\coordsyst}[1][1=1.25]{%
  \coordinate (O) at (0,0,0);
  \draw[-to] (O) -- (#1,0,0) node[above left=-.75mm] {$x$};
  \draw[-to] (O) -- (0,#1,0) node[below left=-.75mm] {$y$};
  \draw[-to] (O) -- (0,0,#1) node[right] {$z$};
}
\def\tscale{1.4}
\let\tsize\normalsize
\def\tskip{14mm}
\begin{tabular}{>{\bullet\;\bfseries}p{3.25cm}<{:} p{5cm} m{2.75cm} c}
  Körlap
   & $\rvec \varrho (s;t) = \begin{bmatrix} s \cos t \\ s \sin t \\ 0 \end{bmatrix}$
   & $s \in [0;r]$ \newline $t \in [0, 2\pi)$
   & \begin{tikzpicture}[font=\tsize, baseline, scale=\tscale]
           \draw[-to] (-1.25,0) -- (1.25,0) node[above left=-.75mm] {$x$};
           \draw[-to] (0,-.9) -- (0,1.1) node[below left=-.75mm] {$y$};
           \draw[to-to, thick, blue-base] (0,0) circle (.6);
           \node at (.3,.15) {\scriptsize$r$};
           \node at (-.15,.3) {\scriptsize$r$};
         \end{tikzpicture}
  \\[\tskip]
    Ellipszislap
   & $\rvec \varrho (s;t) = \begin{bmatrix} a \, s \cos t \\ b \, s \sin t \\ 0 \end{bmatrix}$
   & $s \in [0;1]$ \newline $t \in [0, 2\pi)$
   & \begin{tikzpicture}[font=\tsize, baseline, scale=\tscale]
         \draw[-to] (-1.25,0) -- (1.25,0) node[above left=-.75mm] {$x$};
         \draw[-to] (0,-.8) -- (0,1) node[below left=-.75mm] {$y$};
         \draw[to-to, thick, blue-base] (0,0) ellipse (.8 and .6);
         \node at (.4,.15) {\scriptsize$a$};
         \node at (-.15,.3) {\scriptsize$b$};
       \end{tikzpicture}
  \\[\tskip]
  Hengerfelület
   & $\rvec \varrho (s;t) = \rvec r_0(s) + t \rvec n$
   & $s \in \mathcal D_{\rvec r_0}$ \newline $t \in [0, T]$
   & \begin{tikzpicture}[font=\tsize, baseline, scale=\tscale]
       \coordinate (A) at (0,-.15);
       \coordinate (B) at (.6,-.2);
       \coordinate (C) at (.5,.5);
       \coordinate (D) at (0,.35);
       \coordinate (E) at (-.66,.45);
       \coordinate (F) at (-.75,-.35);

       \foreach \c in {A,B,C,D,E,F} {
           \coordinate (\c-) at ($(\c) + (.5,.5)$);
           \coordinate (\c+) at ($(\c) - (.375,.375)$);

           \draw[opacity=.1] (\c-) -- (\c+) coordinate[pos=.25] (\c75);
         }


       \draw[smooth cycle, thick, blue-base, fill=white] plot coordinates {
           (A+) (B+) (C+) (D+) (E+) (F+)
         };
       \draw[smooth cycle, thick, yellow-base, opacity=.25] plot[xshift=1cm] coordinates {
           (A-) (B-) (C-) (D-) (E-) (F-)
         };

       \draw[-to, draw=red-base, thick] (E+) -- (E75) node[below left, xshift=-3mm] {\scriptsize$\rvec n$};
       \node[above left=-1mm] at (A+) {\scriptsize$\rvec r_0(s)$};
     \end{tikzpicture}
  \\[\tskip]
  Forgásfelület
   & $\rvec \varrho (s;t) = \begin{bmatrix} s \cos t \\ s \sin t \\ f(s) \end{bmatrix}$
   & $s \in [0;2\pi)$ \newline $t \in \mathcal D_f$
   & \begin{tikzpicture}[font=\tsize, baseline, scale=\tscale]
           \draw[-to] (-1.25,-.4) -- ++(2.5,0) node[above left=-.75mm] {$x$};
           \draw[-to] (0,-.8) -- (0,1) node[below left=-.75mm] {$z$};

           \draw [thick,draw=blue-base, yshift=-4mm]
           plot [domain=-1:1, samples=100]
           (\x,1.4*\x*\x - 0.6*\x*\x*\x*\x);

           \node at (0,0) {\scriptsize $\;\;z = f(x)$};

           \draw[to-to, thick, blue-base] (0,.4) ellipse (1 and .2);
         \end{tikzpicture}
  \\[\tskip]
    Gömbfelület
   & $\rvec \varrho (s;t) = \begin{bmatrix} R \sin s \cos t \\ R \sin s \sin t \\ R \cos s \end{bmatrix}$
   & $s \in [0;\pi]$ \newline $t \in [0, 2\pi)$
   &
  \\[\tskip]
  Ellipszoid
   & $\rvec \varrho (s;t) = \begin{bmatrix} a \sin s \cos t \\ b \sin s \sin t \\ c \cos s \end{bmatrix}$
   & $s \in [0;\pi]$ \newline $t \in [0, 2\pi)$
   &
  \\[\tskip]
    Тórusz
   & $\rvec \varrho (s;t) = \begin{bmatrix} (R + r \cos s) \cos t \\ (R + r \cos s) \sin t \\ r \sin s \end{bmatrix}$
   & $s \in [0;2\pi)$ \newline $t \in [0, 2\pi)$
   & \begin{tikzpicture}[font=\tsize, baseline, scale=\tscale]
         \draw[-to] (-1.25,0) -- ++(2.5,0) node[above left=-.75mm] {$x$};
         \draw[-to] (-.6,-.6) -- ++(0,1.2) node[below left=-.75mm] {$z$};

         \draw[gray] (.65,0) coordinate(C) -- ++(0,-.5) coordinate[pos=.8] (A);
         \coordinate (O) at (-.6,0);

         \draw [thick,draw=blue-base] (.65, 0) circle (.25);
         \draw [thick,fill=red-base, draw=red-base] (.65, 0) circle (.04);

         \draw[to-to, draw=yellow-base, thick] (A) -- (A -| O) node[midway, above] {\scriptsize$R$};
         \draw[draw=yellow-base, thick, -to] (C) -- ++(150:.6) -- ++(150:-.35) node[pos=.75, above] {\scriptsize$r$};
       \end{tikzpicture}
  \\[\tskip]
  Kúp
   & $\rvec \varrho (s;t) = \begin{bmatrix} s \cos t \\ s \sin t \\ s \end{bmatrix}$
   & $s \in [0;U]$ \newline $t \in [0, 2\pi)$
   & \begin{tikzpicture}[font=\tsize, baseline, scale=\tscale]
       \draw[-to] (-1.25,-.6) -- ++(2.5,0) node[above left=-.75mm] {$x$};
       \draw[-to] (0,-.8) -- (0,1) node[below left=-.75mm] {$z$};

       \draw[thick,draw=blue-base] (-1,.4) -- ++(1,-1) -- ++(1,1);

       \draw[to-to, thick, blue-base] (0,.4) ellipse (1 and .15);

       \node at (0,0) {\scriptsize $z = x$};
     \end{tikzpicture}
\end{tabular}
\egroup

\end{document}
