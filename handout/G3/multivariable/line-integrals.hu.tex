\documentclass[lang=magyar]{math-handout}

\title{Vonalmenti integrálok}
\area{Többváltozós analízis}
\subject{Matematika G3}
\date{Utoljára frissítve: \today}
\author{Sándor Tibor}
\docno{3}

\begin{document}
\allowdisplaybreaks

\maketitle

\vspace{1em}

\begin{summary}
  Ebben a fejezetben megismerkedünk a vonalmenti integrálok fogalmával, és
  azzal, hogy hogyan számíthatjuk ki őket.
\end{summary}

\vspace{-1em}

\section{Elméleti áttekinő}

\vspace{1em}

\begin{definition}[Reguláris görbe]
  Legyen $I \subset \mathbb R$ nem feltétlenül korlátos intervallum. Ekkor az
  $\rvec r : I \rightarrow \mathbb R^3$ immerziót reguláris görbének nevezzük.
\end{definition}

\begin{definition}[Pályasebesség, Ívhossz]
  A $v: I \rightarrow \mathbb R, t \mapsto \norma{\dot{\rvec r}(t)}$
  függvényt pályasebességnek hívjuk.

  A pályasebesség $I$ feletti integrálját a görbe ívhosszának nevezzük:
  \[
    L(\rvec r) = \int_I \norma{\dot{\rvec r}(t)} \, \diff t
    \text.
  \]

\end{definition}

\vfill

\begin{definition}[Irányított görbe]
  Egy $\rvec r: [a;b] \rightarrow \mathbb R^3$ görbe irányított, ha adott egy
  rendezés ($\leq$) a paraméterértékeken. Ekkor $t_1 < t_2$ esetén
  $\rvec r(t_1)$ a görbe korábbi pontja, $\rvec r(t_2)$-höz képest. Ha
  $\rvec r(a) = \rvec r(b)$, akkor a görbe zárt.
\end{definition}

\vfill

\begin{definition}[Skalármező görbe menti skalárértékű integrálja]
  Legyen $\varphi: U \subseteq \mathbb R^3 \rightarrow \mathbb R$ leképezés,
  $\rvec r: [a;b] \rightarrow \gamma \subset U, t \mapsto \rvec r(t)$ pedig a
  $\gamma$ görbe parametrizált egyenlete. Ekkor az $\varphi$ skalármező $\gamma$
  görbe menti skalárértékű integrálja:
  \[
    \int_\gamma \varphi(\rvec r) \, \diff s =
    \int_a^b \varphi(\rvec r(t)) \norma{\dot{\rvec r}(t)} \, \diff t
    \text.
  \]
\end{definition}

\vfill

\begin{definition}[Vektormező görbe menti skalár- és vektorértékű integrálja]
  Legyen $\rvec v: U \subseteq \mathbb R^3 \rightarrow \mathbb R^3$ vektormező,
  $\rvec r: [a;b] \rightarrow \gamma \subset U, t \mapsto \rvec r(t)$ pedig a
  $\gamma$ görbe parametrizált egyenlete. Ekkor az $\rvec v$ vektormező $\gamma$
  görbe menti
  \begin{itemize}
    \item skalárértékű integrálja:
          \(
            \displaystyle
            \int_\gamma \scalar{\rvec v(\rvec r)}{\diff \rvec r} =
            \int_a^b \scalar{\rvec v(\rvec r(t))}{\dot{\rvec r}(t)} \diff t
            \text,
          \)
    \item vektorértékű integrálja:
          \(
            \displaystyle
            \int_\gamma \rvec v(\rvec r) \times \diff \rvec r =
            \int_a^b \rvec v(\rvec r(t)) \times \dot{\rvec r}(t) \, \diff t
            \text.
          \)
  \end{itemize}
\end{definition}

\clearpage
\section{Feladatok}

\end{document}
