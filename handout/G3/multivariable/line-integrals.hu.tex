\documentclass[lang=magyar]{math-handout}

\title{Vonalmenti integrálok}
\area{Többváltozós analízis}
\subject{Matematika G3}
\date{Utoljára frissítve: \today}
\author{Sándor Tibor}
\docno{3}

\begin{document}
\allowdisplaybreaks

\maketitle

\vspace{1em}

\begin{summary}
  Ebben a fejezetben megismerkedünk a vonalmenti integrálok fogalmával, és
  azzal, hogy hogyan számíthatjuk ki őket.
\end{summary}

\vspace{-1em}

\section{Elméleti áttekinő}

\vspace{1em}

\begin{definition}[Reguláris görbe]
  Legyen $I \subset \mathbb R$ nem feltétlenül korlátos intervallum. Ekkor az
  $\rvec r : I \rightarrow \mathbb R^3$ immerziót reguláris görbének nevezzük.
\end{definition}

\begin{definition}[Pályasebesség, Ívhossz]
  A $v: I \rightarrow \mathbb R, t \mapsto \norma{\dot{\rvec r}(t)}$
  függvényt pályasebességnek hívjuk.

  A pályasebesség $I$ feletti integrálját a görbe ívhosszának nevezzük:
  \[
    L(\rvec r) = \int_I \norma{\dot{\rvec r}(t)} \, \diff t
    \text.
  \]

\end{definition}

\vfill

\begin{definition}[Irányított görbe]
  Egy $\rvec r: [a;b] \rightarrow \mathbb R^3$ görbe irányított, ha adott egy
  rendezés ($\leq$) a paraméterértékeken. Ekkor $t_1 < t_2$ esetén
  $\rvec r(t_1)$ a görbe korábbi pontja, $\rvec r(t_2)$-höz képest. Ha
  $\rvec r(a) = \rvec r(b)$, akkor a görbe zárt.
\end{definition}

\vfill

\begin{definition}[Skalármező görbe menti skalárértékű integrálja]
  Legyen $\varphi: U \subseteq \mathbb R^3 \rightarrow \mathbb R$ leképezés,
  $\rvec r: [a;b] \rightarrow \gamma \subset U, t \mapsto \rvec r(t)$ pedig a
  $\gamma$ görbe parametrizált egyenlete. Ekkor az $\varphi$ skalármező $\gamma$
  görbe menti skalárértékű integrálja:
  \[
    \int_\gamma \varphi(\rvec r) \, \diff s =
    \int_a^b \varphi(\rvec r(t)) \norma{\dot{\rvec r}(t)} \, \diff t
    \text.
  \]
\end{definition}

\vfill

\begin{definition}[Vektormező görbe menti skalár- és vektorértékű integrálja]
  Legyen $\rvec v: U \subseteq \mathbb R^3 \rightarrow \mathbb R^3$ vektormező,
  $\rvec r: [a;b] \rightarrow \gamma \subset U, t \mapsto \rvec r(t)$ pedig a
  $\gamma$ görbe parametrizált egyenlete. Ekkor az $\rvec v$ vektormező $\gamma$
  görbe menti
  \begin{itemize}
    \item skalárértékű integrálja:
          \(
            \displaystyle
            \int_\gamma \scalar{\rvec v(\rvec r)}{\diff \rvec r} =
            \int_a^b \scalar{\rvec v(\rvec r(t))}{\dot{\rvec r}(t)} \diff t
            \text,
          \)
    \item vektorértékű integrálja:
          \(
            \displaystyle
            \int_\gamma \rvec v(\rvec r) \times \diff \rvec r =
            \int_a^b \rvec v(\rvec r(t)) \times \dot{\rvec r}(t) \, \diff t
            \text.
          \)
  \end{itemize}
\end{definition}

\clearpage
\section{Feladatok}

\clearpage
\section{Segédlet}

\subsection{Görbék paraméterezése}

\begin{tabular}{
  >{\bullet\;\bfseries}p{2.75cm}<{:}
  p{5.675cm}
  p{2.75cm}
  >{\centering\arraybackslash}m{4cm}
  }
  Egyenes
   & $\rvec r(t) = \rvec r_0 + t \rvec v$
   & $t \in \mathbb R$
   & \relativestandalone{../../../graphics/curve-parametrization/line}
  \\[14mm]
  Szakasz
   & $\rvec r(t) = \rvec r_0 + t (\rvec r_1 - \rvec r_0)$
   & $t \in [0;1]$
   & \relativestandalone{../../../graphics/curve-parametrization/line-segment}
  \\[14mm]
  Körvonal
   & $\rvec r(t) = \begin{bmatrix} r \cos t \\ r \sin t \\ 0 \end{bmatrix}$
   & $t \in [0;2\pi)$
   & \relativestandalone{../../../graphics/curve-parametrization/circle}
  \\[14mm]
    Ellipszis
   & $\rvec r(t) = \begin{bmatrix} a \cos t \\ b \sin t \\ 0 \end{bmatrix}$
   & $t \in [0;2\pi)$
   & \relativestandalone{../../../graphics/curve-parametrization/ellipse}
  \\[14mm]
  Spirál
   & $\rvec r(t) = \begin{bmatrix} a \cos t \\ a \sin t \\ bt \end{bmatrix}$
   & $t \in \mathbb R$
   & \relativestandalone{../../../graphics/curve-parametrization/spiral}
\end{tabular}

\subsection{Koordináta-transzformációk}

\def\arraystretch{1.1}
\begin{tabular}{
  >{\bullet\;}
  m{2.75cm}
  m{3cm}
  m{2.25cm}
  m{2.75cm}
  >{\centering\arraybackslash}m{4cm}
  }
  \textbf{Polár:}
   & $x = r \cos \varphi$ \newline
  $y = r \sin \varphi$
   & $r \in [0; R]$ \newline
  $\varphi \in [0; 2\pi)$
   & $\det \rmat J = r$
   & \relativestandalone{../../../graphics/coordinate-systems/polar}
  \\[12mm]
    \textbf{Henger:}
   & $x = r \cos \varphi$ \newline
  $y = r \sin \varphi$ \newline
  $z = z$
   & $r \in [0; R]$ \newline
  $\varphi \in [0; 2\pi)$ \newline
    $z \in \mathbb R$
   & $\det \rmat J = r$
   & \relativestandalone{../../../graphics/coordinate-systems/cylindrical}
  \\[12mm]
  \textbf{Gömb:}
   & $x = r \sin \varphi \cos \vartheta $ \newline
  $y = r \sin \varphi \sin \vartheta $ \newline
  $z = r \cos \varphi$
   & $r \in [0; R]$ \newline
  $\varphi \in [0; \pi]$ \newline
  $\vartheta \in [0; 2\pi)$
   & $\det \rmat J = r^2 \sin \varphi$
   & \relativestandalone{../../../graphics/coordinate-systems/spherical}
  \\
\end{tabular}

\end{document}
