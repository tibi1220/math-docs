\documentclass[lang=magyar]{math-handout}

\title{Integrálátalakító tételek}
\area{Többváltozós analízis}
\subject{Matematika G3}
\date{Utoljára frissítve: \today}
\author{Sándor Tibor}
\docno{5}

\begin{document}
\allowdisplaybreaks

\maketitle

\vspace{1em}

\begin{summary}
  Megismerkedünk a többváltozós integrálátalakító tételekkel. A Stokes-tétel
  segítségével a felületi integrálok számítását visszavezetjük a peremre, a
  Gauss-Osz\-to\-grad\-szkij-tétel segítségével a térfogati integrálok számítását
  visszavezetjük a térfogat peremére.
\end{summary}

\vspace{-1em}

\section{Elméleti áttekinő}

\vspace{1em}

\begin{theorem}[Stokes -- tétel]
  Legyen $S$ egy $\mathbb R^3$-beli irányított, parametrizált felület.
  Legyen továbbá $\rvec v : \mathbb R^3 \rightarrow \mathbb R^3$ legalább
  egyszer folytonosan differenciálható vektormező. Jelölje a $\partial S =
    \gamma$ az $S$ peremét indukált irányítással. Ekkor:
  \[
    \int \limits_S \scalar{\rot \rvec v}{\diff \rvec S}
    =
    \oint \limits_{\partial S} \scalar{\rvec v}{\diff \rvec r}
    \text.
  \]
  Ha $\rvec v$ skalárpotenciálos, akkor az integrál értéke zérus, hiszen
  $\rot \rvec v = \rot \grad \varphi = \rvec 0$.
\end{theorem}

\vfill

\begin{theorem}[Gauss -- Osztogradszkij -- tétel]
  Legyen $V$ egy $\mathbb R^3$-beli irányított, parametrizált térfogat. Legyen
  továbbá $\rvec v : \mathbb R^3 \rightarrow \mathbb R^3$ legalább egyszer
  folytonosan differenciálható vektormező. Jelölje $\partial V = S$ a $V$
  peremét indukált irányítással. Ekkor:
  \[
    \int \limits_V \Div \rvec v \, \diff V
    =
    \oint \limits_{\partial V} \scalar{\rvec v}{\diff \rvec S}
    \text.
  \]
  Ha $\rvec v$ vektorpotenciálos, akkor az integrál értéke zérus, hiszen
  $\Div \rvec v = \Div \rot \rvec \varphi = 0$.
\end{theorem}

\vfill

\begin{theorem}[Green -- tétel asszimetrikus alakja]
  Legyenek $\varphi; \psi: \mathbb R^3 \rightarrow \mathbb R$ kétszeresen
  folytonos skalármezők, $V \subset \mathbb R^3$ parametrizált, irányított
  tértartomány, $\partial V = S$ perem indukált irányítással. Ekkor:
  \[
    \int \limits_V
    \psi \, \Delta \varphi +
    \scalar{\grad \psi}{\grad \varphi}
    \diff V
    =
    \oint \limits_{\partial V} \scalar{\psi \grad \varphi}{\diff \rvec S}
    \text.
  \]
\end{theorem}

\vfill

\begin{theorem}[Green -- tétel szimmetrikus alakja]
  Legyenek $\varphi; \psi: \mathbb R^3 \rightarrow \mathbb R$ kétszeresen
  folytonos skalármezők, $V \subset \mathbb R^3$ parametrizált, irányított
  tértartomány, $\partial V = S$ perem indukált irányítással. Ekkor:
  \[
    \int \limits_V
    \psi \, \Delta \varphi + \varphi \, \Delta \psi
    \; \diff V
    =
    \oint \limits_{\partial V}
    \scalar{\psi \grad \varphi - \varphi \grad \psi}{\diff \rvec S}
  \]
\end{theorem}

\clearpage
\section{Feladatok}

\begin{exercise}{A Stokes-tétel alkalmazása}
  Adott az $\rvec F(x;y;z) = y \uvec i + z \uvec j + x^3 \uvec k$ függvény.
  Számítsuk ki az $\rvec F$ skalármező $x = 0$ síkon elhelyezkedő, 2 egység
  sugarú körön vett vonalmenti integrálját.

  \exsol{%
    Megoldás a Stokes-tétel segítségével:\\[2mm]
    Először számítsuk ki a $\rot \rvec F$ rotációt:
    \[
      \rot \rvec F = \begin{bmatrix}
        \partial_x \\ \partial_y \\ \partial_z
      \end{bmatrix} \times \begin{bmatrix}
        y \\ z \\ x^3
      \end{bmatrix}
      =
      \begin{bmatrix}
        -1 \\ -3x^2 \\ -1
      \end{bmatrix}
      \text.
    \]
  }
\end{exercise}

\end{document}
