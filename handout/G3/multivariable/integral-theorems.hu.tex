\documentclass[lang=magyar]{math-handout}

\title{Integrálátalakító tételek}
\area{Többváltozós analízis}
\subject{Matematika G3}
\date{Utoljára frissítve: \today}
\author{Sándor Tibor}
\docno{5}

\begin{document}
\allowdisplaybreaks

\maketitle

\vspace{1em}

\begin{summary}
  Bevezetjük a Nabla formális differenciálopertor fogalmát. Megismerkedünk a
  gradiens, a divergencia és a rotáció fogalmával, bevezetjük a skalár- és a
  vektorpotenciált, valamint megismerkedünk az ezekkel kapcsolatos alapvető
  azonosságokkal.
\end{summary}

\vspace{-1em}

\section{Elméleti áttekinő}

\begin{theorem}[Stokes -- tétel]
  Legyen $S : [a;b]^2 \rightarrow \mathbb R^3$ irányított,
  parametrizált felület. Irányítsuk a peremet a jobbkézszabály szerint.
  Legyen továbbá $\rvec v : \mathbb R^3 \rightarrow \mathbb R^3$ legalább
  egyszer folytonosan differenciálható vektormező. Ekkor:
  \[
    \int \limits_S \scalar{\rot \rvec v}{\diff \rvec S}
    =
    \oint \limits_{\partial S} \scalar{\rvec v}{\diff \rvec r}
    \text.
  \]
\end{theorem}

\begin{theorem}[Gauss -- Osztogradszkij -- tétel]
  Legyen $V : [a;b]^3 \rightarrow \mathbb R^3$ irányított, parametrizált
  térfogat. Legyen továbbá $\rvec v : \mathbb R^3 \rightarrow \mathbb R^3$
  legalább egyszer folytonosan differenciálható vektormező. Jelölje
  $\partial V = S$ a $V$ peremét indukált irányítással. Ekkor:
  \[
    \int \limits_V \Div \rvec v \, \diff V
    =
    \oint \limits_{\partial V} \scalar{\rvec v}{\diff \rvec S}
    \text.
  \]
\end{theorem}

\begin{theorem}[Green -- tétel asszimetrikus alakja]
  Legyenek $\varphi; \psi: \mathbb R^3 \rightarrow \mathbb R$ kétszeresen
  folytonos skalármezők, $V \subset \mathbb R^3$ parametrizált, irányított
  tértartomány, $\partial V = S$ perem indukált irányítással. Ekkor:
  \[
    \int \limits_V
    \psi \, \Delta \varphi +
    \scalar{\grad \psi}{\grad \varphi}
    \diff V
    =
    \int \limits_{\partial V} \scalar{\psi \grad \varphi}{\diff \rvec S}
    \text.
  \]
\end{theorem}

\begin{theorem}[Green -- tétel szimmetrikus alakja]
  Legyenek $\varphi; \psi: \mathbb R^3 \rightarrow \mathbb R$ kétszeresen
  folytonos skalármezők, $V \subset \mathbb R^3$ parametrizált, irányított
  tértartomány, $\partial V = S$ perem indukált irányítással. Ekkor:
  \[
    \int \limits_V
    \psi \, \Delta \varphi + \varphi \, \Delta \psi
    \; \diff V
    =
    \int \limits_{\partial V} \scalar{\psi \grad \varphi - \varphi \grad \psi}{\diff \rvec S}
  \]
\end{theorem}


\section{Feladatok}

\end{document}
