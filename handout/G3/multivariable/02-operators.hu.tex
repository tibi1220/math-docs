\documentclass[lang=magyar]{math-handout}

\title{Operátorok}
\area{Többváltozós analízis}
\subject{Matematika G3}
\date{Utoljára frissítve: \today}
\author{Sándor Tibor}
\docno{2}

\begin{document}
\allowdisplaybreaks

\maketitle

\vspace{1em}

\begin{summary}
  Bevezetjük a Nabla és Laplace differenciálopertorokat, megismerkedünk a
  gradiens, a divergencia és a rotáció fogalmával, valamint az ezekhez
  kapcsolódó alapvető azonosságokkal.
\end{summary}

\vspace{-1em}

\section{Elméleti áttekinő}

\vfill

\begin{definition}[Nabla -- operátor]
  Az $\mathbb R^n$-beli descartes koordináta-rendszerben, ahol $\rvec x = (x_1;
    x_2; \dots; x_n)$ egy tetszőleges pont koordinátái, a standard bázis pedig
  $\{ \uvec e_1; \uvec e_2; \dots; \uvec e_n \}$ a Nabla egy olyan formális
  differenciáloperátor, melynek koordinátái a parciális derivált operátorok,
  vagyis:
  \[
    \nabla = \sum_{i = 1}^n \uvec e_i \pdv{}{x_i}
    =
    \begin{pmatrix}
      \displaystyle\pdv{}{x_1} &
      \displaystyle\pdv{}{x_2} &
      \hdots                   &
      \displaystyle\pdv{}{x_n}
    \end{pmatrix}
    \text,
  \]
  ahol a zárójelben lévő kifejezés egy oszlopvektor.
  %
  % A három dimenziós térben az operátor az alábbi alakot veszi fel:
  % \[
  %   \nabla = \begin{pmatrix}
  %     \displaystyle\pdv{}{x} &
  %     \displaystyle\pdv{}{y} &
  %     \displaystyle\pdv{}{z}
  %   \end{pmatrix}
  %   \text.
  % \]
\end{definition}

\vfill

\begin{note}[Rotáció, Divergencia, Gradiens]
  \vspace*{-1em}
  \begin{center}
    \def\arraystretch{1.5}
    \newenvironment{bm}{\bgroup\renewcommand*{\arraystretch}{1.1}\begin{bmatrix}}{\end{bmatrix}\egroup}
    \newcommand{\dspl}[3]{\begin{bm}#1\\#2\\#3\end{bm}}
    \newcommand\nablavec{\dspl{\partial_x}{\partial_y}{\partial_z}}

    \begin{tabular}{*{3}{>{\centering\arraybackslash}p{3.5cm}}}
      \def\arraystretch{1}
      % &
      \bfseries Rotáció
       & \bfseries Divergencia
       & \bfseries Gradiens
      \\
      \hline
      % Jelölés & 
      $\rot \rvec v$
       & $\Div \rvec v$
       & $\grad \varphi$
      \\
      % Operátor & 
      $\nabla \times \rvec v$
       & $\scalar{\nabla}{\rvec v}$
       & $\nabla \cdot \varphi$
      \\
      % Számítás &
      $\nablavec \times \dspl{v_x}{v_y}{v_z}$
       & $\scalar{\nablavec}{\dspl{v_x}{v_y}{v_z}}$
       & $\dspl{\partial_x \varphi}{\partial_y \varphi}{\partial_z \varphi}$
      \\
      % Ért. tart. & 
      $\mathcal D_{\rvec v} = \mathbb R^3$
       & $\mathcal D_{\rvec v} = \mathbb R^3$
       & $\mathcal D_{\varphi} = \mathbb R^3$
      \\
      % Ért. készl. &
      $\mathcal R_{\rvec v} = \mathbb R^3$
       & $\mathcal R_{\rvec v} = \mathbb R^3$
       & $\mathcal R_{\varphi} = \mathbb R$
      \\
      % Ért. készl. &
      $\mathcal R_{\rot \rvec v} = \mathbb R^3$
       & $\mathcal R_{\Div \rvec v} = \mathbb R$
       & $\mathcal R_{\grad \varphi} = \mathbb R^3$
      \\
      % \hline
    \end{tabular}
  \end{center}

  Speciális esetek:
  \begin{itemize}
    \item ha $\Div \rvec v = 0$, akkor a vektromező forrásmentes,
    \item ha $\rot \rvec v = \nvec$, akkor a vektromező örvénymentes.
  \end{itemize}
\end{note}

\vfill

\begin{definition}[Laplace -- operátor]
  A Laplace-operátor egy másodrendű differenciáloperátor az $n$ dimenziós
  térben. Megadja egy skalármező gradiensének divergenciáját, azaz:
  \[
    \triangle \varphi
    =
    \scalar{\nabla}{\nabla} \varphi
    =
    \Div \grad \varphi
    \text.
  \]
\end{definition}

\clearpage
\section{Feladatok}

\relativeinclude{../../../exercise/multivariable/operator/nabla-identities.hu}
\relativeinclude{../../../exercise/multivariable/operator/grad-laplacian-calc.hu}
\relativeinclude{../../../exercise/multivariable/operator/curl-div-calc.hu}
% \relativeinclude{../../../exercise/multivariable/operator/inverse-grad.hu}
% \relativeinclude{../../../exercise/multivariable/operator/find-potential.hu}

\clearpage
\section{Segédlet}

\subsection{Azonosságok}

Legyenek $\varphi, \psi: V \rightarrow \mathbb R$ skalármezők, $\rvec u;
  \rvec v; \rvec w: V \rightarrow V$ vektormezők, $\lambda; \mu \in \mathbb R$
skalárok.
\begin{itemize}
  \item Teljesül a linearitás:\vspace{-.33cm}
        \begin{align*}
          \grad \left( \lambda \, \varphi + \mu \, \psi \right)
           & =
          \lambda \grad \varphi + \mu \grad \psi
          \text,
          \\
          \Div \left( \lambda \, \rvec v + \mu \, \rvec w \right)
           & =
          \lambda \Div \rvec v + \mu \Div \rvec w
          \text,
          \\
          \rot \left( \lambda \, \rvec v + \mu \, \rvec w \right)
           & =
          \lambda \rot \rvec v + \mu \rot \rvec w
          \text.
        \end{align*}
  \item Zérusság:\vspace{-.33cm}
        \begin{align*}
          \rot \grad \varphi & \equiv \nvec
          \text,
          \\
          \Div \rot \rvec v  & \equiv 0
          \text.
        \end{align*}
  \item Deriválási szabályokhoz hasonlóan:
        \begin{align*}
          \grad \left( \varphi \, \psi \right)
           & =
          \varphi \, \grad \psi + \psi \, \grad \varphi
          \text,
          \\
          \Div \left( \varphi \, \rvec v \right)
           & =
          \varphi \, \Div \rvec v + \scalar{\grad \varphi}{\rvec v}
          \text,
          \\
          \rot \left( \varphi \, \rvec v \right)
           & =
          \varphi \, \rot \rvec v + \grad \varphi \times \rvec v
          \text.
        \end{align*}
  \item Egyéb szabályok:\vspace{-.33cm}
        \begin{align*}
          \rot \rot \rvec v
           & =
          \grad \Div \rvec v - \Delta \rvec v
          \text,
          \\
          \rot \left( \rvec u \times \rvec v \right)
           & =
          \rvec u \, \Div \rvec v - \rvec v \, \Div \rvec u
          + (\Diff \rvec u) \rvec v - (\Diff \rvec v) \rvec u
          \text,
          \\
          \Div \left( \rvec u \times \rvec v \right)
           & =
          \scalar{\rvec v}{\rot \rvec u} - \scalar{\rvec u}{\rot \rvec v}
          \text,
          \\
          \grad \left( \scalar{\rvec u}{\rvec v} \right)
           & =
          \rvec u \times \rot \rvec v + \rvec v \times \rot \rvec u
          + (\Diff \rvec u) \rvec v + (\Diff \rvec v) \rvec u
        \end{align*}
\end{itemize}

\end{document}
