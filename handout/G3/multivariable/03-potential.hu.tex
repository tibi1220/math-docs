\documentclass[lang=magyar]{math-handout}

\title{Potenciálosság}
\area{Többváltozós analízis}
\subject{Matematika G3}
\date{Utoljára frissítve: \today}
\author{Sándor Tibor}
\docno{3}

\begin{document}
\allowdisplaybreaks

\maketitle

\vspace{1em}

\begin{summary}
  Bevezetjük a skalár- és vektorpotenciálosság fogalmát, valamint megtanuljuk,
  hogy hogyan lehet kiszámítani egy vektormező potenciálfüggvényeit.
\end{summary}

\vspace{-1em}

\section{Elméleti áttekinő}

\begin{definition}[Skalárpotenciálosság]
  Egy $\rvec v: V \rightarrow V$ vektormező skalárpotenciálos, ha létezik olyan
  $\varphi: V \rightarrow \mathbb R$ skalármező, hogy $\rvec v = \grad \varphi$.
\end{definition}

\begin{definition}[Vektorpotenciálosság]
  Egy $\rvec v: V \rightarrow V$ vektormező vektorpotenciálos, ha létezik olyan
  $\rvec u: V \rightarrow V$ vektormező, hogy $\rvec v = \rot \rvec u$.
\end{definition}

\begin{theorem}
  Legyen $\rvec v: V \rightarrow V$ mindenhol értelmezett, legalább egyszer
  differenciálható vektormező. Ekkor:
  \begin{itemize}
    \item $\rvec v$ skalárpotenciálos
          $\;\Leftrightarrow\;$
          $\rot \rvec v = \nvec$,
          hiszen $\rot \grad \varphi \equiv \nvec$,
    \item $\rvec v$ vektorpotenciálos
          $\;\Leftrightarrow\;$
          $\Div \rvec v = 0$,
          hiszen $\Div \rot \rvec u \equiv 0$.
  \end{itemize}
\end{theorem}

\begin{note}[Potenciálfüggvények számítása]
  Legyen $\varphi$ skalármező $\rvec v$ vektormező skalárpotenciálja. Ebben
  az esetben tudjuk, hogy $\rvec v = \grad \varphi$, vagyis
  \[
    \rvec v = \left(
    \pdv{\varphi}{x_1};
    \pdv{\varphi}{x_2};
    \dots;
    \pdv{\varphi}{x_n}
    \right)^\transpose
    \text.
  \]
  Ilyen esetben a potenciálfüggvény az alábbi módon számítható:
  \[
    \varphi (\rvec r)
    = \int_0^{x_1} \rvec v(\xi; x_2; \dots; x_n) \,\diff \xi
    + \int_0^{x_2} \rvec v(0; \xi; \dots; x_n) \,\diff \xi
    + \dots
    + \int_0^{x_n} \rvec v(0; 0; \dots; \xi) \,\diff \xi
    \text.
  \]

  Legyen $\rvec u$ vektormező $\rvec v$ vektormező vektorpotenciálja.
  A potenciál számtalan alakban előállhat, ezért keressük ezt az alábbi alakban:
  \[
    \rvec u = \left( u_x; u_y; 0 \right)^\transpose
  \]
  A potenciál komponensei az alábbi módon számíthatóak:
  \[
    u_x = \int_0^z v_y(x; y; \zeta) \,\diff \zeta
    \text,
    \qquad
    u_y = \int_0^x v_z(\xi; y; 0) \,\diff \xi
    - \int_0^z v_x(x; y; \zeta) \,\diff \zeta
    \text.
  \]
\end{note}

\clearpage
\section{Feladatok}

\relativeinclude{../../../exercise/multivariable/operator/inverse-grad.hu}
\relativeinclude{../../../exercise/multivariable/operator/find-potential.hu}

\end{document}
