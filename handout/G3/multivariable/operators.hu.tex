\documentclass[lang=magyar]{math-handout}

\title{Operátorok}
\area{Többváltozós analízis}
\subject{Matematika G3}
\date{Utoljára frissítve: \today}
\author{Sándor Tibor}
\docno{2}

\begin{document}
\allowdisplaybreaks

\maketitle

\vspace{1em}

\begin{summary}
  Bevezetjük a Nabla formális differenciálopertor fogalmát. Megismerkedünk a
  gradiens, a divergencia és a rotáció fogalmával, bevezetjük a skalár- és a
  vektorpotenciált, valamint megismerkedünk az ezekkel kapcsolatos alapvető
  azonosságokkal.
\end{summary}

\vspace{-1em}

\section{Elméleti áttekinő}

\begin{definition}[Lineáris leképezés adjugáltja]
  $! \; (V; \, \scalar{}{})$ Euklideszi tér, $\varphi: V \rightarrow V$ lineáris
  leképezés. A $\varphi^{*} : V \rightarrow V$ lineáris leképezést a $\varphi$
  leképezés adjungáltjának hívjuk, ha $\scalar{\varphi(\rvec{v}_1)}{\rvec{v}_2}
    \; = \; \scalar{\rvec{v}_1}{\varphi^{*}(\rvec{v}_2)}$ $\forall \; \rvec{v}_1
    ;\; \rvec{v}_2 \in V$.
\end{definition}

\begin{note}[Adjugált leképezés mátrix reprezentációja]
  Reprezentálja $\varphi$-t $\rmat{A}$, $\varphi^{*}$-ot pedig $\rmat{A}^{*}$:
  \begin{align*}
    \scalar{\varphi(\rvec{v})}{\rvec{w}}
    \; & = \;
    \scalar{\rvec{v}}{\varphi^{*}(\rvec{w})}
    \text,
    \\[2mm]
    \rmat{A}\rvec{v} \cdot \rvec{w}
       & =
    \rvec{v} \cdot \rmat{A}^{*} \rvec{w}
    \text,
    \\[2mm]
    \begin{bmatrix}
      a_{11} & a_{12} \\
      a_{21} & a_{22}
    \end{bmatrix}
    \begin{bmatrix}
      v_1 \\ v_2
    \end{bmatrix}
    \cdot
    \begin{bmatrix}
      w_1 \\ w_2
    \end{bmatrix}
       & =
    \begin{bmatrix}
      v_1 \\ v_2
    \end{bmatrix}
    \cdot
    \begin{bmatrix}
      a_{11}^{*} & a_{12}^{*} \\
      a_{21}^{*} & a_{22}^{*}
    \end{bmatrix}
    \begin{bmatrix}
      w_1 \\ w_2
    \end{bmatrix}
    \text,
    \\[2mm]
    \begin{bmatrix}
      a_{11} v_1 + a_{12} v_2 \\
      a_{21} v_1 + a_{22} v_2
    \end{bmatrix}
    \cdot
    \begin{bmatrix}
      w_1 \\
      w_2
    \end{bmatrix}
       & =
    \begin{bmatrix}
      v_1 \\
      v_2
    \end{bmatrix}
    \cdot
    \begin{bmatrix}
      a_{11}^{*} w_1 + a_{12}^{*} w_2 \\
      a_{21}^{*} w_1 + a_{22}^{*} w_2
    \end{bmatrix}
    \text,
  \end{align*}
  \begin{alignat*}{4}
    \scalar{\rmat{A}\rvec{v}}{\rvec{w}}
     & = \circled{$a_{11}$}{green!40!gray} v_1 w_1
     & + \circled{$a_{12}$}{red!40!gray} v_2 w_1
     & + \circled{$a_{21}$}{cyan!40!gray} v_1 w_2
     & + \circled{$a_{22}$}{yellow!40!gray} v_2 w_2
    \text,
    \\
    \scalar{\rmat{A}^{*}\rvec{w}}{\rvec{v}}
     & = \circled{$a_{11}^{*}$}{green!40!gray} w_1 v_1
     & + \circled{$a_{12}^{*}$}{cyan!40!gray} w_2 v_2
     & + \circled{$a_{21}^{*}$}{red!40!gray} w_1 v_2
     & + \circled{$a_{22}^{*}$}{yellow!40!gray} w_2 v_2
    \text.
  \end{alignat*}
  Megállapthatjuk, hogy $\rmat{A}^{*} = \rmat{A}^\transpose$.
\end{note}

\begin{statement}[Leképezés felbontása]
  Legyen $\varphi \in \End V$, ekkor $!\exists$ olyan $\mathcal{A}$ és
  $\mathcal{S}$ antiszimmetrikus és szimmetrikus leképezés, ahol $\varphi =
    \mathcal{A} + \mathcal{S}$, melyek az endomorfizmusok vektorterét 2
  diszjunkt halmazra bontják:
  \[
    \mathcal{A} := \frac{\varphi - \varphi^{*}}{2}
    \text,
    \hspace{5mm} \text{és} \hspace{5mm}
    \mathcal{S} := \frac{\varphi + \varphi^{*}}{2}
    \text.
  \]
\end{statement}

\begin{note}
  Az antiszimmetrikus leképezések és a $V$-beli vektorok között tudunk
  egy-egyértelmű hozzárendelést találni:
  \[
    \rmat{A} \in \mathcal{A}
    \hspace{2.5mm} \leftrightarrow \hspace{2.5mm}
    \rvec{v} \in V
    \text.
  \]
  Keressünk egy olyan $\rvec{v}$ vektort,
  melyre teljesül az alábbi egyenlet:
  \begin{align*}
    \rmat{A}\rvec{w}
     & = \rvec{v} \times \rvec{w}
    \text,
    \\[2mm]
    \begin{bmatrix}
      0       & a_{12}  & a_{13} \\
      -a_{12} & 0       & a_{23} \\
      -a_{13} & -a_{23} & 0
    \end{bmatrix}
    \begin{bmatrix}
      w_1 \\ w_2 \\ w_3
    \end{bmatrix}
     & =
    \begin{bmatrix}
      v_1 \\ v_2 \\ v_3
    \end{bmatrix}
    \times
    \begin{bmatrix}
      w_1 \\ w_2 \\ w_3
    \end{bmatrix}
    \text,
    \\[2mm]
    \begin{bmatrix}
      \rected{$+a_{12}$}{red}  w_2 \rected{$+a_{13}$}{blue}  w_3 \\
      \rected{$-a_{12}$}{red}  w_1 \rected{$+a_{23}$}{green} w_3 \\
      \rected{$-a_{13}$}{blue} w_1 \rected{$-a_{23}$}{green} w_2
    \end{bmatrix}
     & =
    \begin{bmatrix}
      \rected{$v_2$}{blue}  w_3 \rected{$-v_3$}{red}   w_2 \\
      \rected{$v_3$}{red}   w_1 \rected{$-v_1$}{green} w_3 \\
      \rected{$v_1$}{green} w_2 \rected{$-v_2$}{blue}  w_1
    \end{bmatrix}
    \text,
    \hspace{20mm}
    \\[2mm]
    \rvec{v}
     & =
    \begin{bmatrix}
      -a_{23} \\ a_{13}  \\ -a_{12}
    \end{bmatrix}
    \text.
  \end{align*}
\end{note}

\begin{definition}[Vektorinvariáns]
  Egy antiszimmetrikus lineáris transzformáció mindig leírható egy rögzített
  vektorral való keresztszorzással. Ez a vektor a leképezés vektorinvariánsa.
\end{definition}

\begin{statement}[Lineáris leképezés nyoma]
  Egy lineáris transzformáció főátlójában lévő elemek összege minden
  koordinátarendszerben ugyanannyi, tehát a koordináta-transzformáció nem
  befolyásolja. Ezt nevezzük a lineáris leké- pezés nyomának. (trace / spur)
\end{statement}

\begin{note}[Nabla operátor bevezetése]
  Legyen $\rvec f: V \rightarrow V$ függvény. Vegyük ennek a deriváltját.
  \[
    \Diff\rvec f
    =
    \begin{bmatrix}
      \partial_1 f_1 & \partial_2 f_1 & \dots  & \partial_n f_1 \\
      \partial_1 f_2 & \partial_2 f_2 & \dots  & \partial_n f_2 \\
      \vdots         & \vdots         & \ddots & \vdots         \\
      \partial_1 f_n & \partial_2 f_n & \dots  & \partial_n f_n
    \end{bmatrix}
    =
    \begin{bmatrix}
      \grad^\transpose f_1 \\
      \grad^\transpose f_2 \\
      \vdots               \\
      \grad^\transpose f_n \\
    \end{bmatrix}
    \in
    \mathcal M_{n \times n}
    \text.
  \]
  Definiáljuk az alábbi fogalmakat:
  \begin{itemize}
    \item rotáció
          \tabto{2.4cm} – \tabto{3cm}
          $\rot f := \Diff f - \Diff f ^{*}$,

    \item divergencia
          \tabto{2.4cm} – \tabto{3cm}
          $\Div f := \tr \left( \Diff f \right)$.
  \end{itemize}
  $V = \mathbb{R}^3$ esetén:
  \[
    \rot \rvec f
    =
    \begin{bmatrix}
      0                               &
      \partial_2 f_1 - \partial_1 f_2 &
      \partial_3 f_1 - \partial_1 f_3
      \\
      \partial_1 f_2 - \partial_2 f_1 &
      0                               &
      \partial_3 f_2 - \partial_2 f_3
      \\
      \partial_1 f_3 - \partial_3 f_1 &
      \partial_2 f_3 - \partial_3 f_2 &
      0
    \end{bmatrix}
    \text.
  \]
  A mátrix vektorinvariánsa:
  \[
    \rvec{v}
    =
    \begin{bmatrix}
      \partial_2 f_3 - \partial_3 f_2 \\
      \partial_3 f_1 - \partial_1 f_3 \\
      \partial_1 f_2 - \partial_2 f_1
    \end{bmatrix}
    =
    \begin{bmatrix}
      \partial_1 \\ \partial_2 \\ \partial_3
    \end{bmatrix}
    \times
    \begin{bmatrix}
      f_1 \\ f_2 \\ f_3
    \end{bmatrix}
    =
    \nabla \times \rvec f
    \text,
  \]
  ahol $\nabla$ a Nabla operátor (formális differenciálopertor).
\end{note}

\begin{note}
  Gradiens, divergencia és rotáció számítása a Nabla operátorral:
  \[
    \grad \varphi = \nabla \varphi
    \text,\hspace{1cm}
    \Div \rvec v = \scalar{\nabla}{\rvec v}
    \text,\hspace{1cm}
    \rot \rvec v = \nabla \times \rvec v
    \text.
  \]
\end{note}

% \begin{note}[Zérusság]
%   Az alábbi 2 összefüggés mindig teljesül
%   \[
%     \Div \rot \rvec v \equiv 0
%     \text,\hspace{1cm}
%     \rot \grad \varphi \equiv \nvec
%   \]
% \end{note}

\begin{definition}[Skalárpotenciálosság]
  Egy $\rvec v: V \rightarrow V$ vektormező skalárpotenciálos, ha létezik olyan
  $\varphi: V \rightarrow \mathbb R$ skalármező, hogy $\rvec v = \grad \varphi$.
\end{definition}

\begin{definition}[Vektorpotenciálosság]
  Egy $\rvec v: V \rightarrow V$ vektormező vektorpotenciálos, ha létezik olyan
  $\rvec u: V \rightarrow V$ vektormező, hogy $\rvec v = \rot \rvec u$.
\end{definition}

\begin{theorem}
  Legyen $\rvec v: V \rightarrow V$ mindenhol értelmezett, legalább egyszer
  differenciálható vektormező. Ekkor:
  \begin{itemize}
    \item $\rvec v$ skalárpotenciálos
          $\;\Leftrightarrow\;$
          $\rot \rvec v = \nvec$,
    \item $\rvec v$ vektorpotenciálos
          $\;\Leftrightarrow\;$
          $\Div \rvec v = 0$.
  \end{itemize}
\end{theorem}

\begin{identities}
  Legyenek $\varphi, \psi: V \rightarrow \mathbb R$ skalármezők, $\rvec u;
    \rvec v; \rvec w: V \rightarrow V$ vektormezők, $\lambda; \mu \in \mathbb R$
  skalárok.
  \begin{itemize}
    \item Teljesül a linearitás:\vspace{-.33cm}
          \begin{align*}
            \grad \left( \lambda \, \varphi + \mu \, \psi \right)
             & =
            \lambda \grad \varphi + \mu \grad \psi
            \text,
            \\
            \Div \left( \lambda \, \rvec v + \mu \, \rvec w \right)
             & =
            \lambda \Div \rvec v + \mu \Div \rvec w
            \text,
            \\
            \rot \left( \lambda \, \rvec v + \mu \, \rvec w \right)
             & =
            \lambda \rot \rvec v + \mu \rot \rvec w
            \text.
          \end{align*}
    \item Zérusság:\vspace{-.33cm}
          \begin{align*}
            \rot \grad \varphi & \equiv \nvec
            \text,
            \\
            \Div \rot \rvec v  & \equiv 0
            \text.
          \end{align*}
    \item Deriválási szabályokhoz hasonlóan:
          \begin{align*}
            \grad \left( \varphi \, \psi \right)
             & =
            \varphi \, \grad \psi + \psi \, \grad \varphi
            \text,
            \\
            \Div \left( \varphi \, \rvec v \right)
             & =
            \varphi \, \Div \rvec v + \scalar{\grad \varphi}{\rvec v}
            \text,
            \\
            \rot \left( \varphi \, \rvec v \right)
             & =
            \varphi \, \rot \rvec v + \grad \varphi \times \rvec v
            \text.
          \end{align*}
    \item Egyéb szabályok:\vspace{-.33cm}
          \begin{align*}
            \rot \rot \rvec v
             & =
            \grad \Div \rvec v - \Delta \rvec v
            \text,
            \\
            \rot \left( \rvec u \times \rvec v \right)
             & =
            \rvec u \, \Div \rvec v - \rvec v \, \Div \rvec u
            + (\Diff \rvec u) \rvec v - (\Diff \rvec v) \rvec u
            \text,
            \\
            \Div \left( \rvec u \times \rvec v \right)
             & =
            \scalar{\rvec v}{\rot \rvec u} - \scalar{\rvec u}{\rot \rvec v}
            \text,
            \\
            \grad \left( \scalar{\rvec u}{\rvec v} \right)
             & =
            \rvec u \times \rot \rvec v + \rvec v \times \rot \rvec u
            + (\Diff \rvec u) \rvec v + (\Diff \rvec v) \rvec u
          \end{align*}
  \end{itemize}
\end{identities}

\section{Feladatok}

\begin{exercise}{%
    Számítsuk ki az alábbi vektormezők divergenciáját és rotációját!
  }
  \begin{enumerate}[a)]
    \item $\rvec f(x;y;z) = (yz) \,\uvec i + (xz) \,\uvec j + (xy) \,\uvec k$
    \item $\rvec f(x;y;z) = (3xy + z^2) \,\uvec i + (6 e^z) \,\uvec j + (-5x^y) \,\uvec k$
  \end{enumerate}

  \exsol{%
    \begin{enumerate}[a)]
      \item $\rvec g(x;y;z) = (yz) \,\uvec i + (xz) \,\uvec j + (xy) \,\uvec k$
            \vspace{3mm}
            \begin{itemize}
              \item A divergencia:\\[3mm]
                    \(
                      \displaystyle
                      \Div \rvec f
                      = \scalar{\nabla}{\rvec f}
                      = \pdv{yz}{x} + \pdv{xz}{y} + \pdv{xy}{z}
                      = 0 + 0 + 0
                      = 0
                      \text.
                    \)
              \item A rotáció:\\[3mm]
                    \(
                      \displaystyle
                      \rot \rvec f
                      = \nabla \times \rvec f
                      = \begin{bmatrix}
                        \partial_x \\ \partial_y \\ \partial_z
                      \end{bmatrix} \times \begin{bmatrix}
                        yz \\ xz \\ xy
                      \end{bmatrix} = \begin{bmatrix}
                        0 - 0 \\ 0 - 0 \\ 0 - 0
                      \end{bmatrix} = \begin{bmatrix}
                        0 \\ 0 \\ 0
                      \end{bmatrix}
                      \text.
                    \)
            \end{itemize}

      \item $\rvec g(x;y;z) = (3xy + z^2) \,\uvec i + (6 e^z) \,\uvec j + (-5x^y) \,\uvec k$
            \vspace{3mm}
            \begin{itemize}
              \item A divergencia:\\[3mm]
                    \(
                      \displaystyle
                      \Div \rvec g
                      = \scalar{\nabla}{\rvec g}
                      = \pdv{(3xy + z^2)}{x} + \pdv{(6 e^z)}{y} + \pdv{(-5x^y)}{z}
                      = 3y + 0 + 0
                      = 3y
                      \text.
                    \)
              \item A rotáció:\\[3mm]
                    \(
                      \displaystyle
                      \rot \rvec g
                      = \nabla \times \rvec g
                      = \begin{bmatrix}
                        \partial_x \\ \partial_y \\ \partial_z
                      \end{bmatrix} \times \begin{bmatrix}
                        3xy + z^2 \\ 6e^z \\ -5x^y
                      \end{bmatrix} = \begin{bmatrix}
                        -5 x^y \ln x - 6 e^z \\
                        2z + 5 y x^{y-1}     \\
                        -3x
                      \end{bmatrix}
                      \text.
                    \)
            \end{itemize}
    \end{enumerate}
  }
\end{exercise}

\begin{exercise}{Számítsuk ki az alábbi skalármezők gradiensét!}
  \begin{enumerate}[a)]
    \item $f(x;y;z) = xyz$
    \item $g(x;y;z) = 6x^y + \sin e^z$
  \end{enumerate}

  \exsol{%
    \begin{enumerate}[a)]
      \item $f(x;y;z) = xyz$
            \vspace{3mm}
            \begin{itemize}
              \item A gradiens:\\[3mm]
                    \(
                      \displaystyle
                      \grad f
                      = \nabla f
                      = \begin{bmatrix}
                        \partial_x \\ \partial_y \\ \partial_z
                      \end{bmatrix} f
                      = \begin{bmatrix}
                        yz \\ xz \\ xy
                      \end{bmatrix}
                      \text.
                    \)
            \end{itemize}

      \item $g(x;y;z) = 6x^y + \sin e^z$
            \vspace{3mm}
            \begin{itemize}
              \item A gradiens:\\[3mm]
                    \(
                      \displaystyle
                      \grad g
                      = \nabla g
                      = \begin{bmatrix}
                        \partial_x \\ \partial_y \\ \partial_z
                      \end{bmatrix} f
                      = \begin{bmatrix}
                        6 y x^{y - 1} \\
                        6 x^y \ln x   \\
                        e^z \cos e^z
                      \end{bmatrix}
                      \text.
                    \)
            \end{itemize}
    \end{enumerate}
  }
\end{exercise}

\begin{exercise}{Adjuk meg azon skalármezőket, melyek gradiensei az alábbiak!}
  \begin{enumerate}[a)]
    \item $\rvec F(x;y;z) = \ijk{y \sin xy}{x \sin xy}{3}$
    \item $\rvec G(x;y;z) = \ijk{y/x}{\ln x}{2/z^{2}}$
  \end{enumerate}

  \exsol{%
    \begin{enumerate}[a)]
      \item $\rvec F(x;y;z) = \ijk{y \sin xy}{x \sin xy}{3}$

            \vspace{3mm}
            A gradiens tulajdonságai alapján:
            \[
              F_x = \pdv{f}{x} = y \sin xy
              \text,\hspace{1cm}
              F_y = \pdv{f}{y} = x \sin xy
              \text,\hspace{1cm}
              F_z = \pdv{f}{z} = 3
              \text.
            \]
            % Integráljuk a parciális deriváltakat:
            % \begin{alignat*}{9}
            %   f_1 &  = \int \pdv{f}{x} \,\diff x
            %       && = \int y \sin xy \,\diff x
            %       && = - \cos xy
            %       && + C_1(y;z)
            %   \\
            %   f_2 &  = \int \pdv{f}{y} \,\diff y
            %       && = \int x \sin xy \,\diff y
            %       && = - \cos xy
            %       && + C_2(x;z)
            %   \\
            %   f_3 &  = \int \pdv{f}{z} \,\diff z
            %       && = \int 3 \,\diff z
            %       && = 3z
            %       && + C_3(x;y)
            % \end{alignat*}
            % Tudjuk, hogy $f = f_1 = f_2 = f_3$, a konstansok tehát:
            % \[
            %   C_1(y;z) = 3z + C
            %   \text,\hspace{1cm}
            %   C_2(x;z) = 3z + C
            %   \text,\hspace{1cm}
            %   C_3(x;y) = - \cos xy + C
            %   \text.
            % \]
            Az eredeti függvény az alábbi alakban írható fel:
            \[
              f(\rvec r) =
              \int_0^x F_x(\xi; y; z) \,\diff \xi +
              \int_0^y F_y(0; \eta; z) \,\diff \eta +
              \int_0^z F_z(0; 0; \zeta) \,\diff \zeta
              \text.
            \]
            Végezzük el az integrálást:
            \begin{align*}
              f(\rvec r)
               & =
              \int_0^x y \sin \xi y \,\diff \xi +
              \int_0^y 0 \,\diff \eta +
              \int_0^z 3 \,\diff \zeta
              \\
               & =
              1 - \cos xy + 0 + 3z + C
              \text.
            \end{align*}
            Az eredeti függvény tehát:
            \[
              f(x;y;z) = 3z - \cos xy + C
              \text.
            \]

      \item $\rvec G(x;y;z) = \ijk{y/x}{\ln x}{2/z^{2}}$

            \vspace{3mm}
            A gradiens tulajdonságai alapján:
            \[
              G_x = \pdv{g}{x} = \frac{y}{x}
              \text,\hspace{1cm}
              G_y = \pdv{g}{y} = \ln x
              \text,\hspace{1cm}
              G_z = \pdv{g}{z} = \frac{2}{z^2}
              \text.
            \]
            % Integráljuk a parciális deriváltakat:
            % \begin{alignat*}{9}
            %   g_1 &  = \int \pdv{g}{x} \,\diff x
            %       && = \int \frac{y}{x} \,\diff x
            %       && = y \ln x
            %       && + C_1(y;z)
            %   \\
            %   g_2 &  = \int \pdv{g}{y} \,\diff y
            %       && = \int \ln x \,\diff y
            %       && = y \ln x
            %       && + C_2(x;z)
            %   \\
            %   g_3 &  = \int \pdv{g}{z} \,\diff z
            %       && = \int \frac{2}{z^2} \,\diff z
            %       && = - \frac{2}{z}
            %       && + C_3(x;y)
            % \end{alignat*}
            % Tudjuk, hogy $g = g_1 = g_2 = g_3$, a konstansok tehát:
            % \[
            %   C_1(y;z) = - \frac{2}{z} + C
            %   \text,\hspace{1cm}
            %   C_2(x;z) = - \frac{2}{z} + C
            %   \text,\hspace{1cm}
            %   C_3(x;y) = y \ln x + C
            %   \text.
            % \]
            Az eredeti függvény az alábbi alakban írható fel:
            \[
              g(\rvec r) =
              \int_0^x G_x(\xi; 0; 0) \,\diff \xi +
              \int_0^y G_y(x; \eta; 0) \,\diff \eta +
              \int_0^z G_z(x; y; \zeta) \,\diff \zeta
              \text.
            \]
            Végezzük el az integrálást:
            \begin{align*}
              g(\rvec r)
               & =
              \int_0^x 0 \,\diff \xi +
              \int_0^y \ln x \,\diff \eta +
              \int_0^z \frac{2}{\zeta^2} \,\diff \zeta
              \\
               & =
              0 + y \ln x - \frac{2}{z} + C
              \text.
            \end{align*}
            Az eredeti függvény tehát:
            \[
              g(x;y;z) = y \ln x - \frac{2}{z} + C
            \]
    \end{enumerate}
  }
\end{exercise}

\begin{exercise}{%
    Vizsgáljuk meg, hogy az alábbi vektormezők skalár- illetve
    vektorpotenciálisak-e! Amennyiben igen, adjuk meg a potenciálfüggvényeket!
    A valós konsztansokat legyenek zérusak, valamint a vektorpotenciált --
    amennyiben létezik -- olyan módon adjuk meg, hogy a harmadik komponense
    zérus legyen.
  }
  \begin{enumerate}[a)]
    \item $\rvec v(\rvec r) = \ijk{y + z}{x + z}{x + y}$
    \item $\rvec w(\rvec r) = \ijk{e^{x + \sin y}}{e^{x + \sin y} \cos y}{0}$
    \item $\rvec u(\rvec r) = \ijk{2zx^3}{3z}{-3x^2z^2}$
  \end{enumerate}

  \exsol{%
    \begin{enumerate}[a)]
      \item $\rvec v(\rvec r) = \ijk{y + z}{x + z}{x + y}$

            \begin{itemize}
              \item A vektormező skalárpotenciálos, ha rotációja zérus:
                    \[
                      \rot \rvec v
                      =
                      \begin{bmatrix}
                        \partial_x \\ \partial_y \\ \partial_z
                      \end{bmatrix}
                      \times
                      \begin{bmatrix}
                        y + z \\ x + z \\ x + y
                      \end{bmatrix}
                      =
                      \begin{bmatrix}
                        \partial_y (x + y) - \partial_z (x + z) \\
                        \partial_z (y + z) - \partial_x (x + y) \\
                        \partial_x (x + z) - \partial_y (y + z)
                      \end{bmatrix}
                      =
                      \begin{bmatrix}
                        1 - 1 \\
                        1 - 1 \\
                        1 - 1
                      \end{bmatrix}
                      =
                      \begin{bmatrix}
                        0 \\ 0 \\ 0
                      \end{bmatrix}
                    \]

                    A potenciálfüggvény:
                    \begin{align*}
                      \varphi(\rvec r)
                       & =
                      \int_0^x v_x(\xi; y; z) \,\diff \xi +
                      \int_0^y v_y(0; \eta; z) \,\diff \eta +
                      \int_0^z v_z(0; 0; \zeta) \,\diff \zeta
                      \\
                       & =
                      \int_0^x (y + z) \,\diff \xi +
                      \int_0^y (0 + z) \,\diff \eta +
                      \int_0^z (0 + 0) \,\diff \zeta
                      \\
                       & =
                      xy + xz + yz + C
                      \text.
                    \end{align*}

                    A kereseett potenciálfüggvény:
                    \[
                      \varPhi (\rvec r)
                      =
                      xy + xz + yz
                      \text.
                    \]

              \item A vektormező vektorpotenciálos, ha divergenciája zérus:
                    \[
                      \Div \rvec v
                      =
                      \pdv{\rvec v}{x} + \pdv{\rvec v}{y} + \pdv{\rvec v}{z}
                      =
                      0 + 0 + 0
                      =
                      0
                      \text.
                    \]

                    A potenciálfüggvény:
                    \begin{align*}
                      V_x(\rvec r)
                       & =
                      \int_0^z v_y(x; y; \zeta) \,\diff \zeta
                      =
                      \int_0^z (x + \zeta) \,\diff \zeta
                      =
                      xz + \frac{z^2}{2} + C_x
                      \text,
                      \\
                      V_y(\rvec r)
                       & =
                      \int_0^x v_z(\xi; y; 0) \,\diff \xi -
                      \int_0^z v_x(x; y; \zeta) \,\diff \zeta
                      \\
                       & =
                      \int_0^x (\xi + y) \,\diff \xi -
                      \int_0^z (y + \zeta) \,\diff \zeta
                      \\
                       & =
                      \frac{x^2}{2} + xy -
                      \frac{z^2}{2} - yz + C_y
                    \end{align*}

                    A keresett vektorpotenciál:
                    \[
                      \rvec V(\rvec r)
                      =
                      %  & =
                      % \ijk{V_x(\rvec r)}{V_y(\rvec r)}{0}
                      % \\
                      %  & =
                      \ijk{xz + \frac{z^2}{2}}
                      {\frac{x^2}{2} + xy - \frac{z^2}{2} - yz}
                      {0}
                      \text.
                    \]
            \end{itemize}

      \item $\rvec w(\rvec r) = \ijk{e^{x + \sin y}}{e^{x + \sin y} \cos y}{0}$
            \begin{itemize}
              \item Egy vektormező skalárpotenciálos, ha rotációja zérus:
                    \[
                      \rot \rvec w
                      =
                      \begin{bmatrix}
                        \partial_x \\ \partial_y \\ \partial_z
                      \end{bmatrix}
                      \times
                      \begin{bmatrix}
                        e^{x + \sin y} \\ e^{x + \sin y} \cos y \\ 0
                      \end{bmatrix}
                      % =
                      % \begin{bmatrix}
                      %   \partial_y 0 - \partial_z e^{x + \sin y} \cos y \\
                      %   \partial_z e^{x + \sin y} - \partial_x 0        \\
                      %   \partial_x e^{x + \sin y} \cos y - \partial_y e^{x + \sin y}
                      % \end{bmatrix}
                      =
                      \begin{bmatrix}
                        0 - 0 \\ 0 - 0 \\ e^{x + \sin y} \cos y - e^{x + \sin y} \cos y
                      \end{bmatrix}
                      =
                      \begin{bmatrix}
                        0 \\ 0 \\ 0
                      \end{bmatrix}
                      \text.
                    \]

                    A potenciálfüggvény:
                    \begin{align*}
                      \psi(\rvec r)
                       & =
                      \int_0^x w_x(\xi; y; z) \,\diff \xi +
                      \int_0^y w_y(0; \eta; z) \,\diff \eta +
                      \int_0^z w_z(0; 0; \zeta) \,\diff \zeta
                      \\
                       & =
                      \int_0^x e^{\xi + \sin y} \,\diff \xi +
                      \int_0^y e^{\sin \eta} \cos \eta \,\diff \eta +
                      \int_0^z 0 \,\diff \zeta
                      \\
                       & =
                      (e^x - 1) e^{\sin y} + e^{\sin y} - 1 + 0 + C
                      \text.
                    \end{align*}

                    A keresett potenciálfüggvény:
                    \[
                      \varPsi(\rvec r)
                      =
                      e^{x + \sin y}
                      \text.
                    \]

                    Egy vektormező vektorpotenciálos, ha divergenciája zérus:
                    \[
                      \Div \rvec w
                      =
                      \pdv{\rvec w}{x} + \pdv{\rvec w}{y} + \pdv{\rvec w}{z}
                      =
                      e^{x + \sin y} + e^{x + \sin y}(\cos^2 y - \sin y)
                      \neq 0
                      \text.
                    \]

                    Mivel $\Div \rvec w \neq 0$, ezért nem létezik $\rvec w$-nek
                    vektorpotenciálja.
            \end{itemize}

      \item $\rvec u(\rvec r) = \ijk{2zx^3}{3z}{-3x^2z^2}$
            \begin{itemize}
              \item Egy vektormező skalárpotenciálos, ha rotációja zérus.
                    \[
                      \rot \rvec u
                      =
                      \begin{bmatrix}
                        \partial_x \\ \partial_y \\ \partial_z
                      \end{bmatrix}
                      \times
                      \begin{bmatrix}
                        2zx^3 \\ 3z \\ -3x^2z^2
                      \end{bmatrix}
                      =
                      \begin{bmatrix}
                        0 - 3           \\
                        2 x^3 + 6 x z^2 \\
                        0 - 0
                      \end{bmatrix}
                      \neq
                      \nvec
                    \]

                    Mivel $\rot \rvec u \neq \nvec$, ezért $\rvec u$-nak nem
                    létezik skalárpotenciálja.

              \item Egy vektormező vektorpotenciálos, ha divergenciája zérus.
                    \[
                      \Div \rvec u
                      =
                      \pdv{\rvec u}{x} + \pdv{\rvec u}{y} + \pdv{\rvec u}{z}
                      =
                      6 x^2 z + 0 -6 x^2 z
                      =
                      0
                      \text.
                    \]

                    A potenciálfüggvény:
                    \begin{align*}
                      U_x(\rvec r)
                       & =
                      \int_0^z u_y(x; y; \zeta) \,\diff \zeta
                      =
                      \int_0^z  (3 \zeta) \,\diff \zeta
                      =
                      \frac{3 z^2}{2} + C_x
                      \text,
                      \\
                      U_y(\rvec r)
                       & =
                      \int_0^x u_z(\xi; y; 0) \,\diff \xi -
                      \int_0^z u_x(x; y; \zeta) \,\diff \zeta
                      \\
                       & =
                      \int_0^x (0) \,\diff \xi -
                      \int_0^z (2 \zeta x^3)  \,\diff \zeta
                      =
                      0 - x^3 z^2 + C_y
                      \text.
                    \end{align*}

                    A keresett vektorpotenciál:
                    \[
                      \rvec U(\rvec r)
                      = \ijk{\frac{3z^2}{2}}{-x^3 z^2}{0}
                      \text.
                    \]
            \end{itemize}
    \end{enumerate}
  }
\end{exercise}

\end{document}
