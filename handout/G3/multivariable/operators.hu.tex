\documentclass[lang=magyar]{math-handout}

\title{Operátorok}
\area{Többváltozós analízis}
\subject{Matematika G3}
\date{Utoljára frissítve: \today}
\author{Sándor Tibor}
\docno{2}

\begin{document}
\allowdisplaybreaks

\maketitle

\vspace{1em}

\begin{summary}
  Bevezetjük a Nabla formális differenciálopertor fogalmát. Megismerkedünk a
  gradiens, a divergencia és a rotáció fogalmával, bevezetjük a skalár- és a
  vektorpotenciált, valamint megismerkedünk az ezekkel kapcsolatos alapvető
  azonosságokkal.
\end{summary}

\vspace{-1em}

\section{Elméleti áttekinő}

\begin{definition}[Lineáris leképezés adjugáltja]
  $! \; (V; \, \scalar{}{})$ Euklideszi tér, $\varphi: V \rightarrow V$ lineáris
  leképezés. A $\varphi^{*} : V \rightarrow V$ lineáris leképezést a $\varphi$
  leképezés adjungáltjának hívjuk, ha $\scalar{\varphi(\rvec{v}_1)}{\rvec{v}_2}
    \; = \; \scalar{\rvec{v}_1}{\varphi^{*}(\rvec{v}_2)}$ $\forall \; \rvec{v}_1
    ;\; \rvec{v}_2 \in V$.
\end{definition}

\begin{note}[Adjugált leképezés mátrix reprezentációja]
  Reprezentálja $\varphi$-t $\rmat{A}$, $\varphi^{*}$-ot pedig $\rmat{A}^{*}$:
  \begin{align*}
    \scalar{\varphi(\rvec{v})}{\rvec{w}}
    \; & = \;
    \scalar{\rvec{v}}{\varphi^{*}(\rvec{w})}
    \text,
    \\[2mm]
    \rmat{A}\rvec{v} \cdot \rvec{w}
       & =
    \rvec{v} \cdot \rmat{A}^{*} \rvec{w}
    \text,
    \\[2mm]
    \begin{bmatrix}
      a_{11} & a_{12} \\
      a_{21} & a_{22}
    \end{bmatrix}
    \begin{bmatrix}
      v_1 \\ v_2
    \end{bmatrix}
    \cdot
    \begin{bmatrix}
      w_1 \\ w_2
    \end{bmatrix}
       & =
    \begin{bmatrix}
      v_1 \\ v_2
    \end{bmatrix}
    \cdot
    \begin{bmatrix}
      a_{11}^{*} & a_{12}^{*} \\
      a_{21}^{*} & a_{22}^{*}
    \end{bmatrix}
    \begin{bmatrix}
      w_1 \\ w_2
    \end{bmatrix}
    \text,
    \\[2mm]
    \begin{bmatrix}
      a_{11} v_1 + a_{12} v_2 \\
      a_{21} v_1 + a_{22} v_2
    \end{bmatrix}
    \cdot
    \begin{bmatrix}
      w_1 \\
      w_2
    \end{bmatrix}
       & =
    \begin{bmatrix}
      v_1 \\
      v_2
    \end{bmatrix}
    \cdot
    \begin{bmatrix}
      a_{11}^{*} w_1 + a_{12}^{*} w_2 \\
      a_{21}^{*} w_1 + a_{22}^{*} w_2
    \end{bmatrix}
    \text,
  \end{align*}
  \begin{alignat*}{4}
    \scalar{\rmat{A}\rvec{v}}{\rvec{w}}
     & = \circled{$a_{11}$}{green!40!gray} v_1 w_1
     & + \circled{$a_{12}$}{red!40!gray} v_2 w_1
     & + \circled{$a_{21}$}{cyan!40!gray} v_1 w_2
     & + \circled{$a_{22}$}{yellow!40!gray} v_2 w_2
    \text,
    \\
    \scalar{\rmat{A}^{*}\rvec{w}}{\rvec{v}}
     & = \circled{$a_{11}^{*}$}{green!40!gray} w_1 v_1
     & + \circled{$a_{12}^{*}$}{cyan!40!gray} w_2 v_2
     & + \circled{$a_{21}^{*}$}{red!40!gray} w_1 v_2
     & + \circled{$a_{22}^{*}$}{yellow!40!gray} w_2 v_2
    \text.
  \end{alignat*}
  Megállapthatjuk, hogy $\rmat{A}^{*} = \rmat{A}^\transpose$.
\end{note}

\begin{statement}[Leképezés felbontása]
  Legyen $\varphi \in \End V$, ekkor $!\exists$ olyan $\mathcal{A}$ és
  $\mathcal{S}$ antiszimmetrikus és szimmetrikus leképezés, ahol $\varphi =
    \mathcal{A} + \mathcal{S}$, melyek az endomorfizmusok vektorterét 2
  diszjunkt halmazra bontják:
  \[
    \mathcal{A} := \frac{\varphi - \varphi^{*}}{2}
    \text,
    \hspace{5mm} \text{és} \hspace{5mm}
    \mathcal{S} := \frac{\varphi + \varphi^{*}}{2}
    \text.
  \]
\end{statement}

\begin{note}
  Az antiszimmetrikus leképezések és a $V$-beli vektorok között tudunk
  egy-egyértelmű hozzárendelést találni:
  \[
    \rmat{A} \in \mathcal{A}
    \hspace{2.5mm} \leftrightarrow \hspace{2.5mm}
    \rvec{v} \in V
    \text.
  \]
  Keressünk egy olyan $\rvec{v}$ vektort,
  melyre teljesül az alábbi egyenlet:
  \begin{align*}
    \rmat{A}\rvec{w}
     & = \rvec{v} \times \rvec{w}
    \text,
    \\[2mm]
    \begin{bmatrix}
      0       & a_{12}  & a_{13} \\
      -a_{12} & 0       & a_{23} \\
      -a_{13} & -a_{23} & 0
    \end{bmatrix}
    \begin{bmatrix}
      w_1 \\ w_2 \\ w_3
    \end{bmatrix}
     & =
    \begin{bmatrix}
      v_1 \\ v_2 \\ v_3
    \end{bmatrix}
    \times
    \begin{bmatrix}
      w_1 \\ w_2 \\ w_3
    \end{bmatrix}
    \text,
    \\[2mm]
    \begin{bmatrix}
      \rected{$+a_{12}$}{red}  w_2 \rected{$+a_{13}$}{blue}  w_3 \\
      \rected{$-a_{12}$}{red}  w_1 \rected{$+a_{23}$}{green} w_3 \\
      \rected{$-a_{13}$}{blue} w_1 \rected{$-a_{23}$}{green} w_2
    \end{bmatrix}
     & =
    \begin{bmatrix}
      \rected{$v_2$}{blue}  w_3 \rected{$-v_3$}{red}   w_2 \\
      \rected{$v_3$}{red}   w_1 \rected{$-v_1$}{green} w_3 \\
      \rected{$v_1$}{green} w_2 \rected{$-v_2$}{blue}  w_1
    \end{bmatrix}
    \text,
    \hspace{20mm}
    \\[2mm]
    \rvec{v}
     & =
    \begin{bmatrix}
      -a_{23} \\ a_{13}  \\ -a_{12}
    \end{bmatrix}
    \text.
  \end{align*}
\end{note}

\begin{definition}[Vektorinvariáns]
  Egy antiszimmetrikus lineáris transzformáció mindig leírható egy rögzített
  vektorral való keresztszorzással. Ez a vektor a leképezés vektorinvariánsa.
\end{definition}

\begin{statement}[Lineáris leképezés nyoma]
  Egy lineáris transzformáció főátlójában lévő elemek összege minden
  koordinátarendszerben ugyanannyi, tehát a koordináta-transzformáció nem
  befolyásolja. Ezt nevezzük a lineáris leké- pezés nyomának. (trace / spur)
\end{statement}

\begin{note}[Nabla operátor bevezetése]
  Legyen $\rvec f: V \rightarrow V$ függvény. Vegyük ennek a deriváltját.
  \[
    \Diff\rvec f
    =
    \begin{bmatrix}
      \partial_1 f_1 & \partial_2 f_1 & \dots  & \partial_n f_1 \\
      \partial_1 f_2 & \partial_2 f_2 & \dots  & \partial_n f_2 \\
      \vdots         & \vdots         & \ddots & \vdots         \\
      \partial_1 f_n & \partial_2 f_n & \dots  & \partial_n f_n
    \end{bmatrix}
    =
    \begin{bmatrix}
      \grad^\transpose f_1 \\
      \grad^\transpose f_2 \\
      \vdots               \\
      \grad^\transpose f_n \\
    \end{bmatrix}
    \in
    \mathcal M_{n \times n}
    \text.
  \]
  Definiáljuk az alábbi fogalmakat:
  \begin{itemize}
    \item rotáció
          \tabto{2.4cm} – \tabto{3cm}
          $\rot f := \Diff f - \Diff f ^{*}$,

    \item divergencia
          \tabto{2.4cm} – \tabto{3cm}
          $\Div f := \tr \left( \Diff f \right)$.
  \end{itemize}
  $V = \mathbb{R}^3$ esetén:
  \[
    \rot \rvec f
    =
    \begin{bmatrix}
      0                               &
      \partial_2 f_1 - \partial_1 f_2 &
      \partial_3 f_1 - \partial_1 f_3
      \\
      \partial_1 f_2 - \partial_2 f_1 &
      0                               &
      \partial_3 f_2 - \partial_2 f_3
      \\
      \partial_1 f_3 - \partial_3 f_1 &
      \partial_2 f_3 - \partial_3 f_2 &
      0
    \end{bmatrix}
    \text.
  \]
  A mátrix vektorinvariánsa:
  \[
    \rvec{v}
    =
    \begin{bmatrix}
      \partial_2 f_3 - \partial_3 f_2 \\
      \partial_3 f_1 - \partial_1 f_3 \\
      \partial_1 f_2 - \partial_2 f_1
    \end{bmatrix}
    =
    \begin{bmatrix}
      \partial_1 \\ \partial_2 \\ \partial_3
    \end{bmatrix}
    \times
    \begin{bmatrix}
      f_1 \\ f_2 \\ f_3
    \end{bmatrix}
    =
    \nabla \times \rvec f
    \text,
  \]
  ahol $\nabla$ a Nabla operátor (formális differenciálopertor).
\end{note}

\begin{note}
  Gradiens, divergencia és rotáció számítása a Nabla operátorral:
  \[
    \grad \varphi = \nabla \varphi
    \text,\hspace{1cm}
    \Div \rvec v = \scalar{\nabla}{\rvec v}
    \text,\hspace{1cm}
    \rot \rvec v = \nabla \times \rvec v
    \text.
  \]
\end{note}

% \begin{note}[Zérusság]
%   Az alábbi 2 összefüggés mindig teljesül
%   \[
%     \Div \rot \rvec v \equiv 0
%     \text,\hspace{1cm}
%     \rot \grad \varphi \equiv \nvec
%   \]
% \end{note}

\begin{definition}[Skalárpotenciálosság]
  Egy $\rvec v: V \rightarrow V$ vektormező skalárpotenciálos, ha létezik olyan
  $\varphi: V \rightarrow \mathbb R$ skalármező, hogy $\rvec v = \grad \varphi$.
\end{definition}

\begin{definition}[Vektorpotenciálosság]
  Egy $\rvec v: V \rightarrow V$ vektormező vektorpotenciálos, ha létezik olyan
  $\rvec u: V \rightarrow V$ vektormező, hogy $\rvec v = \rot \rvec u$.
\end{definition}

\begin{theorem}
  Legyen $\rvec v: V \rightarrow V$ mindenhol értelmezett, legalább egyszer
  differenciálható vektormező. Ekkor:
  \begin{itemize}
    \item $\rvec v$ skalárpotenciálos
          $\;\Leftrightarrow\;$
          $\rot \rvec v = \nvec$,
    \item $\rvec v$ vektorpotenciálos
          $\;\Leftrightarrow\;$
          $\Div \rvec v = 0$.
  \end{itemize}
\end{theorem}

\begin{identities}
  Legyenek $\varphi, \psi: V \rightarrow \mathbb R$ skalármezők, $\rvec u;
    \rvec v; \rvec w: V \rightarrow V$ vektormezők, $\lambda; \mu \in \mathbb R$
  skalárok.
  \begin{itemize}
    \item Teljesül a linearitás:\vspace{-.33cm}
          \begin{align*}
            \grad \left( \lambda \, \varphi + \mu \, \psi \right)
             & =
            \lambda \grad \varphi + \mu \grad \psi
            \text,
            \\
            \Div \left( \lambda \, \rvec v + \mu \, \rvec w \right)
             & =
            \lambda \Div \rvec v + \mu \Div \rvec w
            \text,
            \\
            \rot \left( \lambda \, \rvec v + \mu \, \rvec w \right)
             & =
            \lambda \rot \rvec v + \mu \rot \rvec w
            \text.
          \end{align*}
    \item Zérusság:\vspace{-.33cm}
          \begin{align*}
            \rot \grad \varphi & \equiv \nvec
            \text,
            \\
            \Div \rot \rvec v  & \equiv 0
            \text.
          \end{align*}
    \item Deriválási szabályokhoz hasonlóan:
          \begin{align*}
            \grad \left( \varphi \, \psi \right)
             & =
            \varphi \, \grad \psi + \psi \, \grad \varphi
            \text,
            \\
            \Div \left( \varphi \, \rvec v \right)
             & =
            \varphi \, \Div \rvec v + \scalar{\grad \varphi}{\rvec v}
            \text,
            \\
            \rot \left( \varphi \, \rvec v \right)
             & =
            \varphi \, \rot \rvec v + \grad \varphi \times \rvec v
            \text.
          \end{align*}
    \item Egyéb szabályok:\vspace{-.33cm}
          \begin{align*}
            \rot \rot \rvec v
             & =
            \grad \Div \rvec v - \Delta \rvec v
            \text,
            \\
            \rot \left( \rvec u \times \rvec v \right)
             & =
            \rvec u \, \Div \rvec v - \rvec v \, \Div \rvec u
            + (\Diff \rvec u) \rvec v - (\Diff \rvec v) \rvec u
            \text,
            \\
            \Div \left( \rvec u \times \rvec v \right)
             & =
            \scalar{\rvec v}{\rot \rvec u} - \scalar{\rvec u}{\rot \rvec v}
            \text,
            \\
            \grad \left( \scalar{\rvec u}{\rvec v} \right)
             & =
            \rvec u \times \rot \rvec v + \rvec v \times \rot \rvec u
            + (\Diff \rvec u) \rvec v + (\Diff \rvec v) \rvec u
          \end{align*}
  \end{itemize}
\end{identities}

\clearpage
\section{Feladatok}

\relativeinclude{../../../exercise/multivariable/operator/curl-rot-calc.hu}
\relativeinclude{../../../exercise/multivariable/operator/grad-calc.hu}
\relativeinclude{../../../exercise/multivariable/operator/inverse-grad.hu}
\relativeinclude{../../../exercise/multivariable/operator/find-potential.hu}

\end{document}
