\documentclass[lang=magyar]{math-handout}

\title{Ismétlés}
\area{Többváltozós analízis}
\subject{Matematika G3}
\date{Utoljára frissítve: \today}
\author{Sándor Tibor}
\docno{1}

\begin{document}
\allowdisplaybreaks

\maketitle

\vspace{1em}

\begin{summary}
  Ebben a leckében a vektoranalízissel kapcsolatos alapfogalmakat fogjuk
  átismételni. Felelevenítjük a vektortér és lineáris leképezés fogalmát,
  és kapcsolatot teremtünk a vektorok és mátrixok között.
\end{summary}

\vspace{-1em}

\section{Elméleti áttekinő}

\vspace{1em}

\begin{definition}[Vektortér]
  Legyen $V$ egy nemüreshalmaz, és $+; \lambda$ két művelet, $\mathbb T$ test.
  $(V; +; \lambda)$ a $\mathbb T$ test feletti vektortér, ha az alábbiak
  teljesülnek:
  \begin{itemize}
    \def\arraystretch{1.2}
    \item $(V; +)$ Abel csoport:\\[1mm]
          \begin{tabular}{p{45mm} l}
            -- asszociatív:        &
            $\rvec a + (\rvec b + \rvec c) = (\rvec a + \rvec b) + \rvec c$,
            \\
            -- kommutatív:         &
            $\rvec a + \rvec b = \rvec b + \rvec a$,
            \\
            -- létezik zérus elem: &
            $\exists \nvec \in V \text{, melyre } \rvec a + \nvec = \rvec a$,
            \\
            -- létezik inverz:     &
            $\forall \rvec a \text{-re } \exists - \rvec a \text{, hogy } \rvec a + (- \rvec a) = \nvec$.
          \end{tabular}
    \item $(V; \lambda)$-ra pedig igaz:\\[1mm]
          \begin{tabular}{p{45mm} l}
            -- asszociatív:      & $(\alpha \beta) \rvec a = \alpha (\beta \rvec a)$.
            \\
            -- egységelem:       & $\varepsilon \in T \text{-re } \varepsilon\rvec a = \rvec a$,
            \\
            -- disztributivitás: & $\alpha(\rvec a + \rvec b) = \alpha \rvec a + \alpha \rvec b$,
            \\
                                 & $(\alpha + \beta) \rvec a = \alpha \rvec a + \beta \rvec a$.
          \end{tabular}
  \end{itemize}
\end{definition}

\begin{definition}[Lineáris függőség / függetlenség]
  A $(V; +; \lambda)$ vektortér $\rvec a_1; \rvec a_2; \dots; \rvec a_n$
  vektorait \textbf{lineárisan függő}nek mondjuk, ha a
  \[
    \lambda_1 \rvec a_1 +
    \lambda_2 \rvec a_2 +
    \dots +
    \lambda_n \rvec a_n =
    \nvec
  \]
  vektoregyenletnek létezik a triviálistól különböző megoldása is.
  \\[2mm]
  Ellenkező esetben \textbf{lineárisan független}ek.
\end{definition}

\begin{definition}[Altér]
  Legyen $(V; +; \lambda)$ a $T$ test feletti vektortér, és
  $\emptyset \neq L \subset V$. $L$-t altérnek nevezzük $V$-ben, ha
  $(L; +; \lambda)$ ugyancsak vektortér.
\end{definition}

\begin{definition}[Generátorrendszer]
  Legyen $\emptyset \neq G \subset V$. Ekkor $G$ által generált altérnek
  nevezzük azt a legszűkebb alteret, amely tartalmazza $G$-t. Ha ez az
  altér maga $V$, akkor $G$ generátorrendszere $V$-nek. ($\mathcal L(G)=V$)
\end{definition}

\begin{definition}[Bázis]
  A $V$ vektortér egy lineárisan független generátorrendszerét a $V$ bázisának
  hívjuk.
\end{definition}

\begin{definition}[Lineáris leképezés]
  Legyenek $V_1$ és $V_2$ ugyanazon $\mathbb T$ test feletti vektorterek. Legyen
  $\varphi: V_1 \rightarrow V_2$ leképezés, melyet lineáris leképezésnek
  nevezünk, ha
  \[
    \varphi ( \alpha \rvec{a} + \beta \rvec{b} )
    = \alpha \cdot \varphi ( \rvec{a} )
    + \beta \cdot \varphi ( \rvec{b} )
  \]
  bármely $\rvec{a} ; \rvec{b} \in V_1$ és
  $\alpha; \beta \in \mathbb{T}$ esetén teljesül.
\end{definition}

\begin{definition}[Homomorfizmus]
  \vspace{-1em}
  \[
    \Hom ( V_1; V_2 ) :=
    \bigset{\varphi : V_1 \rightarrow V_2}{\varphi \; \text{lineáris}}
  \]
\end{definition}

\begin{definition}[Endomorfizmus]
  \vspace{-1em}
  \[
    V_1 = V_2 = V \; \rightarrow \;
    \Hom ( V; V ) = \End (V)
  \]
\end{definition}

\begin{note}[Lineáris leképezések mátrixreprezentációja]
  Legyen $V_1$ és $V_2$ ugyanazon test feletti vektorterek,
  $\dim V_1 = n$ és $\dim V_2 = k$. Ekkor a $\varphi: V_1 \rightarrow V_2$
  leképezést reprezentáló mátrix $n \times k$ dimenziós.
\end{note}

\clearpage
\section{Feladatok}

\relativeinput{../../../exercise/vector-space/base-generator-none.hu}
\relativeinput{../../../exercise/vector-space/linear-independence.hu}
\relativeinput{../../../exercise/vector-space/vector-space-tf.hu}
\relativeinput{../../../exercise/vector-space/linear-map-tf.hu}
\relativeinput{../../../exercise/vector-space/linear-map.hu}

\end{document}
