\documentclass[exercise]{math-standalone}

\begin{document}

\newcommand{\gcc}[1]{\cellcolor{gray!25}{#1}}

\begin{exercise}{Határozzuk meg az alábbi mátrixok rangjait!}
  \[
    \rmat A = \begin{bmatrix}
      1 & 3 & 4 & 5 \\
      3 & 6 & 9 & 9 \\
      2 & 3 & 5 & 4
    \end{bmatrix} % (2)
    \hspace{2cm}
    \rmat B = \begin{bmatrix}
      2  & -2  & 1 & 6  \\
      4  & -4  & 2 & -2 \\
      10 & -10 & 5 & 2
    \end{bmatrix} % (1)
  \]

  \exsol{%
    Először határozzuk meg az $\rmat A$ mátrix rangját!
    \begin{align*}
      \rg \rmat A
       & = \rg
      \begin{bamatrix}{4}{5.5mm}
        1 & 3 & 4 & 5 \\
        3 & 6 & 9 & 9 \\
        2 & 3 & 5 & 4
      \end{bamatrix}
      \;
      \begin{matrix}
        \\\\(+S_1 - S_2)
      \end{matrix}
      \\
       & = \rg
      \begin{bamatrix}{4}{5.5mm}
        1 & 3 & 4 & 5 \\
        3 & 6 & 9 & 9 \\
        0 & 0 & 0 & 0
      \end{bamatrix}
      \;
      \begin{matrix}
        \\(-3S_1)\\\,
      \end{matrix}
      \\
       & = \rg
      \begin{bamatrix}{4}{5.5mm}
        \gcc{1} & \gcc{3}  & \gcc{4}  & \gcc{5}  \\
        0       & \gcc{-3} & \gcc{-3} & \gcc{-6} \\
        0       & 0        & 0        & 0
      \end{bamatrix}
      =2
    \end{align*}

    \tcbline

    Most pedig határozzuk meg $\rmat B$ mátrix rangját!
    \begin{align*}
      \rg \rmat B
       & = \rg
      \begin{bamatrix}{4}{7.5mm}
        2  & -2  & 1 & 6  \\
        4  & -4  & 2 & -2 \\
        10 & -10 & 5 & 2
      \end{bamatrix}
      \;
      \begin{matrix}
        \\\\(-S_1 - 2S_2)
      \end{matrix}
      \\
       & = \rg
      \begin{bamatrix}{4}{7.5mm}
        2  & -2  & 1 & 6  \\
        4  & -4  & 2 & -2 \\
        0 & 0 & 0 & 0
      \end{bamatrix}
      \;
      \begin{matrix}
        \\(-2S_1)\\\,
      \end{matrix}
      \\
       & = \rg
      \begin{bamatrix}{4}{7.5mm}
        \gcc{2} & \gcc{-2} & \gcc{1} & \gcc{6}   \\
        0       & 0        & 0       & \gcc{-14} \\
        0       & 0        & 0       & 0
      \end{bamatrix}
      =2
    \end{align*}
  }
\end{exercise}

\end{document}
