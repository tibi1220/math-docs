\documentclass[exercise]{math-standalone}

\begin{document}

\begin{exercise}{Oldjuk meg az alábbi mátrix-egyenleteket!}
  \[
    \rmat A = \begin{bmatrix}
      2 & 3 \\
      3 & 5
    \end{bmatrix}
    \hspace{2cm}
    \rmat B = \begin{bmatrix}
      1 & 2 \\
      3 & 4 \\
      6 & 3
    \end{bmatrix}
  \]
  \begin{enumerate}[a)]
    \item $\rmat X \cdot \rmat A = \rmat B$
    \item $\rmat A \cdot \rmat X = \rmat B$
    \item $2(\rmat A + \rmat X) = 3(\rmat X - \rmat A^{-1})$
    \item $\rmat B \cdot \rmat B^{\mathsf T} = \rmat A \cdot \rmat X$
  \end{enumerate}

  \exsol[18.25cm]{
    \begin{enumerate}[a)]
      \item $\rmat X \cdot \rmat A = \rmat B$

            Rendezzük $\rmat X$-re az egyenletet, vagyis szorozzuk meg az
            egyenlet mindkét oldalát $\rmat A$ inverzével. Ekkor az alábbi
            egyenletet kapjuk:
            \[
              \rmat X = \rmat B \cdot \rmat A^{-1}
              \text{.}
            \]
            Az $\rmat A$ mátrix inverzét már korábban meghatároztuk.
            Az egyenletbe behelyettesítve:
            \[
              \rmat X
              = \rmat B \cdot\rmat A^{-1} = \begin{bmatrix}
                1 & 2 \\
                3 & 4 \\
                6 & 3
              \end{bmatrix} \begin{bmatrix}
                5  & -3 \\
                -3 & 2
              \end{bmatrix} = \begin{bmatrix}
                5 - 6   & -3 + 4  \\
                15 - 12 & -9 + 8  \\
                30-9    & -18 + 6
              \end{bmatrix} = \begin{bmatrix}
                -1 & 1   \\
                3  & -1  \\
                21 & -12
              \end{bmatrix}
              \text.
            \]

            \tcbline
      \item $\rmat A \cdot \rmat X = \rmat B$

            Ez az egyenlet nem megoldható, hiszen a $\rmat A^{-1} \rmat B$
            szorzás nem értelmezett, hiszen $\rmat A^{-1} \in
              \mathbb R^{2 \times 2}$, $\rmat B \in \mathbb R^{3 \times 2}$,
            vagyis az $\rmat A^{-1}$ mátrix oszlopainak száma nem egyezik meg
            a $\rmat B$ mátrix sorainak számával.

            \tcbline
      \item $2(\rmat A + \rmat X) = 3(\rmat X - \rmat A^{-1})$

            Rendezzük $\rmat X$-re az egyenletet:
            \begin{gather*}
              2 \rmat A + 2 \rmat X = 3 \rmat X - 3 \rmat A^{-1}
              \quad \rightarrow \quad
              \rmat X = 2 \rmat A + 3 \rmat A^{-1}
              \\
              \rmat X = 2 \begin{bmatrix}
                2 & 3 \\ 3 & 5
              \end{bmatrix} + 3 \begin{bmatrix}
                5 & -3 \\ -3 & 2
              \end{bmatrix} = \begin{bmatrix}
                19 & -3 \\ -3 & 16
              \end{bmatrix}
            \end{gather*}

            \tcbline
      \item $\rmat B \cdot \rmat B^{\mathsf T} = \rmat A \cdot \rmat X$

            Az egyenlet nem megoldható, hiszen $(\rmat B \cdot \rmat B^{\mathsf T})
              \in \mathbb R^{3 \times 3}$, $\rmat A \in \mathbb R^{2 \times 2}$.
    \end{enumerate}
  }
\end{exercise}

\end{document}
