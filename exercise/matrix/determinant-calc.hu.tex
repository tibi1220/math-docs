\documentclass[exercise]{math-standalone}

\begin{document}

\begin{exercise}{Határozzuk meg az alábbi mátrixok determinánsát!}
  \[
    \rmat A = \begin{bmatrix}
      1 & 4 & 8 \\
      0 & 1 & 0 \\
      2 & 4 & 6
    \end{bmatrix}
    \hspace{2cm}
    \rmat B = \begin{bmatrix}
      a  & 3  & x  \\
      -a & -2 & x  \\
      0  & 1  & 2x \\
    \end{bmatrix}
  \]

  \exsol{
    Határozzuk meg az $\rmat A$ mátrix determinánsát a kifejtési tétel
    segítségével. Mivel a második sor csak 1 nemzérus elemet tartalmaz, ezért
    innen fogunk kiindulni:
    \[
      \det \rmat A
      = \begin{vmatrix}
        1 & 4 & 8 \\
        0 & 1 & 0 \\
        2 & 4 & 6
      \end{vmatrix}
      = 1 \cdot \begin{vmatrix}
        1 & 8 \\
        2 & 6
      \end{vmatrix}
      = 1 \cdot (1 \cdot 6 - 2 \cdot 8)
      = -10
      \text.
    \]

    \tcbline

    A $\rmat B$ mátrix esetében vegyük észre, hogy a harmadik sor elemei
    pont az előző két sor megfelelő elemeinek összegei. Ebből következik,
    hogy a mátrix rangja nem maximális, vagyis $\det \rmat B = 0$.
  }
\end{exercise}

\end{document}
