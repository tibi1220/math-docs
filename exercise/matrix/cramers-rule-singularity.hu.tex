\documentclass[exercise]{math-standalone}

\begin{document}

\begin{exercise}{%
    Milyen együtthatók esetén nem lesz megoldható az egyenletrendszer
    Cra\-mer-sza\-bállyal?
  }
  \[
    \begin{array}{*{7}{c}}
      a x_{1}  & + & -2 x_{2} & + & 1 x_{3} & = & 1 \\
      -4 x_{1} & + & b x_{2}  & + & 2 x_{3} & = & 2 \\
      -8 x_{1} & + & 3b x_{2} & + & 4 x_{3} & = & 3
    \end{array}
  \]

  \exsol{%
    A Cramer-szabály akkor alkalmazható, ha az együttható mátrix reguláris.
    Írjuk fel a mátrixot, majd vizsgáljuk meg, hogy milyen együtthatók esetén
    lesz szinguláris, hiszen ilyen esetekben nem használható a Cramer-szabály:
    \[
      \rmat A = \begin{bmatrix}
        a  & -2 & 1 \\
        -4 & b  & 2 \\
        -8 & 3b & 4
      \end{bmatrix}
      \text.
    \]
    Az mátrix az alábbi esetekben lesz szinguláris:
    \begin{itemize}
      \item ha az első és a harmadik oszlop egymás skalárszorosa,
            vagyis ha $a = -2$,
      \item ha a második és a harmadik sor egymás skalárszorosa,
            vagyis ha $b = 0$.
    \end{itemize}
    Tehát a Cramer-szabályt akkor nem tudjuk alkalmazni,
    ha $a = 2$, vagy ha $b = 0$.
  }
\end{exercise}

\end{document}
