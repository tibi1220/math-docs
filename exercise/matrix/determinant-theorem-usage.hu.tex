\documentclass[exercise]{math-standalone}

\begin{document}

\begin{exercise}{Válaszoljuk meg az alábbi rövid kérdéseket!}
  \begin{enumerate}[a)]
    \item Az $\rmat A \in \mathbb R^{3 \times 3}$ mátrix determinánsa 2.
          Legyen $\rmat A' = 3 \rmat A$.
          Mennyi $\det(\rmat A')$ értéke?
    \item A $\rmat B \in \mathbb R^{3 \times 3}$ mátrix determinánsa 4.
          Az első két oszlopát felcseréltük, a harmadikat pedig
          elosztottuk kettővel. Mennyi lesz az így keletkező $\rmat B'$ mátrix
          determinánsa?
    \item Határozzuk meg $\det(\rmat A' \cdot \rmat B')$ értéket!
  \end{enumerate}

  \exsol{%
    \begin{enumerate}[a)]
      \item A $\det (\lambda \rmat A) = \lambda^n \det \rmat A$ tétel alapján:
            \[
              \det (3 \rmat A') = 3^3 \cdot \det \rmat A = 27 \cdot 2 = 54
              \text.
            \]

            \tcbline
      \item Az oszlopok felcserélése miatt a determináns a $(-1)$-szeresére
            változott, majd a harmadik oszlop 2-vel való osztása miatt a felére
            csökkent, vagyis:
            \[
              \det \rmat B' = \det \rmat B \cdot (-1) \cdot (1/2) = -2
              \text.
            \]

            \tcbline
      \item A determinánsok szorzástétele alapján:
            \[
              \det (\rmat A' \cdot \rmat B')
              = \det \rmat A' \cdot \det \rmat B'
              = 54 \cdot (-2)
              = -108
              \text.
            \]
    \end{enumerate}
  }
\end{exercise}

\end{document}
