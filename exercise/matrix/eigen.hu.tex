\documentclass[exercise]{math-standalone}

\begin{document}

\begin{exercise}{%
    Határozzuk meg az alábbi mátrixok sajátértékeit és sajátvektorait!
    A sajátvektorok hosszai legyenek egységnyiek, valamint az első koordinátájuk
    legyen pozitív!
  }
  \[
    \rmat A = \begin{bmatrix}
      2 & 3 \\
      6 & 5
    \end{bmatrix}
    \hspace{1cm}
    \rmat B = \begin{bmatrix}
      1 & 0 \\
      0 & 1
    \end{bmatrix}
    \hspace{1cm}
    \rmat C = \begin{bmatrix}
      -8  & 6  \\
      -15 & 11 \\
    \end{bmatrix}
  \]

  \exsol[21.6cm]{%
  Az $\rmat A$ mátrix sajátértékeinek meghatározásához írjuk fel a
  karakterisztikus egyenletet:
  \[
    \det \left( \rmat A - \lambda \imat \right) = 0
    \text{.}
  \]
  Számítsuk ki ezen determináns értékét:
  \[
    \begin{vmatrix}
      2 - \lambda & 3          \\
      6           & 5 -\lambda
    \end{vmatrix}
    = (2 - \lambda)(5 - \lambda) - 3 \cdot 6
    = 10 - 7\lambda + \lambda^2 - 18
    = \lambda^2 - 7\lambda - 8
    = (\lambda - 8)(\lambda + 1)
    = 0
    \text{.}
  \]
  Vagyis a sajátértékek: $\lambda_1 = 8$ és $\lambda_2 = -1$.

  \vspace{.66em}
  A sajátvektorokat az alábbi egyenlet segítségével kereshetjük:
  \[
    (\rmat A - \lambda_i \imat) \rvec v_i = 0
    \text{.}
  \]

  A $\lambda_1 = 8$-hoz tartozó sajátvektor:
  \[
    \rmat A - \lambda_1 \imat = \begin{bmatrix}
      2 - 8 & 3     \\
      6     & 5 - 8
    \end{bmatrix} = \begin{bmatrix}
      -6 & 3  \\
      6  & -3
    \end{bmatrix}
    \quad \rightarrow \quad
    \left[\begin{matrix}
        -6 & 3  \\
        6  & -3
      \end{matrix}\right.\left|\begin{matrix}
        \,0 \\ \,0
      \end{matrix}\right]
    \quad \sim \quad
    \left[\begin{matrix}
        -6 & 3 \\
        0  & 0
      \end{matrix}\right.\left|\begin{matrix}
        \,0 \\ \,0
      \end{matrix}\right]
    \text{.}
  \]
  Ezek alapján a koordináták közötti viszony:
  \[
    -6v_{11} + 3v_{12} = 0
    \quad \rightarrow \quad
    v_{12} = 2 v_{11}
    \text{.}
  \]
  A sajátvektor paraméteresen, majd egységhosszúra normálva:
  \[
    \rvec v_1 = t_1 \begin{bmatrix}
      1 \\ 2
    \end{bmatrix}
    \quad
    \rightarrow
    \quad
    \uvec v_1 = \frac{1}{\sqrt{1^2 + 2^2}}\begin{bmatrix}
      1 \\ 2
    \end{bmatrix} = \begin{bmatrix}
      1 / \sqrt{5} \\
      2 / \sqrt{5}
    \end{bmatrix}
    \text{.}
  \]

  A $\lambda_2 = -1$-hoz tartozó sajátvektor:
  \[
    \rmat A - \lambda_2 \imat = \begin{bmatrix}
      2 + 1 & 3     \\
      6     & 5 + 1
    \end{bmatrix} = \begin{bmatrix}
      3 & 3 \\
      6 & 6
    \end{bmatrix}
    \quad \rightarrow \quad
    \left[\begin{matrix}
        3 & 3 \\
        6 & 6
      \end{matrix}\right.\left|\begin{matrix}
        \,0 \\ \,0
      \end{matrix}\right]
    \quad \sim \quad
    \left[\begin{matrix}
        1 & 1 \\
        0 & 0
      \end{matrix}\right.\left|\begin{matrix}
        \,0 \\ \,0
      \end{matrix}\right]
    \text{.}
  \]
  Ezek alapján a koordináták közötti viszony:
  \[
    v_{21} + v_{22} = 0
    \quad \rightarrow \quad
    v_{22} = -v_{21}
    \text{.}
  \]
  A sajátvektor paraméteresen, majd egységhosszúra normálva:
  \[
    \rvec v_2 = t_2 \begin{bmatrix}
      1 \\ -1
    \end{bmatrix}
    \quad
    \rightarrow
    \quad
    \uvec v_2 = \frac{1}{\sqrt{1^2 + 1^2}}\begin{bmatrix}
      1 \\ -1
    \end{bmatrix} = \begin{bmatrix}
      1 / \sqrt{2} \\
      -1 / \sqrt{2}
    \end{bmatrix}
    \text{.}
  \]

  \tcbline

  A $\rmat{B}$ mátrixról ránézésre megállapítható, hogy sajátértékei $\lambda_1
    = \lambda_2 = 1$. Keressük meg a sajátvektorait:
  \[
    \rmat B - \lambda_{12} \imat = \begin{bmatrix}
      1 - 1 & 0     \\
      0     & 1 - 1
    \end{bmatrix} = \begin{bmatrix}
      0 & 0 \\
      0 & 0
    \end{bmatrix}
    \quad \rightarrow \quad
    \left[\begin{matrix}
        0 & 0 \\
        0 & 0
      \end{matrix}\right.\left|\begin{matrix}
        \,0 \\ \,0
      \end{matrix}\right]
    \text.
  \]

  Látható, hogy ennek a lineáris egyenletrendszernek végtelen sok megoldása van.
  A sajátvektorok ennek tudatában paraméteresen:
  \[
    \rvec v = \begin{bmatrix}
      t_1 \\ t_2
    \end{bmatrix} = \begin{bmatrix}
      t_1 \\ 0
    \end{bmatrix} + \begin{bmatrix}
      0 \\ t_2
    \end{bmatrix} = t_1 \begin{bmatrix}
      1 \\ 0
    \end{bmatrix} + t_2 \begin{bmatrix}
      0 \\ 1
    \end{bmatrix}
    \text{.}
  \]
  Az sajátvektorok egységnyi hosszúra normálva:
  \[
    \uvec v_1 = \begin{bmatrix}
      1 \\ 0
    \end{bmatrix}
    \text{,}
    \quad \text{és} \quad
    \uvec v_2 = \begin{bmatrix}
      0 \\ 1
    \end{bmatrix}
    \text{.}
  \]

  \tcbline

  A $\rmat C$ mátrix sajátértékeinek meghatározásához írjuk fel a
  karakterisztikus egyenletet:
  \[
    \det \left( \rmat C - \lambda \imat \right) = 0
    \text{.}
  \]
  Számítsuk ki ezen determináns értékét:
  \[
    \begin{vmatrix}
      -8 - \lambda & 6           \\
      -15          & 11 -\lambda
    \end{vmatrix}
    = (-8 - \lambda)(11 - \lambda) - (-15) \cdot 6
    = \lambda^2 - 3\lambda + 2
    = (\lambda - 2)(\lambda - 1)
    = 0
    \text{.}
  \]
  Vagyis a sajátértékek: $\lambda_1 = 1$ és $\lambda_2 = 2$.

  \vspace{.66em}
  A sajátvektorokat az alábbi egyenlet segítségével kereshetjük:
  \[
    (\rmat C - \lambda_i \imat) \rvec v_i = 0
    \text{.}
  \]

  A $\lambda_1 = 1$-hez tartozó sajátvektor:
  \[
    \rmat C - \lambda_1 \imat = \begin{bmatrix}
      -8 - 1 & 6      \\
      -15    & 11 - 1
    \end{bmatrix} = \begin{bmatrix}
      -9  & 6  \\
      -15 & 10
    \end{bmatrix}
    \quad \rightarrow \quad
    \left[\begin{matrix}
        -9  & 6  \\
        -15 & 10
      \end{matrix}\right.\left|\begin{matrix}
        \,0 \\ \,0
      \end{matrix}\right]
    \quad \sim \quad
    \left[\begin{matrix}
        -9 & 6 \\
        0  & 0
      \end{matrix}\right.\left|\begin{matrix}
        \,0 \\ \,0
      \end{matrix}\right]
    \text{.}
  \]
  Ezek alapján a koordináták közötti viszony:
  \[
    -9v_{11} + 6v_{12} = 0
    \quad \rightarrow \quad
    2 v_{12} = 3 v_{11}
    \text{.}
  \]
  A sajátvektor paraméteresen, majd egységhosszúra normálva:
  \[
    \rvec v_1 = t_1 \begin{bmatrix}
      2 \\ 3
    \end{bmatrix}
    \quad
    \rightarrow
    \quad
    \uvec v_1 = \frac{1}{\sqrt{2^2 + 3^2}}\begin{bmatrix}
      2 \\ 3
    \end{bmatrix} = \begin{bmatrix}
      2 / \sqrt{13} \\
      3 / \sqrt{13}
    \end{bmatrix}
    \text{.}
  \]

  A $\lambda_2 = 2$-höz tartozó sajátvektor:
  \[
    \rmat C - \lambda_2 \imat = \begin{bmatrix}
      -8 - 2 & 6      \\
      -15    & 11 - 2
    \end{bmatrix} = \begin{bmatrix}
      -10 & 6 \\
      -15 & 9
    \end{bmatrix}
    \quad \rightarrow \quad
    \left[\begin{matrix}
        -10 & 6 \\
        -15 & 9
      \end{matrix}\right.\left|\begin{matrix}
        \,0 \\ \,0
      \end{matrix}\right]
    \quad \sim \quad
    \left[\begin{matrix}
        -10 & 6 \\
        0   & 0
      \end{matrix}\right.\left|\begin{matrix}
        \,0 \\ \,0
      \end{matrix}\right]
    \text{.}
  \]
  Ezek alapján a koordináták közötti viszony:
  \[
    -10v_{21} + 6v_{22} = 0
    \quad \rightarrow \quad
    3v_{22} = 5v_{21}
    \text{.}
  \]
  A sajátvektor paraméteresen, majd egységhosszúra normálva:
  \[
    \rvec v_2 = t_2 \begin{bmatrix}
      3 \\ 5
    \end{bmatrix}
    \quad
    \rightarrow
    \quad
    \uvec v_2 = \frac{1}{\sqrt{3^2 + 5^2}}\begin{bmatrix}
      3 \\ 5
    \end{bmatrix} = \begin{bmatrix}
      3 / \sqrt{34} \\
      5 / \sqrt{34}
    \end{bmatrix}
    \text{.}
  \]
  }
\end{exercise}

\end{document}
