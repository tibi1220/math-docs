\documentclass[exercise]{math-standalone}

\begin{document}

\begin{exercise}{Lineárisak-e az alábbi leképezések?}
  \begin{enumerate}[a)]
    \item \(
          \varphi: \mathbb R^2 \rightarrow \mathbb R^2;
          (x; y) \mapsto (e^x; y)
          \),
    \item \(
          \varphi: \mathbb R^2 \rightarrow \mathbb R^2;
          (x; y) \mapsto (42; 0)
          \),
    \item \(
          \varphi: \mathbb R^2 \rightarrow \mathbb R^2;
          (x; y) \mapsto (\arctan \tan x; -12)
          \),
    \item \(
          \varphi: \mathbb R^2 \rightarrow \mathbb R^2;
          (x; y) \mapsto (y + x; 2x)
          \),
  \end{enumerate}

  \exsol{%
    \begin{enumerate}[a)]
      \item Nem lineáris, hiszen az $e^x$ függvény sem az.
      \item Nem lineáris, hiszen $\alpha \cdot 42 = 42$, nem igaz tetszőleges
            $\alpha$ számra,
      \item Nem lineáris, hiszen $\alpha \arctan \tan x \neq \arctan \tan \alpha x$.
      \item Lineáris, hiszen teljesülnek az alábbi feltételek:
    \end{enumerate}
    \[
      \varphi \begin{bmatrix}
        x_1 + x_2 \\ y_1 + y_2
      \end{bmatrix} = \begin{bmatrix}
        (y_1 + y_2) + (x_1 + x_2) \\ 2 (x_1 + x_2)
      \end{bmatrix} = \begin{bmatrix}
        y_1 + x_1 \\ 2x_1
      \end{bmatrix} + \begin{bmatrix}
        y_2 + x_2 \\ 2x_2
      \end{bmatrix} = \varphi \begin{bmatrix}
        x_1 \\ y_1
      \end{bmatrix} + \varphi \begin{bmatrix}
        x_1 \\ y_1
      \end{bmatrix}
      \text,
    \]\[
      \varphi \begin{bmatrix}
        \alpha x \\ \alpha y
      \end{bmatrix} = \begin{bmatrix}
        \alpha y + \alpha x \\ 2\alpha x
      \end{bmatrix} = \begin{bmatrix}
        \alpha(y + x) \\ \alpha(2x)
      \end{bmatrix} = \alpha \begin{bmatrix}
        y + x \\ 2x
      \end{bmatrix} = \alpha \varphi \begin{bmatrix}
        x \\ y
      \end{bmatrix}
      \text.
    \]
  }
\end{exercise}

\end{document}
