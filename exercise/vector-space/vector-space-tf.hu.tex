\documentclass[exercise]{math-standalone}

\begin{document}

\begin{exercise}{%
    Döntsük el, hogy alteret alkotnak-e az alábbi számhármasok
    $\mathbb R^3$-ban?
  }
  \begin{enumerate}[a)]
    \item $Q_1 = \bigset{(x_1; x_2; x_3)}{x_1 + x_2 = 0}$
    \item $Q_2 = \bigset{(x_1; x_2; x_3)}{x_1 = \pi}$
    \item $Q_3 = \bigset{(x_1; x_2; x_3)}{x_1 = x_2 = x_3}$
    \item $Q_4 = \bigset{(x_1; x_2; x_3)}{x_1 = (x_2)^2}$
  \end{enumerate}

  \exsol{%
    \begin{enumerate}[a)]
      \item Igen, mert a műveletek sosem mutatnak ki a vektortérből:
            \begin{enumerate}[1)]
              \item Összeadásra való ellenőrzés:
                    $\forall \rvec a ; \rvec b \in Q_1:
                      \rvec a + \rvec b \overset{?}{\in} Q_1$.
                    \[
                      \rvec a := \begin{pmatrix} a \\ -a \\ c \end{pmatrix}
                      \quad
                      \rvec b := \begin{pmatrix} b \\ -b \\ d \end{pmatrix}
                      \quad \rightarrow \quad
                      \rvec a + \rvec b
                      = \begin{pmatrix} a + b \\ -a - b \\ c + d \end{pmatrix}
                      = \begin{pmatrix} a + b \\ -(a+b) \\ c + d \end{pmatrix}
                      \in Q_1
                    \]

              \item Skalárral való szorzásra való ellenőrzés:
                    $\forall \rvec a \in Q_1 \text{ és } \forall \lambda \in \mathbb R:
                      \lambda \rvec a \overset{?}{\in} Q_1$.
                    \[
                      \rvec a := \begin{pmatrix} a \\ -a \\ c \end{pmatrix}
                      \quad \rightarrow \quad
                      \lambda \rvec a
                      = \begin{pmatrix} \lambda a \\ \lambda (-a) \\ \lambda c \end{pmatrix}
                      = \begin{pmatrix} \lambda a \\ -(\lambda a) \\ \lambda c \end{pmatrix}
                      \in Q_1
                    \]
            \end{enumerate}

            \tcbline
      \item Nem, hiszen $\lambda \pi \neq \pi$ (, kivéve ha $\lambda = 1$).
            % \[
            %   \lambda \begin{pmatrix} \pi \\ a \\ b \end{pmatrix}
            %   = \begin{pmatrix} \lambda \pi \\ \lambda a \\ \lambda b \end{pmatrix}
            %   \not\in Q_2
            %   \text.
            % \]

            \tcbline
      \item Igen, mert a műveletek sosem mutatnak ki a vektortérből:
            \begin{enumerate}[1)]
              \item Összeadásra való ellenőrzés:
                    $\forall \rvec a ; \rvec b \in Q_3:
                      \rvec a + \rvec b \overset{?}{\in} Q_1$.
                    \[
                      \rvec a := \begin{pmatrix} a \\ a \\ a \end{pmatrix}
                      \quad
                      \rvec b := \begin{pmatrix} b \\ b \\ b \end{pmatrix}
                      \quad \rightarrow \quad
                      \rvec a + \rvec b
                      = \begin{pmatrix} a + b \\ a + b \\ a + b \end{pmatrix}
                      \in Q_3
                    \]

              \item Skalárral való szorzásra való ellenőrzés:
                    $\forall \rvec a \in Q_3 \text{ és } \forall \lambda \in \mathbb R:
                      \lambda \rvec a \overset{?}{\in} Q_3$.
                    \[
                      \rvec a := \begin{pmatrix} a \\ a \\ a \end{pmatrix}
                      \quad \rightarrow \quad
                      \lambda \rvec a
                      = \begin{pmatrix} \lambda a \\ \lambda a \\ \lambda a \end{pmatrix}
                      \in Q_3
                    \]
            \end{enumerate}

            \tcbline
      \item Nem, hiszen $(a + b)^2 \neq a^2 + b^2$ (, kivéve ha $ab = 0$).
            % \[
            %   \begin{pmatrix} a^2 \\ a \\ c \end{pmatrix} +
            %   \begin{pmatrix} b^2 \\ b \\ d \end{pmatrix} =
            %   \begin{pmatrix} a^2 + b^2 \\ a + b \\ c + d \end{pmatrix} \neq
            %   \begin{pmatrix} (a + b)^2 \\ a + b \\ c + d \end{pmatrix} \text.
            % \]
    \end{enumerate}
  }
\end{exercise}

\end{document}
