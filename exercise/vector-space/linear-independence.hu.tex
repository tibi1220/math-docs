\documentclass[exercise]{math-standalone}

\begin{document}

\begin{exercise}{Vektorteres kiskérdések}
  \begin{enumerate}[a)]
    \item Bázist alkot-e az alábbi vektorhármas $\mathbb R^3$-ban?
          \[
            \Big\{\; (1;2;0);\; (2;0;1);\; (0;1;2) \;\Big\}
          \]

    \item Lineárisan független-e a valós együtthatós polinomok vektorterében
          az alábbi vektorhármas?
          \[
            \Big\{\; x^2 - x - 2 ;\; x + 1 ;\; x^2 + x \;\Big\}
          \]

    \item Hány dimenziós vektorteret alkot az azon legfeljebb 5-ödfokú
          polinomok halmaza, melynek elemeire teljesül, hogy a nulladrendű tag
          együtthatója megegyezik a harmadrendű tag együtthatójával?
  \end{enumerate}

  \exsol[10.75cm]{%
    \begin{enumerate}[a)]
      \item A vektorhármas lineárisan független, ha vegyes szorzatuk nem zérus,
            vagyis $\rvec a \cdot (\rvec b \times \rvec c) \neq 0$:
            \[
              \begin{pmatrix} 1 \\ 2 \\ 0 \end{pmatrix} \cdot \left(
              \begin{pmatrix} 2 \\ 0 \\ 1 \end{pmatrix} \times
              \begin{pmatrix} 0 \\ 1 \\ 2 \end{pmatrix}
              \right)
              =
              \begin{pmatrix} 1 \\ 2 \\ 0 \end{pmatrix} \cdot
              \begin{pmatrix} -1 \\ -4 \\ 2 \end{pmatrix}
              = -1 - 8 = -9 \neq 0
              \text.
            \]
            Megállapíthatjuk, hogy ezen vektorok nem koplanárisak, ebből
            következik, hogy lineárisan függetlenek, vagyis bázist alkotnak
            $\mathbb R^3$-ban.

      \item Ezen vektorhármas \textbf{nem} lineárisan független, hiszen:
            \[
              x^2 - x - 2 = -2(x + 1) + 1(x^2 + x)
              \text.
            \]

      \item A vektortér egy lehetséges bázisa:
            \[
              \Big\{\; 1 + x^3 ;\; x ;\; x^2 ;\; x^4 ;\; x^5 \;\Big\}
              \text.
            \]
            A bázis elemszáma 5, tehát a vektortér 5 dimenziós.
    \end{enumerate}
  }
\end{exercise}

\end{document}
