\documentclass[exercise]{math-standalone}

\begin{document}

\begin{exercise}{%
    Állítsuk elő az alábbi függvények $x_0$ körüli Taylor sorát!
    Adjuk meg a nulladik, első, és $n$-edik tag együtthatóját! Mekkora a
    konvergencia-sugár?
  }
  \newcommand{\tit}[2]{\begin{tabular}{*{2}{p{3.5cm}}} #1 & #2 \end{tabular}}
  \begin{enumerate}[a)]
    \item \tit{$f(x) = \dfrac{x - 3}{x - 5}$}{$x_0 = 2$}
    \item \tit{$g(x) = e^{-3x}$}{$x_0 = -1$}
  \end{enumerate}

  \exsol{%
    \newcommand{\tit}[2]{\begin{tabular}{*{2}{p{3.5cm}}} #1 & #2 \end{tabular}}
    \begin{enumerate}[a)]
      \item \tit{$f(x) = \dfrac{x - 3}{x - 5}$}{$x_0 = 2$}\\[3mm]
            %
            Hozzuk a függvényt egyszerűbb alakra:
            \[
              f(x)
              = \frac{x - 3}{x - 5}
              = \frac{x - 5 + 2}{x - 5}
              = 1 + \frac{2}{x - 5}
              \text.
            \]
            Állítsuk elő a függvény deriváltjait, és értékeljük ki őket az
            $x_0 = 2$ pontban.
            \begin{alignat*}{9}
               & f(x)       &  & = 1 + \frac{2}{x-5} \hspace{2cm}     &  & f(x_0)       &  & = +\frac{1}{3}                                                 \\
               & f'(x)      &  & = \frac{-2}{(x-5)^2} \hspace{2cm}    &  & f'(x_0)      &  & = -\frac{2}{9}                                                 \\
               & f''(x)     &  & = \frac{4}{(x-5)^3}                  &  & f''(x_0)     &  & = -\frac{4}{27}                                                \\
               & f'''(x)    &  & = \frac{-12}{(x-5)^4}                &  & f'''(x_0)    &  & = -\frac{4}{27}                                                \\
              % & f''''(x)   &  & = \frac{48}{(x-5)^5}                 &  & f''''(x_0)   &  & = -\frac{16}{81}                                               \\
               & f^{(n)}(x) &  & = \frac{(-1)^n 2 \, n!}{(x-5)^{n+1}} &  & f^{(n)}(x_0) &  & = \frac{(-1)^n 2 \, n!}{(-3)^{n+1}} = \frac{-2 \, n!}{3^{n+1}}
            \end{alignat*}
            Ezek alapján a Taylor-sor:
            \begin{align*}
              T(x)
               & = \frac{1}{3 \cdot 0!}
              - \frac{2}{9 \cdot 1!} (x - 2)
              - \frac{4}{27\cdot 2!} (x - 2)^2
              + \dots
              + \frac{-2 \, n!}{3^{n+1} n!} (x - 2)^n
              \\[2mm]
               & = \frac{1}{3}
              - \frac{2}{9} (x - 2)
              - \frac{2}{27} (x - 2)^2
              + \dots
              + \frac{-2}{3^{n+1}} (x - 2)^n
              \\[2mm]
               & = \frac{1}{3}
              + \sum_{n = 1}^\infty \frac{-2}{3^{n+1}} (x - 2)^n
              \text.
            \end{align*}
            A keresett tényezők:
            \[
              a_0 = \frac{1}{3}
              \text, \qquad
              a_1 = -\frac{2}{9}
              \text, \qquad
              a_n = \frac{-2}{3^{n+1}}
              \text.
            \]
            A konvergencia-sugár:
            \[
              r
              = \frac{1}{\displaystyle\limsup_{n \rightarrow \infty} \left|
                \dfrac{a_{n+1}}{a_n}
                \right|
              }
              = \frac{1}{\displaystyle\limsup_{n \rightarrow \infty} \left|
                \dfrac{-2}{3^{n+2}} \dfrac{3^{n+1}}{-2}
                \right|
              }
              = \frac{1}{\displaystyle\lim_{n \rightarrow \infty}\dfrac{1}{3}}
              = \frac{1}{1/3}
              = 3
              \text.
            \]

      \item \tit{$g(x) = e^{-3x}$}{$x_0 = 1$}\\[3mm]
            %
            Állítsuk elő a függvény deriváltjait, és értékeljük ki őket az
            $x_0 = 1$ pontban.
            \begin{alignat*}{9}
               & f(x)       &  & = e^{-3x}                 &  & f(x_0)       &  & = e^{-3}        \\
               & f'(x)      &  & = -3xe^{-3x} \hspace{2cm} &  & f'(x_0)      &  & = -3 e^{-3}     \\
               & f''(x)     &  & = 9x^2e^{-3x}             &  & f''(x_0)     &  & = 9 e^{-3}      \\
               & f^{(n)}(x) &  & = (-3x)^n e^{-3x}         &  & f^{(n)}(x_0) &  & = (-3)^n e^{-3}
            \end{alignat*}
            Ezek alapján a Taylor-sor:
            \begin{align*}
              T(x)
               & = \frac{e^{-3}}{0!}
              + \frac{-3e^{-3}}{1!}(x - 1)
              + \frac{9e^{-3}}{2!}(x - 1)^2
              + \dots
              + \frac{(-3)^n e^{-3}}{n!} (x - 1)^n
              \\
               & = \sum_{n = 0}^\infty
              \frac{(-3)^n e^{-3}}{n!}(x - 1)^n
              \text.
            \end{align*}
            A keresett tényezők:
            \[
              a_0 = e^{-3}
              \text, \qquad
              a_1 = -3 e^{-3}
              \text, \qquad
              a_n = \frac{(-3)^n e^{-3}}{n!}
              \text.
            \]
            A konvergencia-sugár:
            \[
              r
              = \frac{1}{\displaystyle\limsup_{n \rightarrow \infty} \left|
                \dfrac{a_{n+1}}{a_n}
                \right|
              }
              = \frac{1}{\displaystyle\limsup_{n \rightarrow \infty} \left|
                \dfrac{(-3)^{n+1} \vphantom{e^{-3}}}{(n+1)!}
                \dfrac{n!}{(-3)^n \vphantom{e^{-3}}}
                \right|}
              = \frac{1}{\displaystyle\lim_{n \rightarrow \infty} \left|
                \dfrac{-3}{n+1}
                \right|}
              = \frac{1}{0}
              = \infty
              \text.
            \]
    \end{enumerate}
  }
\end{exercise}

\end{document}
