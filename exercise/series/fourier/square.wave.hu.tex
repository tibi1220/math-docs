\documentclass[exercise]{math-standalone}

\begin{document}

\begin{exercise}{%
    Állítsuk elő a $[-\pi; \pi]$ intervallumon az alábbi, $2\pi$ szerint
    periodikus függvények Fourier-sorát! Adjuk meg az összegfüggvények
    $9 \pi$-ben felvett értékét, valamint az $a_0$, $a_5$ és $b_5$
    együtthatókat!
  }
  \begin{enumerate}[a)]
    \item $f(x) = \sgn x$
    \item $g(x) = 1 - 2\sgn x$
    \item $h(x) = a + b \sgn x$
  \end{enumerate}

  \exsol{%
    A Fourier-sor az alábbi alakot veszi fel:
    \[
      f(x) = a_0 + \sum_{n = 1}^{\infty} \left(
      a_n \cos nx + b_n \sin nx
      \right)
      \text.
    \]

    Az egyes együtthatók az alábbi képletekkel számíthatóak:
    \begin{align*}
      a_0 & =
      \frac{1}{2\pi} \int_0^{2\pi} f(x) \, \mathrm d x
      \text,
      \\
      a_n & =
      \frac{1}{\pi} \int_0^{2\pi} f(x) \cos(n x) \, \mathrm d x
      \text,
      \\
      b_n & =
      \frac{1}{\pi} \int_0^{2\pi} f(x) \sin(n x) \, \mathrm d x
      \text.
    \end{align*}

    \tcbline
    %
    Egyszerű eset: $f(x) = \sgn x$. \\[3mm]
    %
    Ebben az esetben a függvényünk tisztán páratlan, vagyis $a_0 = a_n = 0$.
    Csak szinuszos együtthatóink vannak. Határozzuk meg ezeket:
    \begin{align*}
      b_n
       & = \frac{1}{\pi} \int_0^{2\pi} \sgn x \sin(n x) \, \mathrm d x
      \\
       & = \frac{1}{\pi} \left(
      \int_0^{\pi} \sin(n x) \, \mathrm d x -
      \int_\pi^{2\pi} \sin(n x) \, \mathrm d x
      \right)
      \\
       & = \frac{1}{\pi} \left(
      \Bigg[ \frac{-\cos nx}{n} \Bigg]_{0}^{\pi} -
      \Bigg[ \frac{-\cos nx}{n} \Bigg]_{\pi}^{2\pi}
      \right)
      \\
       & = \frac{1}{\pi} \left(
      \frac{%
        - \cos \pi n
        + \cos 0
        + \cos 2 \pi n
        - \cos \pi n
      }{n}
      \right)
      \\
       & = \frac{2}{\pi} \left(
      \frac{1 - \cos \pi n}{n}
      \right)
      = \begin{cases}
          \sfrac{4}{n \pi}, & \text{ha } n \text{ páratlan,} \\
          0,                & \text{ha } n \text{ páros.}
        \end{cases}
    \end{align*}

    Vagyis a páros indexű, szinuszos együtthatók zérusak, a páratlan indexűek
    értéke pedig:
    \[
      b_n = \frac{4}{n\pi}
      \text.
    \]

    A keresett együtthatók értéke tehát:
    \[
      a_0 = 0
      \text,
      \quad
      a_5 = 0
      \text,
      \quad
      b_5 = \frac{4}{5 \pi}
      \text.
    \]

    Az összegfüggvény $9\pi$-ben felvett értéke:
    \[
      F(9 \pi) = F(\pi) = \frac{f(\pi^-) + f(\pi^+)}{2} = \frac{1-1}{2} = 0
      \text.
    \]

    \tcbline

    Bonyolultabb eset: $g(x) = 1 - 2\sgn x$. \\[3mm]
    %
    Ebben az esetben a függvény nem páratlan, viszont azzá tehető. Ha az $y$
    tengely mentén negatív irányban eltoljuk egy egységnyit. Keressük most ennek
    a $g^*(x) = -2 \sgn x$ függvénynek a Fourier együtthatóit. Mivel ez a
    függvény már páratlan, ezért csak szinuszos együtthatói lesznek:
    \begin{align*}
      b_n
       & = \frac{1}{\pi} \int_0^{2\pi} -2\sgn x \sin(n x) \, \mathrm d x
      \\
       & = \frac{-2}{\pi} \int_0^{2\pi} \sgn x \sin(n x) \, \mathrm d x
      \\
       & = \frac{-2}{\pi} \left(
      \int_0^{\pi} \sin(n x) \, \mathrm d x -
      \int_\pi^{2\pi} \sin(n x) \, \mathrm d x
      \right)
      \\
       & = \frac{-2}{\pi} \left(
      \Bigg[ \frac{-\cos nx}{n} \Bigg]_{0}^{\pi} -
      \Bigg[ \frac{-\cos nx}{n} \Bigg]_{\pi}^{2\pi}
      \right)
      \\
       & = \frac{-2}{\pi} \left(
      \frac{%
        - \cos \pi n
        + \cos 0
        + \cos 2 \pi n
        - \cos \pi n
      }{n}
      \right)
      \\
       & = \frac{-4}{\pi} \left(
      \frac{1 - \cos \pi n}{n}
      \right)
      = \begin{cases}
          \sfrac{-8}{n \pi}, & \text{ha } n \text{ páratlan,} \\
          0,                 & \text{ha } n \text{ páros.}
        \end{cases}
    \end{align*}

    Vagyis a páros indexű, szinuszos együtthatók zérusak, a páratlan indexűek
    értéke pedig:
    \[
      b_n = \frac{-8}{n\pi}
      \text.
    \]

    Az eredeti függvény együtthatói egyetlen kivétellel megegyeznek $g^*(x)$
    függvény együtthatóival, hiszen az eltolás miatt a $g(x)$ függvénynek
    lesz egy $a_0 = 1$-es együtthatója is. A keresett együtthatók tehát:
    \[
      a_0 = 1
      \text,
      \quad
      a_5 = 0
      \text,
      \quad
      b_5 = \frac{-8}{5 \pi}
      \text.
    \]

    Az összegfüggvény $9\pi$-ben felvett értéke:
    \[
      G(9 \pi) = G(\pi) = \frac{g(\pi^-) + g(\pi^+)}{2} = \frac{-1+3}{2} = 1
      \text.
    \]

    \tcbline

    Általános eset: $h(x) = a + b \sgn x$. \\[3mm]
    %
    Ebben az esetben sem páros a függvényünk, viszont azzá tehető, hogyha
    az $y$ tengely mentén $a$ egységgel lefele toljuk el. $h^*(x) = b \sgn x$
    függvény már páratlan, tehát ennek is csak szinuszos együtthatói lesznek:
    \begin{align*}
      b_n
       & = \frac{1}{\pi} \int_0^{2\pi} b \sgn x \sin(n x) \, \mathrm d x
      \\
       & = \frac{b}{\pi} \int_0^{2\pi} \sgn x \sin(n x) \, \mathrm d x
      \\
       & = \frac{b}{\pi} \left(
      \int_0^{\pi} \sin(n x) \, \mathrm d x -
      \int_\pi^{2\pi} \sin(n x) \, \mathrm d x
      \right)
      \\
       & = \frac{b}{\pi} \left(
      \Bigg[ \frac{-\cos n x}{n} \Bigg]_{0}^{\pi} -
      \Bigg[ \frac{-\cos n x}{n} \Bigg]_{\pi}^{2\pi}
      \right)
      \\
       & = \frac{b}{\pi} \left(
      \frac{%
        - \cos \pi n
        + \cos 0
        + \cos 2 \pi n
        - \cos \pi n
      }{n}
      \right)
      \\
       & = \frac{2b}{\pi} \left(
      \frac{1 - \cos \pi n}{n}
      \right)
      = \begin{cases}
          \sfrac{4b}{n \pi}, & \text{ha } n \text{ páratlan,} \\
          0,                 & \text{ha } n \text{ páros.}
        \end{cases}
    \end{align*}

    Vagyis a páros indexű, szinuszos együtthatók zérusak, a páratlan indexűek
    értéke pedig:
    \[
      b_n = \frac{4b}{n\pi}
      \text.
    \]

    Az eredeti függvény együtthatói egyetlen kivétellel megegyeznek $h^*(x)$
    függvény együtthatóival, hiszen az eltolás miatt a $h(x)$ függvénynek
    lesz egy $a_0 = a$-s együtthatója is. A keresett együtthatók tehát:
    \[
      a_0 = a
      \text,
      \quad
      a_5 = 0
      \text,
      \quad
      b_5 = \frac{4b}{5 \pi}
      \text.
    \]

    Az összegfüggvény $9\pi$-ben felvett értéke:
    \[
      H(9 \pi) = H(\pi) = \frac{h(\pi^-) + h(\pi^+)}{2} = \frac{(a+b)+(a-b)}{2} = a
      \text.
    \]
  }
\end{exercise}

\end{document}
