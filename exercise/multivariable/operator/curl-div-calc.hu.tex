\documentclass[exercise]{math-standalone}

\begin{document}
\begin{exercise}{%
    Számítsuk ki az alábbi vektormezők divergenciáját és rotációját!
    Hol lesznek forrásmentesek, illetve örvénymentesek?
  }
  \begin{enumerate}[a)]
    \item $\rvec u(\rvec r) = \rvec r$
    \item $\rvec v(\rvec r) = (yz) \,\uvec i + (xz) \,\uvec j + (xy) \,\uvec k$
    \item $\rvec w(\rvec r) = (3xy + z^2) \,\uvec i + (6 e^z) \,\uvec j + (-5x^y) \,\uvec k$
  \end{enumerate}

  \exsol{%
    \begin{enumerate}[a)]
      \item $\rvec u(\rvec r) = \rvec r = \ijk{x}{y}{z}$
            \begin{align*}
              \Div \rvec u
               & = \scalar{\nabla}{\rvec u}
              = \pdv{x}{x} + \pdv{y}{y} + \pdv{z}{z}
              = 1 + 1 + 1 = 3
              \\
              \rot \rvec u
               & = \nabla \times \rvec u
              = \begin{bmatrix}
                  \partial_x \\ \partial_y \\ \partial_z
                \end{bmatrix} \times \begin{bmatrix}
                                       x \\ y \\ z
                                     \end{bmatrix} = \begin{bmatrix}
                                                       0 - 0 \\ 0 - 0 \\ 0 - 0
                                                     \end{bmatrix} = \begin{bmatrix}
                                                                       0 \\ 0 \\ 0
                                                                     \end{bmatrix}
            \end{align*}
            A vektormező sehol sem forrásmentes, de örvénymentes az egész
            értelmezési tartományán.

      \item $\rvec v(\rvec r) = (yz) \,\uvec i + (xz) \,\uvec j + (xy) \,\uvec k$
            \begin{align*}
              \Div \rvec v
               & = \scalar{\nabla}{\rvec v}
              = \pdv{yz}{x} + \pdv{xz}{y} + \pdv{xy}{z}
              = 0 + 0 + 0
              = 0
              \\
              \rot \rvec v
               & = \nabla \times \rvec v
              = \begin{bmatrix}
                  \partial_x \\ \partial_y \\ \partial_z
                \end{bmatrix} \times \begin{bmatrix}
                                       yz \\ xz \\ xy
                                     \end{bmatrix} = \begin{bmatrix}
                                                       0 - 0 \\ 0 - 0 \\ 0 - 0
                                                     \end{bmatrix} = \begin{bmatrix}
                                                                       0 \\ 0 \\ 0
                                                                     \end{bmatrix}
            \end{align*}
            A vektormező az egész értelmezési tartományán örvény- és
            forgásmentes.

      \item $\rvec w(\rvec r) = (3xy + z^2) \,\uvec i + (6 e^z) \,\uvec j + (-5x^y) \,\uvec k$
            \begin{align*}
              \Div \rvec w
               & = \scalar{\nabla}{\rvec w}
              = \pdv{(3xy + z^2)}{x} + \pdv{(6 e^z)}{y} + \pdv{(-5x^y)}{z}
              = 3y + 0 + 0
              = 3y
              \\
              \rot \rvec w
               & = \nabla \times \rvec w
              = \begin{bmatrix}
                  \partial_x \\ \partial_y \\ \partial_z
                \end{bmatrix} \times \begin{bmatrix}
                                       3xy + z^2 \\ 6e^z \\ -5x^y
                                     \end{bmatrix} = \begin{bmatrix}
                                                       -5 x^y \ln x - 6 e^z \\
                                                       2z + 5 y x^{y-1}     \\
                                                       -3x
                                                     \end{bmatrix}
            \end{align*}
            A vektormező forrásmentes az $y = 0$ síkon, de sehol sem
            örvénymentes. ($z$ koordináta: $x = 0$, $x$ koordináta: $x \neq 0$,
            ez ellentmondás.)
    \end{enumerate}
  }
\end{exercise}
\end{document}
