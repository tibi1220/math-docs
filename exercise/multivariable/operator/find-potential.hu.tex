\documentclass[exercise]{math-standalone}

\begin{document}
\begin{exercise}{%
    Vizsgáljuk meg, hogy az alábbi vektormezők skalár- illetve
    vektorpotenciálisak-e! Amennyiben igen, adjuk meg a potenciálfüggvényeket!
    A valós konsztansokat legyenek zérusak, valamint a vektorpotenciált --
    amennyiben létezik -- olyan módon adjuk meg, hogy a harmadik komponense
    zérus legyen.
  }
  \begin{enumerate}[a)]
    \item $\rvec v(\rvec r) = \ijk{y + z}{x + z}{x + y}$
    \item $\rvec w(\rvec r) = \ijk{e^{x + \sin y}}{e^{x + \sin y} \cos y}{0}$
    \item $\rvec u(\rvec r) = \ijk{2zx^3}{3z}{-3x^2z^2}$
  \end{enumerate}

  \exsol{%
    \begin{enumerate}[a)]
      \item $\rvec v(\rvec r) = \ijk{y + z}{x + z}{x + y}$

            \begin{itemize}
              \item A vektormező skalárpotenciálos, ha rotációja zérus:
                    \[
                      \rot \rvec v
                      =
                      \begin{bmatrix}
                        \partial_x \\ \partial_y \\ \partial_z
                      \end{bmatrix}
                      \times
                      \begin{bmatrix}
                        y + z \\ x + z \\ x + y
                      \end{bmatrix}
                      =
                      \begin{bmatrix}
                        \partial_y (x + y) - \partial_z (x + z) \\
                        \partial_z (y + z) - \partial_x (x + y) \\
                        \partial_x (x + z) - \partial_y (y + z)
                      \end{bmatrix}
                      =
                      \begin{bmatrix}
                        1 - 1 \\
                        1 - 1 \\
                        1 - 1
                      \end{bmatrix}
                      =
                      \begin{bmatrix}
                        0 \\ 0 \\ 0
                      \end{bmatrix}
                    \]

                    A potenciálfüggvény:
                    \begin{align*}
                      \varphi(\rvec r)
                       & =
                      \int_0^x v_x(\xi; y; z) \,\diff \xi +
                      \int_0^y v_y(0; \eta; z) \,\diff \eta +
                      \int_0^z v_z(0; 0; \zeta) \,\diff \zeta
                      \\
                       & =
                      \int_0^x (y + z) \,\diff \xi +
                      \int_0^y (0 + z) \,\diff \eta +
                      \int_0^z (0 + 0) \,\diff \zeta
                      \\
                       & =
                      xy + xz + yz + C
                      \text.
                    \end{align*}

                    A kereseett potenciálfüggvény:
                    \[
                      \varPhi (\rvec r)
                      =
                      xy + xz + yz
                      \text.
                    \]

              \item A vektormező vektorpotenciálos, ha divergenciája zérus:
                    \[
                      \Div \rvec v
                      =
                      \pdv{\rvec v}{x} + \pdv{\rvec v}{y} + \pdv{\rvec v}{z}
                      =
                      0 + 0 + 0
                      =
                      0
                      \text.
                    \]

                    A potenciálfüggvény:
                    \begin{align*}
                      V_x(\rvec r)
                       & =
                      \int_0^z v_y(x; y; \zeta) \,\diff \zeta
                      =
                      \int_0^z (x + \zeta) \,\diff \zeta
                      =
                      xz + \frac{z^2}{2} + C_x
                      \text,
                      \\
                      V_y(\rvec r)
                       & =
                      \int_0^x v_z(\xi; y; 0) \,\diff \xi -
                      \int_0^z v_x(x; y; \zeta) \,\diff \zeta
                      \\
                       & =
                      \int_0^x (\xi + y) \,\diff \xi -
                      \int_0^z (y + \zeta) \,\diff \zeta
                      \\
                       & =
                      \frac{x^2}{2} + xy -
                      \frac{z^2}{2} - yz + C_y
                    \end{align*}

                    A keresett vektorpotenciál:
                    \[
                      \rvec V(\rvec r)
                      =
                      %  & =
                      % \ijk{V_x(\rvec r)}{V_y(\rvec r)}{0}
                      % \\
                      %  & =
                      \ijk{xz + \frac{z^2}{2}}
                      {\frac{x^2}{2} + xy - \frac{z^2}{2} - yz}
                      {0}
                      \text.
                    \]
            \end{itemize}

      \item $\rvec w(\rvec r) = \ijk{e^{x + \sin y}}{e^{x + \sin y} \cos y}{0}$
            \begin{itemize}
              \item Egy vektormező skalárpotenciálos, ha rotációja zérus:
                    \[
                      \rot \rvec w
                      =
                      \begin{bmatrix}
                        \partial_x \\ \partial_y \\ \partial_z
                      \end{bmatrix}
                      \times
                      \begin{bmatrix}
                        e^{x + \sin y} \\ e^{x + \sin y} \cos y \\ 0
                      \end{bmatrix}
                      % =
                      % \begin{bmatrix}
                      %   \partial_y 0 - \partial_z e^{x + \sin y} \cos y \\
                      %   \partial_z e^{x + \sin y} - \partial_x 0        \\
                      %   \partial_x e^{x + \sin y} \cos y - \partial_y e^{x + \sin y}
                      % \end{bmatrix}
                      =
                      \begin{bmatrix}
                        0 - 0 \\ 0 - 0 \\ e^{x + \sin y} \cos y - e^{x + \sin y} \cos y
                      \end{bmatrix}
                      =
                      \begin{bmatrix}
                        0 \\ 0 \\ 0
                      \end{bmatrix}
                      \text.
                    \]

                    A potenciálfüggvény:
                    \begin{align*}
                      \psi(\rvec r)
                       & =
                      \int_0^x w_x(\xi; y; z) \,\diff \xi +
                      \int_0^y w_y(0; \eta; z) \,\diff \eta +
                      \int_0^z w_z(0; 0; \zeta) \,\diff \zeta
                      \\
                       & =
                      \int_0^x e^{\xi + \sin y} \,\diff \xi +
                      \int_0^y e^{\sin \eta} \cos \eta \,\diff \eta +
                      \int_0^z 0 \,\diff \zeta
                      \\
                       & =
                      (e^x - 1) e^{\sin y} + e^{\sin y} - 1 + 0 + C
                      \text.
                    \end{align*}

                    A keresett potenciálfüggvény:
                    \[
                      \varPsi(\rvec r)
                      =
                      e^{x + \sin y}
                      \text.
                    \]

                    Egy vektormező vektorpotenciálos, ha divergenciája zérus:
                    \[
                      \Div \rvec w
                      =
                      \pdv{\rvec w}{x} + \pdv{\rvec w}{y} + \pdv{\rvec w}{z}
                      =
                      e^{x + \sin y} + e^{x + \sin y}(\cos^2 y - \sin y)
                      \neq 0
                      \text.
                    \]

                    Mivel $\Div \rvec w \neq 0$, ezért nem létezik $\rvec w$-nek
                    vektorpotenciálja.
            \end{itemize}

      \item $\rvec u(\rvec r) = \ijk{2zx^3}{3z}{-3x^2z^2}$
            \begin{itemize}
              \item Egy vektormező skalárpotenciálos, ha rotációja zérus.
                    \[
                      \rot \rvec u
                      =
                      \begin{bmatrix}
                        \partial_x \\ \partial_y \\ \partial_z
                      \end{bmatrix}
                      \times
                      \begin{bmatrix}
                        2zx^3 \\ 3z \\ -3x^2z^2
                      \end{bmatrix}
                      =
                      \begin{bmatrix}
                        0 - 3           \\
                        2 x^3 + 6 x z^2 \\
                        0 - 0
                      \end{bmatrix}
                      \neq
                      \nvec
                    \]

                    Mivel $\rot \rvec u \neq \nvec$, ezért $\rvec u$-nak nem
                    létezik skalárpotenciálja.

              \item Egy vektormező vektorpotenciálos, ha divergenciája zérus.
                    \[
                      \Div \rvec u
                      =
                      \pdv{\rvec u}{x} + \pdv{\rvec u}{y} + \pdv{\rvec u}{z}
                      =
                      6 x^2 z + 0 -6 x^2 z
                      =
                      0
                      \text.
                    \]

                    A potenciálfüggvény:
                    \begin{align*}
                      U_x(\rvec r)
                       & =
                      \int_0^z u_y(x; y; \zeta) \,\diff \zeta
                      =
                      \int_0^z  (3 \zeta) \,\diff \zeta
                      =
                      \frac{3 z^2}{2} + C_x
                      \text,
                      \\
                      U_y(\rvec r)
                       & =
                      \int_0^x u_z(\xi; y; 0) \,\diff \xi -
                      \int_0^z u_x(x; y; \zeta) \,\diff \zeta
                      \\
                       & =
                      \int_0^x (0) \,\diff \xi -
                      \int_0^z (2 \zeta x^3)  \,\diff \zeta
                      =
                      0 - x^3 z^2 + C_y
                      \text.
                    \end{align*}

                    A keresett vektorpotenciál:
                    \[
                      \rvec U(\rvec r)
                      = \ijk{\frac{3z^2}{2}}{-x^3 z^2}{0}
                      \text.
                    \]
            \end{itemize}
    \end{enumerate}
  }
\end{exercise}
\end{document}
