\documentclass[exercise]{math-standalone}

\begin{document}
\begin{exercise}{Adjuk meg azon skalármezőket, melyek gradiensei az alábbiak!}
  \begin{enumerate}[a)]
    \item $\rvec F(x;y;z) = \ijk{y \sin xy}{x \sin xy}{3}$
    \item $\rvec G(x;y;z) = \ijk{y/x}{\ln x}{2/z^{2}}$
  \end{enumerate}

  \exsol{%
    \begin{enumerate}[a)]
      \item $\rvec F(x;y;z) = \ijk{y \sin xy}{x \sin xy}{3}$

            \vspace{3mm}
            A gradiens tulajdonságai alapján:
            \[
              F_x = \pdv{f}{x} = y \sin xy
              \text,\hspace{1cm}
              F_y = \pdv{f}{y} = x \sin xy
              \text,\hspace{1cm}
              F_z = \pdv{f}{z} = 3
              \text.
            \]
            % Integráljuk a parciális deriváltakat:
            % \begin{alignat*}{9}
            %   f_1 &  = \int \pdv{f}{x} \,\diff x
            %       && = \int y \sin xy \,\diff x
            %       && = - \cos xy
            %       && + C_1(y;z)
            %   \\
            %   f_2 &  = \int \pdv{f}{y} \,\diff y
            %       && = \int x \sin xy \,\diff y
            %       && = - \cos xy
            %       && + C_2(x;z)
            %   \\
            %   f_3 &  = \int \pdv{f}{z} \,\diff z
            %       && = \int 3 \,\diff z
            %       && = 3z
            %       && + C_3(x;y)
            % \end{alignat*}
            % Tudjuk, hogy $f = f_1 = f_2 = f_3$, a konstansok tehát:
            % \[
            %   C_1(y;z) = 3z + C
            %   \text,\hspace{1cm}
            %   C_2(x;z) = 3z + C
            %   \text,\hspace{1cm}
            %   C_3(x;y) = - \cos xy + C
            %   \text.
            % \]
            Az eredeti függvény az alábbi alakban írható fel:
            \[
              f(\rvec r) =
              \int_0^x F_x(\xi; y; z) \,\diff \xi +
              \int_0^y F_y(0; \eta; z) \,\diff \eta +
              \int_0^z F_z(0; 0; \zeta) \,\diff \zeta
              \text.
            \]
            Végezzük el az integrálást:
            \begin{align*}
              f(\rvec r)
               & =
              \int_0^x y \sin \xi y \,\diff \xi +
              \int_0^y 0 \,\diff \eta +
              \int_0^z 3 \,\diff \zeta
              \\
               & =
              1 - \cos xy + 0 + 3z + C
              \text.
            \end{align*}
            Az eredeti függvény tehát:
            \[
              f(x;y;z) = 3z - \cos xy + C
              \text.
            \]

      \item $\rvec G(x;y;z) = \ijk{y/x}{\ln x}{2/z^{2}}$

            \vspace{3mm}
            A gradiens tulajdonságai alapján:
            \[
              G_x = \pdv{g}{x} = \frac{y}{x}
              \text,\hspace{1cm}
              G_y = \pdv{g}{y} = \ln x
              \text,\hspace{1cm}
              G_z = \pdv{g}{z} = \frac{2}{z^2}
              \text.
            \]
            % Integráljuk a parciális deriváltakat:
            % \begin{alignat*}{9}
            %   g_1 &  = \int \pdv{g}{x} \,\diff x
            %       && = \int \frac{y}{x} \,\diff x
            %       && = y \ln x
            %       && + C_1(y;z)
            %   \\
            %   g_2 &  = \int \pdv{g}{y} \,\diff y
            %       && = \int \ln x \,\diff y
            %       && = y \ln x
            %       && + C_2(x;z)
            %   \\
            %   g_3 &  = \int \pdv{g}{z} \,\diff z
            %       && = \int \frac{2}{z^2} \,\diff z
            %       && = - \frac{2}{z}
            %       && + C_3(x;y)
            % \end{alignat*}
            % Tudjuk, hogy $g = g_1 = g_2 = g_3$, a konstansok tehát:
            % \[
            %   C_1(y;z) = - \frac{2}{z} + C
            %   \text,\hspace{1cm}
            %   C_2(x;z) = - \frac{2}{z} + C
            %   \text,\hspace{1cm}
            %   C_3(x;y) = y \ln x + C
            %   \text.
            % \]
            Az eredeti függvény az alábbi alakban írható fel:
            \[
              g(\rvec r) =
              \int_0^x G_x(\xi; 0; 0) \,\diff \xi +
              \int_0^y G_y(x; \eta; 0) \,\diff \eta +
              \int_0^z G_z(x; y; \zeta) \,\diff \zeta
              \text.
            \]
            Végezzük el az integrálást:
            \begin{align*}
              g(\rvec r)
               & =
              \int_0^x 0 \,\diff \xi +
              \int_0^y \ln x \,\diff \eta +
              \int_0^z \frac{2}{\zeta^2} \,\diff \zeta
              \\
               & =
              0 + y \ln x - \frac{2}{z} + C
              \text.
            \end{align*}
            Az eredeti függvény tehát:
            \[
              g(x;y;z) = y \ln x - \frac{2}{z} + C
            \]
    \end{enumerate}
  }
\end{exercise}

\end{document}
