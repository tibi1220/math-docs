\documentclass[exercise]{math-standalone}

\begin{document}

\begin{exercise}{%
    Válaszoljuk meg az alábbi kérdéseket az adott függvénnyel kapcsolatban!
  }
  \[
    f(x;y) = \tan\left(\dfrac{x}{y}\right)
  \]
  \begin{enumerate}[a)]
    \item Adjuk meg a függvény $(\pi; 1)$ pontbeli érintősíkjának egyenletét!
    \item Adjuk meg a függvény $(\pi; 1)$ pontbeli $(3;-4)$ irányú érintőjének
          egyenletét!
    \item Melyik pontokban lesz a függvény érintősíkja merőleges az
          $\rvec r = (1; 0; -1)$ irányvektorú egyenesre?
  \end{enumerate}

  \exsol{%
  Az összes részfeladat megoldásához szükségünk van a függvény gradiensére:
  \[
    \grad f(x;y) = \begin{bmatrix}
      \dfrac{1}{y \cos^2(x / y)} &
      \dfrac{-x}{y^2 \cos^2(x / y)}
    \end{bmatrix}^{\mathsf T}
  \]
  \begin{enumerate}[a)]
    \item A parciális deriváltak értéke a $(\pi; 1)$ pontban:
          \[
            \left. \pdv{f(x;y)}{x} \right|_{(\pi; 1)} = 1
            \text, \qquad
            \left. \pdv{f(x;y)}{y} \right|_{(\pi; 1)} = -\pi
            \text.
          \]
          A függvényérték a keresett pontban:
          \[
            f(\pi; 1) = 0 \text.
          \]
          Ezek alapján az érintősík egyenlete:
          \[
            1(x - \pi) - \pi (y - 1) = x - 0
            \text.
          \]
          Az egyszerűsítések után a sík egyenlete:
          \[
            z = x - y \pi
            \text.
          \]
    \item Szükségünk van az adott pontbeli iránymenti deriváltra. Az
          irány normáltja:
          \[
            \uvec v = \frac{1}{\sqrt{3^2 + 4^2}} \begin{bmatrix}
              3 \\ -4
            \end{bmatrix} = \begin{bmatrix}
              3/5 \\ -4/5
            \end{bmatrix}
            \text.
          \]
          A vektor segítségével a keresett derivált értéke:
          \[
            \left.\pdv{f(x;y)}{\uvec v}\right|_{(\pi;1)}=
            \begin{bmatrix}
              1 \\ -\pi
            \end{bmatrix}^{\mathsf T}
            \begin{bmatrix}
              3/5 \\ -4/5
            \end{bmatrix}
            = \frac{3 + 4\pi}{5}
            \text.
          \]
          Az érintő egyenlete ezek alapján:
          \[
            \frac{x - \pi}{3/5} = \frac{y - 1}{-4/5} = \frac{z - 0}{(3 + 4\pi)/5}
            \text.
          \]
    \item Az érintősík normálvektora:
          \[
            \rvec n = \begin{bmatrix}
              \displaystyle\pdv{f}{x} & \displaystyle\pdv{f}{y} & -1
            \end{bmatrix}^{\mathsf T}
            \text.
          \]
          Ebből következik, hogy:
          \[
            \dfrac{1}{y \cos^2(x / y)} = 1
            \text,\qquad
            \dfrac{-x}{y^2 \cos^2(x / y)} = 0
            \text.
          \]
          Az második egyenletből következik, hogy $x = 0$, az elsőből pedig,
          hogy $y = 1$. A keresett pont tehát:
          \[
            (x_\perp; y_\perp) = (0; 1)
            \text.
          \]
  \end{enumerate}
  }
\end{exercise}

\end{document}
