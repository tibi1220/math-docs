\documentclass[exercise]{math-standalone}

\begin{document}

\begin{exercise}{%
    Válaszoljuk meg az alábbi kérdéseket az alábbi függvénnyel kapcsolatban!
  }
  \[
    \rvec f(x;y;z) = \begin{bmatrix}
      x^2 + e^{2z}  \\
      \sin xy + z^6 \\
      xyz
    \end{bmatrix}
  \]
  \begin{enumerate}[a)]
    \item Adjuk meg a függvény Jacobi-mátrixát a $(2; \pi/4; 0)$ pontban!
    \item Adjuk meg az második komponens-függvény $(1; \pi; 1)$ pontbeli
          $\rvec v = (0; 3; 4)$ irányú iránymenti deriváltját!
    \item Adjuk meg a harmadik komponens-függvény $(1;1;1)$ pontbeli
          olyan irányú iránymenti deriváltját, mely irány a komponens-függvény
          legnagyobb növekedésének irányába mutat.
  \end{enumerate}

  \exsol{%
    Határozzuk meg az egyes komponens-függvények gradienseit paraméteresen:
    \begin{align*}
      \grad f_1 & = \left[
        \begin{array}{*{3}{X{2cm}}}
          2x & 0 & 2 e^{2z}
        \end{array}
        \right]^{\mathsf T}
      \text,
      \\
      \grad f_2 & = \left[
        \begin{array}{*{3}{X{2cm}}}
          y \cos xy & x \cos xy & 6 z^5
        \end{array}
        \right]^{\mathsf T}
      \text,
      \\
      \grad f_3 & =\left[
        \begin{array}{*{3}{X{2cm}}}
          yz & xz & xy
        \end{array}
        \right]^{\mathsf T}
      \text.
    \end{align*}

    \begin{enumerate}[a)]
      \item A Jacobi mátrix paraméteresen a gradiensek alapján:
            \[
              \rmat J = \left[
                \begin{array}{*{3}{X{2cm}}}
                  2x        & 0         & 2 e^{2z} \\
                  y \cos xy & x \cos xy & 6 z^5    \\
                  yz        & xz        & xy
                \end{array}
                \right]
              \text.
            \]
            A $(2; \pi/4; 0)$ pontban kiértékelve pedig:
            \[
              \rmat J = \left[
                \begin{array}{*{3}{X{2cm}}}
                  4 & 0 & 2       \\
                  0 & 0 & 0       \\
                  0 & 0 & \pi / 2
                \end{array}
                \right]
              \text.
            \]

      \item Az gradiens adott pontbeli értéke:
            \[
              \grad f_2(1; \pi; 1) =
              \begin{bmatrix}
                -\pi \\ -1 \\ 6
              \end{bmatrix}
              \text.
            \]
            A kijelölt irányba mutató egységvektor:
            \[
              \uvec v = \frac{1}{\sqrt{0^2 + 3^2 + 4^2}}
              \begin{bmatrix}
                0 \\ 3 \\ 4
              \end{bmatrix} =
              \begin{bmatrix}
                0 \\ 3/5 \\ 4/5
              \end{bmatrix}
              \text.
            \]
            A keresett iránymenti derivált értéke az előző két vektor skaláris
            szorzatával egyenlő, vagyis:
            \[
              \left.\pdv{f_2(x;y;z)}{\uvec v}\right|_{(1; \pi; 1)}
              = \begin{bmatrix}
                -\pi \\ -1 \\ 6
              \end{bmatrix}^{\mathsf T} \begin{bmatrix}
                0 \\ 3/5 \\ 4/5
              \end{bmatrix}
              = -3/5 + 6 \cdot 4/5
              = 4,2
              \text.
            \]

      \item A legnagyobb növekmény irányába a gradiens vektor mutat, melynek
            adott pontbeli értéke:
            \[
              \grad f_3(1; 1; 1) =
              \begin{bmatrix}
                1 \\ 1 \\ 1
              \end{bmatrix}
              \text.
            \]
            A keresett derivált tehát:
            \[
              \left.\pdv{f_3(x;y;z)}{\grad f_3|_{(1;1;1)}}\right|_{(1; 1; 1)}
              = \begin{bmatrix}
                1 \\ 1 \\ 1
              \end{bmatrix}^{\mathsf T}\frac{1}{\sqrt 3} \begin{bmatrix}
                1 \\ 1\\ 1
              \end{bmatrix}
              = \sqrt 3
              \text.
            \]
    \end{enumerate}
  }
\end{exercise}

\end{document}
