\documentclass[exercise]{math-standalone}

\begin{document}
\begin{exercise}{%
    Integráljuk az alábbi vektormezőt az első térnyolcadban lévő egységkocka
    felületén kifele mutató irányítással!
  }
  \[
    \rvec v(\coordv) = \ijk{x^2 y z}{x y^2 z}{2 x y z^2}
  \]

  \exsol{%
    A felületi integrál kiszámítása rendkívül hosszadalmas lenne, hiszen a kocka
    felületét 6 különböző felületdarab alkotja, melyek mindegyikén más a
    felületi integrál értéke.
    \\[2mm]
    Alkalmazzuk a Gauss-Osztogradszkij-tételt:
    \[
      \oint_{\partial V} \scalar{\rvec v}{\diff \rvec S}
      =
      \int_V \Div \rvec v \, \diff V
      \text.
    \]

    A vektormező divergenciája:
    \[
      \Div \rvec v
      =
      \pdv{x^2 y z}{x} +
      \pdv{x y^2 z}{y} +
      \pdv{2 x y z^2}{z}
      =
      8 x y z
      \text.
    \]

    Az integrál tehát:
    \[
      \int_0^1 \int_0^1 \int_0^1 8 x y z \, \,\diff x \,\diff y \,\diff z =
      \int_0^1 \int_0^1 4 y z \, \diff y \,\diff z =
      \int_0^1 2 z \, \diff z =
      1
      \text.
    \]
  }
\end{exercise}
\end{document}
