\documentclass[exercise]{math-standalone}

\begin{document}
\begin{exercise}{%
    Integráljuk az alábbi vektormezőt az $A(1;0;0)$, $B(0;2;0)$ $C(0;0;3)$
    csúcsokkal meghatározott háromszögvonal mentén!
  }
  \[
    \rvec v(\coordv) = \ijk{x + y / 2 + z / 3}{y}{z}
  \]

  \exsol{%
    Alkalmazzuk a Stokes-tételt:
    \[
      \oint_{\partial S} \scalar{\rvec v}{\diff \rvec r}
      =
      \int_S \scalar{\rot \rvec v}{\diff \rvec S}
      \text.
    \]

    A vektormező divergenciája:
    \[
      \rot \rvec v
      =
      \begin{bmatrix}
        \partial_x \\ \partial_y \\ \partial_z
      \end{bmatrix}
      \times
      \begin{bmatrix}
        x + y / 2 + z / 3 \\ y \\ z
      \end{bmatrix}
      =
      \begin{bmatrix}
        0 \\ 1/3 \\ -1/2
      \end{bmatrix}
      \text.
    \]

    A felület paraméterezése:
    \[
      \surfv(s;t)
      =
      A + s (B - A) + t (C - A)
      =
      \begin{bmatrix}
        1 - s - t \\ 2s \\ 3t
      \end{bmatrix}
      \text.
    \]

    A felületi normális:
    \[
      \rvec n
      =
      \pdv{\surfv}{s} \times \pdv{\surfv}{t}
      =
      \begin{bmatrix}
        -1 \\ 2 \\ 0
      \end{bmatrix}
      \times
      \begin{bmatrix}
        -1 \\ 0 \\ 3
      \end{bmatrix}
      =
      \begin{bmatrix}
        6 \\ 3 \\ 2
      \end{bmatrix}
      \text.
    \]

    Az integrál tehát:
    \[
      \int_0^1 \int_0^{1 - t}
      \scalar{\rot \rvec v (\surfv(s;t))}{\rvec n}
      \, \diff s \, \diff t
      =
      \int_0^1 \int_0^{1 - t}
      \begin{bmatrix}
        0 \\ 1/3 \\ -1/2
      \end{bmatrix}
      \begin{bmatrix}
        6 \\ 3 \\ 2
      \end{bmatrix}
      \, \diff s \, \diff t
      =
      \int_0^1 \int_0^{1 - t}
      0
      \, \diff s \, \diff t
      =
      0
      \text.
    \]

    Amennyiben nem alkalmazzuk a tételt:
    \begin{enumerate}
      \item $(A) \rightarrow (B)$
            \[
              \arcv(t) = \begin{bmatrix}
                1 - t \\ 2t \\ 0
              \end{bmatrix}
              \qquad
              \dot{\arcv}(t) = \begin{bmatrix}
                -1 \\ 2 \\ 0
              \end{bmatrix}
              \qquad
              \rvec v(\rvec r(t)) = \begin{bmatrix}
                1 - t + t \\ 2t \\ 0
              \end{bmatrix} = \begin{bmatrix}
                1 \\ 2t \\ 0
              \end{bmatrix}
            \]
            \[
              S_1 = \int_0^1 4t - 1 \, \diff t = \left[ 2t^2 - t \right]_0^1 = 1
            \]

      \item $(B) \rightarrow (C)$
            \[
              \arcv(t) = \begin{bmatrix}
                0 \\ 2 - 2t \\ 3t
              \end{bmatrix}
              \qquad
              \dot{\arcv}(t) = \begin{bmatrix}
                0 \\ -2 \\ 3
              \end{bmatrix}
              \qquad
              \rvec v(\rvec r(t)) = \begin{bmatrix}
                1 - t + t \\ 2 - 2t \\ 3t
              \end{bmatrix} = \begin{bmatrix}
                1 \\ 2 - 2t \\ 3t
              \end{bmatrix}
            \]
            \[
              S_2 = \int_0^1 13t - 4 \, \diff t = \left[ 6.5t^2 - 4t \right]_0^1 = 2,5
            \]

      \item $(C) \rightarrow (A)$
            \[
              \arcv(t) = \begin{bmatrix}
                t \\ 0 \\ 3 - 3t
              \end{bmatrix}
              \qquad
              \dot{\arcv}(t) = \begin{bmatrix}
                1 \\ 0 \\ -3
              \end{bmatrix}
              \qquad
              \rvec v(\rvec r(t)) = \begin{bmatrix}
                1 - t + t \\ 0 \\ 3 - 3t
              \end{bmatrix} = \begin{bmatrix}
                1 \\ 0 \\ 3 - 3t
              \end{bmatrix}
            \]
            \[
              S_3 = \int_0^1 9t - 8 \, \diff t = \left[ 4.5t^2 - 8t \right]_0^1 = -3,5
            \]
    \end{enumerate}
    Látható, hogy ugyan azt az eredményt kaptuk:
    \[
      S_1 + S_2 + S_3 = 1 + 2,5 - 3,5 = 0
      \text  \, \diff t = \left[ 9t - 4.5t^2 \right]_0^1 = 4,5.
    \]
  }
\end{exercise}
\end{document}
