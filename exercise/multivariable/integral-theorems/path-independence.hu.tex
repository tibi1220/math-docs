\documentclass[exercise]{math-standalone}

\begin{document}
\begin{exercise}{%
    Integráljuk a $\rvec v(\coordv) = \ijk{x}{y}{z}$ vektormezőt az alábbi
    görbék mentén!
  }
  \begin{enumerate}
    \item
  \end{enumerate}

  \exsol{%
    Vizsgáljuk meg, hogy a vektormező vektorpotenciálos-e:
    \[
      \rot \rvec v
      =
      \begin{bmatrix}
        \partial_x \\ \partial_y \\ \partial_z
      \end{bmatrix}
      \times
      \begin{bmatrix}
        y + z \\ x + z \\ x + y
      \end{bmatrix}
      =
      \begin{bmatrix}
        \partial_x (x + z) - \partial_y (y + z) \\
        \partial_y (x + y) - \partial_z (x + z) \\
        \partial_z (y + z) - \partial_x (x + y)
      \end{bmatrix}
      =
      \begin{bmatrix}
        1 - 1 \\ 1 - 1 \\ 1 - 1
      \end{bmatrix}
      =
      \nvec
      \text.
    \]

    Mivel a vektormező vektorpotenciálos, ezért létezik olyan skalármező,
    melynek gradiense maga a $\rvec v$ vektormező. Az integrál értéke tehát
    csak a kezdő- és végpontoktól függ, melyek jelen esetben megegyeznek,
    vagyis az integrál értéke zérus:
    \[
      \oint_\gamma \scalar{\rvec v}{\diff \arcv} = 0
      \text.
    \]
  }
\end{exercise}
\end{document}
