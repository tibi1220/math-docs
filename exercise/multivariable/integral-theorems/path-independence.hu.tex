\documentclass[exercise]{math-standalone}

\begin{document}
\begin{exercise}{%
    Integráljuk a megadott vektormezőt a (0;1) és (2;0) pontok között az alábbi
    görbék mentén!
  }
  \[
    \rvec v (\coordv) = \ijk{y}{x}{0}
  \]
  \begin{enumerate}[a)]
    \item egyenes szakasz mentén
    \item origó középpontú ellipszis mentén
  \end{enumerate}

  \exsol{%
    Vizsgáljuk meg, hogy a vektormező vektorpotenciálos-e:
    \[
      \rot \rvec v
      =
      \begin{bmatrix}
        \partial_x \\ \partial_y \\ \partial_z
      \end{bmatrix}
      \times
      \begin{bmatrix}
        y \\ x \\ 0
      \end{bmatrix}
      =
      \begin{bmatrix}
        0 \\ 0 \\ 1 - 1
      \end{bmatrix}
      =
      \nvec
      \text.
    \]

    Mivel a vektormező vektorpotenciálos, ezért létezik olyan skalármező,
    melynek gradiense maga a $\rvec v$ vektormező.
    \\[2mm]
    A potenciálfüggvényt a következőképpen határozhatjuk meg:
    \[
      V(\coordv)
      =
      \int_0^x y \, \diff \xi
      + \int_0^y 0 \, \diff \xi
      + \int_0^z 0 \, \diff \xi
      =
      xy + C
      \text.
    \]

    Az integrál értéke tehát csak a kezdő- és végpontoktól függ, tehát mindhárom
    esetben az integrál értéke megegyezik:
    \[
      \int_\gamma \scalar{\rvec v (\coordv)}{\diff \arcv} = V(2;0) - V(0;1) = 0
      \text.
    \]

    Ha nem vettük volna észre, hogy a függvényünk skalárpotenciális:
    \begin{enumerate}[a)]
      \item egyenes szakasz mentén:
            \[
              \rvec r(t) = \begin{bmatrix}
                1 \\ 0
              \end{bmatrix} + t \begin{bmatrix}
                0 - 1 \\ 2 - 0
              \end{bmatrix} = \begin{bmatrix}
                1 - t \\ 2t
              \end{bmatrix}
              \text, \quad
              t \in [0;1]
              \text, \quad
              \dot{\rvec r}(t) = \begin{bmatrix}
                -1 \\ 2
              \end{bmatrix}
            \]
            \[
              % \int_0^1 \scalar{\rvec w(\rvec r(t))}{\dot{\rvec r}(t)} \diff t
              \int_0^1 \begin{bmatrix}
                2t \\ 1 - t
              \end{bmatrix}^\transpose
              \begin{bmatrix}
                -1 \\ 2
              \end{bmatrix}
              \diff t
              = \int_0^1 (-2t + 2 - 2t) \diff t
              = \int_0^1 2 - 4t \, \diff t
              = \left[ 2t - 2t^2 \right]_0^1
              = 0
            \]

      \item ellipszis mentén:
            \[
              \rvec r(t) = \begin{bmatrix}
                \cos t \\ 2 \sin t
              \end{bmatrix}
              \text,\quad
              t \in [0; \pi/2]
              \text,\quad
              \dot{\rvec r}(t) = \begin{bmatrix}
                - \sin t \\ 2 \cos t
              \end{bmatrix}
            \]
            \[
              % \int_{\pi / 2}^0 \scalar{\rvec w(\rvec r(t))}{\dot{\rvec r}(t)} \diff t
              \hspace{-1.3cm}
              \int_0^{\pi / 2} \begin{bmatrix}
                2 \sin t \\ \cos t
              \end{bmatrix}^\transpose
              \begin{bmatrix}
                - \sin t \\ 2 \cos t
              \end{bmatrix}
              \diff t
              = 2 \int_0^{\pi / 2} (\cos^2 t - \sin^2 t) \, \diff t
              = 2 \int_0^{\pi / 2} \cos 2t \, \diff t
              = 2 \left[ \frac{\sin 2t}{2} \right]_0^{\pi / 2}
              = 0
            \]
    \end{enumerate}
  }
\end{exercise}
\end{document}
