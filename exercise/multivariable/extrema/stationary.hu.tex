\documentclass[exercise]{math-standalone}

\begin{document}

\begin{exercise}{%
    Keressük meg az alábbi függvények stacionárius pontjait! Írjuk fel az adott
    pontbeli Hesse-mátrixokat, valamint adjuk meg a pontok típusait!
  }
  \begin{enumerate}[a)]
    \item $f(x; y) = x^2 + 2xy - y^3$
    \item $g(x; y) = -x^2 + 2xy + y^3$
  \end{enumerate}

  \exsol{%
    \begin{enumerate}[a)]
      \item $f(x; y) = x^2 + 2xy - y^3$\\[2mm]
            %
            Állítsuk elő a függvény elsőrendű parciális deriváltjait:
            \[
              \pdv{f}{x} = 2x + 2y
              \text,
              \qquad
              \pdv{f}{y} = 2x - 3y^2
              \text.
            \]
            Stacionárius pont ott található, ahol a függvény összes változó
            szerinti parciális deriváltjai eltűnnek, vagyis:
            \[
              2x + 2y = 0
              \text,
              \qquad
              2x - 3y^2 = 0
              \text.
            \]
            Oldjuk meg az egyenletrendszert:
            \[
              -2y - 3y^2 = 0
              \quad \rightarrow \quad
              y(-2 - 3y) = 0
              \text.
            \]
            A megoldások:
            \[
              (x_1; y_1) = (0; 0)
              \text, \qquad
              (x_2; y_2) = (2/3; -2/3)
              \text.
            \]

            Határozzuk meg a függvény másodrendű parciális deriváltjait:
            \[
              \pdv[order=2]{f}{x} = 2 \text,
              \qquad
              \pdv[order=2]{f}{y} = -6y \text,
              \qquad
              \pdv{f}{x,y}        = 2 \text.
            \]

            A Hesse-mátrix paraméteresen:
            \[
              \rmat H = \begin{bmatrix}
                2 & 2 \\ 2 & -6y
              \end{bmatrix}
              \text.
            \]

            \begin{enumerate}[1)]
              \item Vizsgáljuk meg a mátrix determinánsát $(0; 0)$ pontban:
                    \[
                      \det \rmat H(0;0) = \begin{vmatrix}
                        2 & 2 \\ 2 & 0
                      \end{vmatrix} = -4 \text.
                    \]
                    Mivel a determináns negatív, ezért a stacionárius pont
                    egy nyeregpont.

              \item Most vizsgáljuk meg a mátrix determinánsát $(2/3; -2/3)$
                    pontban:
                    \[
                      \det \rmat H(2/3;-2/3) = \begin{vmatrix}
                        2 & 2 \\ 2 & 4
                      \end{vmatrix} = 4 \text.
                    \]
                    Mivel a determináns pozitív, ezért a stacionárius pont
                    egy lehetséges szélsőérték hely. Mivel a Hesse-mátrix
                    összes aldeterminánsa pozitív, ($\det \rmat H > 0$,
                    $h_{11} > 0$,) ezért ebben a pontban a függvénynek lokális
                    minimuma van.
            \end{enumerate}

            Összegezve tehát:
            \begin{alignat*}{9}
              f & (0  ; 0  )  &  & = 0 \quad     &  & \rightarrow \quad
              \text{nyeregpont,}
              \\
              f & (2/3; -2/3) &  & = -4/27 \quad &  & \rightarrow \quad
              \text{lokális minimum.}
            \end{alignat*}

      \item $g(x; y) = -x^2 + 2xy + y^3$\\[2mm]
            %
            Állítsuk elő a függvény elsőrendű parciális deriváltjait:
            \[
              \pdv{g}{x} = -2x + 2y
              \text,
              \qquad
              \pdv{g}{y} = 2x + 3y^2
              \text.
            \]
            Stacionárius pont ott található, ahol a függvény összes változó
            szerinti parciális deriváltjai eltűnnek, vagyis:
            \[
              -2x + 2y = 0
              \text,
              \qquad
              2x + 3y^2 = 0
              \text.
            \]
            Oldjuk meg az egyenletrendszert:
            \[
              2y + 3y^2 = 0
              \quad \rightarrow \quad
              y(2 + 3y) = 0
              \text.
            \]
            A megoldások:
            \[
              (x_1; y_1) = (0; 0)
              \text, \qquad
              (x_2; y_2) = (-2/3; -2/3)
              \text.
            \]

            Határozzuk meg a függvény másodrendű parciális deriváltjait:
            \[
              \pdv[order=2]{g}{x} = -2 \text,
              \qquad
              \pdv[order=2]{g}{y} = 6y \text,
              \qquad
              \pdv{g}{x,y}        = 2 \text.
            \]

            A Hesse-mátrix paraméteresen:
            \[
              \rmat H = \begin{bmatrix}
                -2 & 2 \\ 2 & 6y
              \end{bmatrix}
              \text.
            \]

            \begin{enumerate}[1)]
              \item Vizsgáljuk meg a mátrix determinánsát $(0; 0)$ pontban:
                    \[
                      \det \rmat H(0;0) = \begin{vmatrix}
                        -2 & 2 \\ 2 & 0
                      \end{vmatrix} = -4 \text.
                    \]
                    Mivel a determináns negatív, ezért a stacionárius pont
                    egy nyeregpont.

              \item Most vizsgáljuk meg a mátrix determinánsát $(2/3; -2/3)$
                    pontban:
                    \[
                      \det \rmat H(-2/3;-2/3) = \begin{vmatrix}
                        -2 & 2 \\ 2 & -4
                      \end{vmatrix} = 4 \text.
                    \]
                    Mivel a determináns pozitív, ezért a stacionárius pont
                    egy lehetséges szélsőérték hely. Mivel a Hesse-mátrix
                    aldeterminánsainak előjele váltakozik, ($\det \rmat H > 0$,
                    $h_{11} < 0$,) ezért ebben a pontban a függvénynek
                    lokális maximuma van.
            \end{enumerate}

            Összegezve tehát:
            \begin{alignat*}{9}
              g & (0  ; 0  )   &  & = 0 \quad    &  & \rightarrow \quad
              \text{nyeregpont,}
              \\
              g & (-2/3; -2/3) &  & = 4/27 \quad &  & \rightarrow \quad
              \text{lokális maximum.}
            \end{alignat*}

    \end{enumerate}
  }
\end{exercise}

\end{document}
