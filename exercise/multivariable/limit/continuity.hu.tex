\documentclass[exercise]{math-standalone}

\begin{document}

\begin{exercise}{Folytonosak-e az alábbi függvények az origóban?}
  % \begin{multicols}{2}
  \begin{enumerate}[a)]
    \item $f(x; y) = \begin{cases}
              \dfrac{2\,x\,y}{\sqrt{x^2 + y^2}} & \text{ ha } (x;y) \neq (0;0) \\
              0                                 & \text{ ha } (x;y) = (0;0)
            \end{cases}$

    \item $g(x) = \begin{cases}
              \dfrac{x^3 - y^3}{y - x} & \text{ ha } y - x \neq 0 \\
              0                        & \text{ ha } y - x = 0
            \end{cases}$
  \end{enumerate}
  % \end{multicols}

  \exsol{
    \begin{enumerate}[a)]
      \item $f(x; y) = \dfrac{2\,x\,y}{\sqrt{x^2 + y^2}}$\\[2mm]
            %
            A függvény folytonos, ha az adott pontban vett határértéke
            megegyezik az adott pontbeli függvényértékkel, vagyis:
            \[
              \lim_{\rvec x \rightarrow \rvec a} f(\rvec x) = f(\rvec a)
              \text.
            \]
            Mivel az origóban vagyunk kíváncsiak $f$ határérékére, azért
            élhetünk az alábbi helyettesítéssel:
            \[
              x := r \cos\varphi
              \text,\qquad
              y := r \sin\varphi
              \text.
            \]
            Végezzük el a helyettesítéseket:
            \[
              \lim_{\rvec x \rightarrow \nvec} f(\rvec x) =
              \lim_{r \rightarrow 0} \frac{
                2 r \cos(\varphi) r \sin(\varphi)
              }{
                \sqrt{(r \cos\varphi)^2 + (r \sin\varphi)^2}
              }
              \text.
            \]
            Végezzük el az egyszerűsítéseket:
            \[
              \lim_{r \rightarrow 0} \frac{
                2 r^2 \cos \varphi \sin \varphi
              }{
                r
              } =
              \lim_{r \rightarrow 0} 2r \cos\varphi \sin\varphi
              = 0
              \text.
            \]
            Mivel a $2 \cos \varphi \sin \varphi$ szorzat korlátos, ezért az
            adott pontbeli határérték 0, amely megegyezik az adott pontbeli
            függvényértékkel, vagyis a függvény \textbf{folytonos}.

      \item $g(x; y) = \dfrac{x^3 - y^3}{{y - x}}$\\[2mm]
            %
            Egyszerűsítsük a törtfüggvényt:
            \[
              \frac{x^3 - y^3}{y - x} =
              \frac{(x - y)(x^2 + xy + y^2)}{-(x - y)} =
              -(x^2 + xy + y^2)
              \text.
            \]
            A határérték tehát az origóban:
            \[
              \lim_{\rvec x \rightarrow \nvec} -(x^2 + xy + y^2)
              = -(0 + 0 + 0)
              = 0
              \text.
            \]
            Mivel az origóban felvett függvényérték megegyezik az adott
            pontbeli határértékkel, ezért a függvény \textbf{folytonos} a
            $(0;0)$ pontban.
    \end{enumerate}
  }
\end{exercise}

\end{document}
